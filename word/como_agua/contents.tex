\section*{ Capítulo 1 }
\subsection*{ Enero }
\subsection*{ Tortas de Navidad }
\textbf{Ingredientes}

\begin{itemize}
\item Una lata de sardinas en jitomate\footnote{die Tomate}
\item ½ de chorizo
\item Una cebolla
\item Orégano
\item Una lata de chiles serranos
\item 10 teleras\footnote{das Brot}
\end{itemize}

\section*{ Manera de hacerse:}
La cebolla~\sidenote{11}tiene que estar finamente picada. Les sugiero ponerse
en un pequeño trozo de cebolla en la mollera\footnote{der Grips, der Verstand}
con el fin de evitar el molesto lagrimeo que se produce cuando uno
la está cortando. Lo malo de llorar cuando una pica cebolla no es
el simple hecho de llorar, sino que a veces uno empieza, como quien dice,
se pica, y ya no puede parar. No sé si a ustedes les ha pasado pero a
mí la mera\footnote{rein, einfach} verdad sí. Infinidad de veces.
Mamá decía que era porque yo soy igual de sensible a la cebolla que
Tita, mi tía abuela.

Dicen que Tita era tan sensible que desde que estaba en
el vientre\footnote{der Bauch, der Unterleib} de mi
bisabuela lloraba y lloraba cuando esta picaba cebolla; su llanto era
tan fuerte que Nacha, la cocinera de la casa, que era medio sorda, lo
escuchaba sin esforzarse. Un día los sollozos fueron tan fuertes que
provocaron que el parto\footnote{die Geburt, die Entbindung} se
adelantara. Y sin que mi abuela pudiera decir ni pío, Tita arribó\footnote{die Ankunft, landen, ankommen} a
este mundo prematuramente\footnote{vorzeitig, frühzeitig},
sobre la mesa de la cocina, entre los olores de una sopa de fideos que
se estaba cocinando, los del tomillo\footnote{der Thymian}, el laurel\footnote{der Lorbeer}, el cilantro, el de la leche hervida,
el de los ajos y, por supuesto, el de la cebolla. Como se imaginarán, la consabida\footnote{alltäglich, bekannt, klassisch, üblich}~\sidenote{12}nalgada\footnote{auf den Hintern, Glücksklaps auf den Arsch} no
fue necesaria pues Tita nació llorando de antemano\footnote{im Voraus, vorneweg}, tal vez porque ella sabía
que es un oráculo determinaba que en esta vida le estaba negado el
matrimonio. Contaba Nacha que Tita fue literalmente empujada a este mundo
por un torrente\footnote{der Sturzbach, der Schwall} impresionante
lágrimas que se desbordaron sobre la mesa y el piso de la cocina.

En la tarde, ya cuando el susto había pasado y el agua, gracias al
efecto de los rayos del sol, se había evaporado\footnote{sich verflüchtigen, verdampfen, verdunsten},
Nacha barrió\footnote{etw. fegen, etw. kehren} el residuo\footnote{das Abfallprodukt, der Rückstand} de las lágrimas
que había quedada sobre la loseta\footnote{die Wandplatte, die Fliese}
roja que cubría el piso. Con esta sal rellenó un costal\footnote{das Querholz, der Sack} de cinco kilos que
utilizaron para cocinar por bastante tiempo. Este inusitado\footnote{ungewöhnlich, unüblich}
nacimiento determinó el hecho de que Tita sintiera un inmenso amor pues
la cocina y que la mayor parte de su vida la pasaba en ella, prácticamente
desde que nació, pues cuando contaba con dos días de edad, su padre, o sea
mi bisabuelo, murió de un infarto. A Mamá Elena, de la impresión, se le
fue la leche. Como en esos tiempos no había leche en polvo ni nada que se
le pareciera, y no pudieron conseguir nodriza\footnote{die Amme}
por ningún lado, se vieron en un verdadero lío para calmar el hambre de la
niña. Nacha, que se las sabía de todas todas respecto a la cocina - y a muchas
otras cosas que ahora no vienen al caso -, se ofreció a hacerse cargo de
la alimentación de Tita. Ella se consideraba la más capacitada para
\glqq{}formarle el estómago a la inocente criaturita\grqq{}, a pesar de que nunca se
casó ni tuvo hijos. Ni siquiera sabía leer ni escribir, pero eso sí: sobre
cocina tenía tan profundos conocimientos como la que más. Mamá Elena aceptó
con agrado\footnote{die Freundlichkeit, jmdm. gefallen, jmdm. behagen}~\sidenote{13}la sugerencia\footnote{der Tipp, der Vorschlag, die Anregung}
pues bastante tenía ya con la tristeza y la enorme responsabilidad de manejar\footnote{mit jmdm./etw. umgehen, etw. handhaben, zurechtkommen, sich zu helfen wissen}
correctamente el rancho, para así poderles
dar a sus hijos la alimentación y educación que se merecían, como para
encima tener que preocuparse por nutrir debidamente a la recién nacida.

Por tanto, desde ese día, Tita se mudó a la cocina y entre atoles\footnote{warmes Getränk aus Maismehl und Milch oder Wasser}
y tés creció de lo más sana y rozagante\footnote{stolz, ansehlich, prächtig}.
Es de explicarse entonces el que se le haya desarrollado un sexto sentido
en todo lo que a comida se refiere. Por ejemplo, sus hábitos alimenticios
estaban condicionados al horario de la cocina: cuando a la mañana Tita
olía que los frijoles ya están cocidos, o cuando a medio día sentía que
el agua ya estaba lista para desplumar a las gallinas, cuando en la tarde
se horneaba el pan para la cena, ella sabía que había llegado la hora de
pedir sus alimentos.

Algunas veces lloraba de balde\footnote{der Kübel, der Eimer, gratis, kostenlos, unentgeltlich},
como cuando Nacha picaba cebolla, pero como las dos sabían la razón de
esas lágrimas, no se tomaban en serio.
Inclusive se convertían en motivo de diversión, a tal grado\footnote{die Stufe, der Rang}
que durante su niñez Tita no diferenciaba bien las lágrimas de la risa de las de
llanto. Para ella reír era una manera de llorar.

De igual forma confundía el gozo\footnote{die Freude, die Wonne}
de vivir con el de comer. No era fácil para una persona que conoció la
vida a través de la cocina entender el mundo exterior. Ese gigantesco
mundo que empezaba de la puerta de la cocina hacia el interior de la casa,
porque el que colindaba\footnote{an etw. grenzen} con
la puerta trasera de la cocina y que daba al patio, a la huerta, a la hortaliza\footnote{grünes Gemüse, Frühgemüse}, sí le pertenecía
por completo, lo dominaba. Todo lo contrario de sus hermanas, a quienes
este~\sidenote{14}mundo les atemorizaba\footnote{sich ängstigen, sich erschrecken}
y encontraba lleno de peligros incógnitos. Les parecían absurdos y
arriesgados los juegos dentro de la cocina, sin embargo, un día Tita
las convenció de que era un espectáculo asombroso el ver cómo bailaban
las gotas agua al caer sobre el comal\footnote{flache, runde Kochplatte aus Ton}
bien caliente.

Pero mientras Tita cantaba y sacudía\footnote{jmdn./etw. durchrütteln}
rítmicamente sus manos mojadas para que las gotas de agua se precipitaran
sobre el comal y \glqq{}danzaran\grqq{}, Rosaura permanecía en un rincón, pasmada\footnote{verblüfft, verdattert} por lo que observaba. En
cambio Gertrudis, como todo aquello donde interviniera, el ritmo, el movimiento
o la música, se vio fuertemente atraída hacía el juego y se integró con
entusiasmo. Entonces a Rosaura no le quedó otra que tratar de hacer lo
propio, pero como casi no se mojó las manos y lo hacía con tanto miedo,
no logró el efecto deseado. Tita entonces trató de ayudarla acercándole
las manos al comal. Rosaura se resistió y esta lucha no paró hasta que
Tita, muy enojada, le soltó las manos y estás, por inercia\footnote{die Trägheit},
cayeron sobre el ardiente comal. Además de ganarse una soberana\footnote{souverän, eine ordentliche Tracht Prügel} paliza\footnote{die Tracht Prügel, die Schläge}, Tita quedó privadas
de jugar con sus hermanas dentro de su mundo. Entonces Nacha se convirtió
en su compañera de diversión. Juntas se dedicaban a inventar juegos y
actividades siempre en relación con la cocina. Como el día en que vieron
en la plaza del pueblo a un señor que formaba figuras de animales con
globos alargados y se les ocurrió repetir el mecanismo pero utilizando
trozos de chorizo. Armaron no solo animales conocidos sino que además
inventaron algunos con cuello de cisne\footnote{der Schwan},
patas de perro y cola de caballo, por citar solamente algunos.

El problema surgía~\sidenote{15}cuando tenían que deshacerlos para freír el
chorizo. La mayoria de las veces Tita se negaba. La única manera en que accedía\footnote{einwilligen in, acceder a} voluntariamente\footnote{freiwillig} a hacerlo y era cuando
se trataba de elaborar las tortas de Navidad, pues le encantaban.
Entonces no solo permitía que se desbaratara\footnote{etw. zunichtemachen, etw. verschwenden, etw. durchbringen}
a uno de sus animales, sino que allegramente observaba cómo se freía.

Hay que tener cuidado de freír el chorizo para las tortas y fuego muy
lento, para que de esta manera quede bien cocido, pero sin dorarse
excesivamente. En cuanto está listo se retira del fuego y se le
incorporan las sardinas, a las que con anterioridad se les ha despejado\footnote{etw. abräumen, etw. entrümpeln, etw. frei machen}
del esqueleto. Es necesario, también, rasparles\footnote{etw. abkratzen, etw. abschaben}
con un cuchillo las manchas negras que tienen sobre la piel.
Junto con las sardinas se mezclan la cebolla, los chiles picados y
el orégano molido\footnote{gemahlen}. Se deja reposar\footnote{sich ausruhen, ruhen, schlafen} la preparación,
antes de rellenar las tortas.

Tita gozaba enormemente este paso ya que mientras reposa el relleno es
muy agradable gozar del olor que despide, los olores tienen la
característica de reproducir tiempos pasados junto con sonidos y olores
nunca igualados\footnote{gleich sein} en el presente. A Tita le
gustaba hacer una gran inhalación y viajar junto con el humo y el olor
tan peculiar\footnote{eigenartig, besonders, speziell} que
percibía hacia los recovecos\footnote{die Krümmung} de su
memoria.
 Vanamente\footnote{abergläubisch, grundlos, selbstgefällig, vergeblich}
trataba de evocar\footnote{etw. hervorrufen, etw. ins Gedächtnis rufen}
la primera vez que olió una de estas tortas, sin resultados, porque tal
vez fue antes de que naciera. Quizá la rara combinación te las sardinas
con el chorizo llamó tanto su atención que la hizo decidirse a renunciar\footnote{sich etw. verkneifen, etw. entsagen, etw. (sein) lassen}
a la paz del éter, escoger\footnote{jmdn./etw. aus (oder: unter) jmdm./etw. auswählen, jmdn./etw. aus (oder: unter) jmdm./etw. aussuchen}
el vientre de Mamá Elena para que fuera a su madre, y de esta manera
ingresar a la~\sidenote{16}familia de la Garza, que comía tan deliciosamente
y que preparaba un chorizo tan especial.

En el rancho de Mamá Elena la preparación del chorizo era todo un rito.
Con un día de anticipación\footnote{der Vorgriff, die Vorverlegeung}
se tenían que empezar a pelar ajos, limpiar chiles y moler especias.
Todas las mujeres de la familia tenían que participar: Mamá Elena, sus hijas
Gertrudis, Rosaura y Tita, Nacha la cocinera y Chencha la sirvienta. Se
sentaban por las tardes en la mesa del comedor y entre pláticas\footnote{der Schwatz} y bromas el tiempo se iba volando
hasta que empezaba a oscurecer. Entonces Mamá Elena decía:
\\- Por hoy ya terminamos con esto. \\
\rule{1em}{0pt}Dicen que el buen entendedor\footnote{der Kenner}
pocas palabras, así que después de escuchar esta frase todas sabían que
era lo que tenía que hacer. Primero recogían la mesa y después se repartían
las labores: una metía a las gallinas, otras sacaba agua de pozo y la
dejaba lista para utilizarla en el desayuno y otra se encargaba\footnote{sich um jmdn./etw. kümmern, etw. übernehmen}
de leña\footnote{das Brennholz, das Reisig} para la estufa\footnote{der Ofen, die Ofenheizung}. Ese día ni se
planchaba ni se bordaba\footnote{(etw.) sticken} ni se cosía
ropa. Después todas se iban a sus recámaras\footnote{die Kleiderkammer, das Schlafzimmer (lat. am.)}
a leer, rezar\footnote{beten} y dormir. Una de esas tardes, antes de que Mamá
Elena dijera que ya se podían levantar de la mesa, Tita, que entonces
contaba con quince años, le anunció con voz temblorosa que Pedro Muzquiz
quería venir a hablar con ella \ndots
\\- ¿Y de que me tiene que venir a hablar ese señor? - dijo Mamá Elena %
luego de un silencio interminable que encogió\footnote{etw. zusammenziehen} %
el alma de Tita. \\
\rule{1em}{0pt}Con voz apenas perceptible respondió:
\\- Yo no sé. %
Mamá Elena le lanzó~\sidenote{17}una mirada que para Tita encerraba todos los %
años de represión que habían flotado sobre la familia y dijo: %
\\- Pues más vale que le informes que si es para pedir tu mano, no lo %
haga. Perdería su tiempo y me haría perder el mío. Sabes muy bien que %
por ser la más chica de las mujeres a ti te corresponde cuidarme hasta %
el día de mi muerte.\\

Dicho esto, Mamá Elena se puso lentamente de pie, guardó sus lentes\footnote{die Brille} dentro de delantal\footnote{die Schürze} y a manera de orden final
repitió:
\\- ¡Por hoy hemos terminado con esto! \\
\rule{1em}{0pt}Tita sabía que dentro de las normas de comunicación de la casa no está
incluido el diálogo, pero, aún así, por primera vez en su vida intentó
protestar a un mandato\footnote{die Amtzeit, das Mandat, der Befehl}
de su madre:
\\- Pero es que yo opino que \ndots %
\\- ¡Tú no opinas nada y se acabó! Nunca, por generaciones, nadie en mi %
familia ha protestado ante esta costumbre y no va a ser una de mis hijas %
quien lo haga. \\
\rule{1em}{0pt}Tita bajó la cabeza y con la misma fuerza con que sus lágrimas cayeron
sobre la mesa, así cayó sobre su destino. Y desde ese momento supieron
ella y la mesa que no podían modificar ni tantito la dirección de estas
fuerzas desconocidas que las obligaban, a la una a compartir con Tita
su sino\footnote{das Schicksal, das Verhängnis}, recibiendo sus
amargas lágrimas desde el momento en que nació, y a la otra
a asumir\footnote{etw. annehmen, etw. übernehmen, etw. verkraften}
esta absurda determinación.
Sin embargo, Tita no estaba conforme. Una gran cantidad de dudas e
inquietudes acudía a su mente.
Por ejemplo, la agradaría~\sidenote{18}tener conocimiento de quién había
iniciado esta tradición familiar. Sería bueno hacerle saber a esta
ingeniosa persona que en su perfecto plan para asegurar la vejez\footnote{die Betagtheit, der Lebensabend} de las mujeres
había una ligera falla. Si Tita no podía casarse ni tener hijos,
¿quién la cuidaría entonces a llegar a la senectud\footnote{das Greisenalter}?
¿Cuál era la solución acertada\footnote{treffend, richtig, ratsam}
en estos casos? ¿O es que no se esperaba que las hijas que se quedaban
a cuidar a sus madres sobrevivieran mucho tiempo después del fallecimiento
de sus progenitoras\footnote{das Elternteil, die Mutter}?
¿Y dónde se quedaban las mujeres que se casaban y no podían tener hijos,
quién se encargaría de atenderlas? Es más, quería saber, ¿cuáles fueron
las investigaciones que se llevaron a cabo para concluir\footnote{etw. vollenden} que la hija menor era la más
indicada para velar\footnote{jmdn./etw. bewachen, etw. verschleiern}
a por su madre y no la hija mayor? ¿Se había tomado alguna vez en cuenta
la opinión de las hijas afectadas? ¿Le estaba permitido al menos, si es
que no se podía casar, el conocer el amor? ¿O ni siquiera eso?

Tita sabía muy bien que todas estas interrogantes\footnote{die Frage, das Fragezeichen::jmdn. befragen, jmdn. ausfragen}
tenían que pasar irremediablemente\footnote{unvermeidbar, nicht zu verhindern}
a formar parte del archivo de preguntas sin respuestas. En la familia de
la Garza si obedecía y punto. Mamá Elena, ignorándola por completo, salió
muy enojado de la cocina y por una semana no le dirigió la palabra.

La reanudación\footnote{die Wiederaufnahme, der Neustart}
de esta semicomunicación se originó\footnote{etw. verursachen, etw. herbeiführen}
cuando, al revisar los vestidos que cada una de las mujeres había estado
cosiendo, Mamá Elena descubrió que aún cuando el confeccionado\footnote{angefertigt, etw. anfertigen, etw. herstellen}
por Tita era el más perfecto no lo había hilvanando\footnote{etw. entwerfen, etw. heften}
antes de coserlo.

- Te~\sidenote{19}felicito\footnote{jmdm. zu etw. gratulieren,  sich über etw. freuen} - le dijo -, %
las puntadas\footnote{die Naht} son perfectas, pero no lo %
hilvanaste, ¿verdad? %
\\- No - respondió Tita, asombrada de que le hubieran levantado la ley del %
silencio. %
\\- Entonces lo vas a tener que deshacer. Lo hilvanas, lo coses nuevamente %
y después vienes a que te lo revise. Para que recuerdes que el flojo\footnote{lasch, schlapp} %
y el mezquino\footnote{geizig, dürftig} andan doble su camino. %
\\- Pero eso es cuando uno se equivoca y usted mismo dijo hace un momento %
que el mío era \ndots %
\\- ¿Vamos a empezar otra vez con la rebeldía? Ya bastante tenías con la de %
haberte atrevido a coser rompiendo las reglas. %
\\- Perdóneme, mami. No lo vuelvo a hacer. \\
\rule{1em}{0pt}Tita logró con estas palabras calmar el enojo de Mamá Elena. Había
puesto mucho cuidado al pronunciar el \glqq{}mami\grqq{} en el momento y con el tono
adecuados. Mamá Elena opinaba que la palabra \glqq{}mamá\grqq{} sonaba despectiva, así
que obligó a sus hijas desde niñas a utilizar la palabra \glqq{}mami\grqq{} cuando
se dirigieran a ella. La única que se resistía o que pronunciaba la
palabra en un tono inadecuado era Tita, motivo por el cual había
recibido infinidad de bofetadas. ¡Pero qué bien lo había hecho en ese
momento! Mamá Elena se sentía reconfortada\footnote{jmdn. trösten, jmdm. wieder Mut zusprechen}
con el pensamiento de que tal vez ya estaba logrando doblegar\footnote{etw. biegen, beugen, krümmen} el carácter de
la más pequeña de sus hijas. Pero desgraciadamente albergó esta esperanza
por muy poco tiempo pues al día siguiente se presentó en casa Pedro Muzquiz
acompañado de su señor padre con la intención de pedir la mano de Tita.
Su presencia en la casa causó gran~\sidenote{20}desconcierto\footnote{die Bestürztheit, die Verwirrung, die Ratlosigkeit}.
No esperaban su visita. Días antes, Tita le había mandado a Pedro un recado\footnote{die Botschaft, der Botengang, die Besorgung}
con el hermano de Nacha pidiéndole que desistiera\footnote{auf etw. verzichten, von etw. ablassen, etw. aufgeben}
de sus propósitos\footnote{die Absicht, der Vorsatz}. Aquel
juró que se lo había entregado a don Pedro, pero el caso es que ellos
se presentaron en la casa. Mamá Elena los recibió en la sala, se comportó
muy amable y les explicó la razón por la que Tita no se podía casar.
\\- Claro que sí lo que les interesa es que Pedro se case, pongo a su %
consideración a mi hija Rosaura, solo 2 años mayor que Tita, pero está %
plenamente disponible y preparada para el matrimonio \ndots \\
\rule{1em}{0pt}%
Al escuchar estas palabras, Chencha por poco tira encima de Mamá Elena
la charola con café y galletas que había llevado a la sala para agasajar\footnote{für jmds./etw. Wohlergehen sorgen, jmdn./etw. gut bewirten}
a don Pascual y a su hijo. Disculpándose, se retiró apresuradamente
hacia la cocina, donde la estaban esperando Tita, Rosaura y Gertrudis
para que les diera un informe detallado de lo que acontecía\footnote{geschehen, sich ereignen}
en la sala. Entró atropelladamente\footnote{überstürzt} y
todas suspendieron de inmediato sus labores para no perderse una sola
de sus palabras.

Se encontraban ahí reunidas con el propósito de preparar tortas de
Navidad. Como su nombre lo indica, estas tortas se elaboran durante la
época navideña, pero en esta ocasión las estaban haciendo para festejar
el cumpleaños de Tita. El treinta de septiembre cumpliría dieciséis años
y quería celebrar los comiendo uno de sus platillos favoritos.
\\- ¿Ay sí, no? ¡So ‘ama’ habla d’estar preparada para el matrimonio, %
como si juera un plato de enchiladas! ¡Y ni~\sidenote{21}ansina, porque pos no es %
lo mismo que lo mesmo! ¡Uno no puede cambiar unos tacos por unas %
enchiladas así como así!\\

% Carlota = Chencha

Chencha no paraba de hacer este tipo de comentarios mientras les
narraba, a su manera, claro, la escena que acababa de presenciar\footnote{etw. (mit eigenen Augen) sehen, bei etw. dabei sein}.
Tita conocía lo exagerada y mentirosa que podía ser Chencha, por lo que no
dejó que la angustia se apoderara de ella. Se negaba a aceptar como
cierto lo que acababa de escuchar. Fingiendo serenidad\footnote{die Gelassenheit, die Abgeklärtheit}, siguió partiendo\footnote{losgehen, aufbrechen, etw. auseinanderreißen, etw. zerbrechen, etw. klassifizieren, etw einteilen}
las teleras, para que sus hermanas y Nacha se encargaran de rellenarlas.

De preferencia las teleras deben ser horneadas en casa. Pero si no se
puede lo más conveniente es encargar en la panadería unas teleras
pequeñas, pues las grandes no funcionan adecuadamente para esta receta.
Después de rellenarlas se meten diez minutos al horno y se sirven
calientes. Lo ideal es dejarlas al sereno\footnote{heiter, ruhig, gelassen}
toda una noche en huertas en una tela\footnote{der Stoff, das Leintuch},
para que el pan se impregne con la grasa del chorizo.

Cuando Tita estaba acabando de envolver las tortas que comerían al día
siguiente, entró en la cocina Mamá Elena para informarles que había
aceptado que Pedro se casara, claro con Rosaura.

Al escuchar la confirmación de la noticia, Tita sintió como si el
invierno le hubiera entrado al cuerpo de golpe y porrazo\footnote{der Keulenschlag}: era tal el frío y tan
seco que le quemó las mejillas y se las puso rojas, rojas, como el
color de las manzanas que tenía frente a ella. Este frío sobrecogedor\footnote{überraschend, überwältigend} la
habría de acompañar por mucho tiempo sin que nada lo pudiera atenuar\footnote{etw. mildern, etw. abschwächen}, ni tan
siquiera cuando Nacha le contó lo que había escuchado cuando acompañaba
a Pascual Muzquiz y a su hijo hasta la entrada del rancho. Nacha caminaba
por delante, tratando de aminorar\footnote{etw. mindern, etw. verlangsamen}
el paso para escuchar mejor la conversación entre padre e hijo. Don Pascual
y Pedro caminaban lentamente y hablaban en voz baja, reprimida\footnote{sich etw. verkneifen, etw. zurückhalten} por
el enojo.
\\- ¿Por qué hiciste esto, Pedro? Quedamos en ridículo aceptando la boda %
con Rosaura. ¿Dónde quedó pues el amor que le juraste a Tita? ¿Qué no %
tienes palabra? %
\\- Claro que la tengo, pero si usted le negaran de una manera rotunda\footnote{kategorisch, rundweg} casarse con la mujer que %
ama y la única salida que le dejaran para estar cerca de ella fuera la %
de casarse con la hermana, ¿no tomaría la misma decisión que yo?\\

Nacha no alcanzó a escuchar la respuesta porque el pulque, el perro del
rancho, salió corriendo, ladrándole a un conejo aquí confundió con un
gato.
\\- Entonces, ¿te vas a casar sin sentir amor? %
\\- No, papá, me caso sintiendo un inmenso e imperecedero\footnote{unvergänglich} %
amor por Tita.\\

Las voces se hacían cada vez menos perceptibles pues eran apagadas por
el ruido que hacían los zapatos al pisar las hojas secas. Fue extraño
que Nacha, que para entonces estaba más sorda, dijera haber escuchar la
conversación. Tita igual le agradeció que se lo hubiera contado pero
eso no modificó la actitud de frío respeto que desde entonces tomó para
con Pedro. Dicen que el sordo no oye pero compone. Tal vez Nacha solo
escuchó las palabras que todos callaron. Esa noche fue imposible que
Tita conciliara el sueño; no sabía explicar lo que~\sidenote{23}sentía. Lástima
que en aquella época no se hubieran descubierto los hoyos\footnote{das Loch, die Grupe, das Grab} negros en el
espacio porque entonces le hubiera sido muy fácil comprender que sentía
un hoyo negro en medio del pecho, por donde se le colaba un frío
infinito.

Cada vez que cerraba los ojos podría revivir muy claramente las escenas
de aqulla noche de Navidad, un año atrás, en que Pedro y su familia
habían sido invitados por primera vez a cenar a su casa, y el frío se le agudizaba\footnote{sich verschärfen, etw. verschärfen}.
A pesar del tiempo transcurrido, ella podía recordar perfectamente los
sonidos, los olores, el roce\footnote{die Reibung, jmdn./etw. leicht berühren, jmdn./etw. streifen} de su vestido nuevo sobre
piso recién encerado\footnote{etw. bohnern, etw. wachsen}; la mirada de Pedro sobre sus hombros \ndots ¡Ese
mirada! Ella caminaba hacia la mesa llevando una charola\footnote{das Tablett} con dulces de yemas\footnote{der Eidotter, das Eigelb} de huevo cuando la sintió, ardiente, quemándole la piel. Giró la
cabeza y sus ojos se encontraron con los de Pedro. En ese momento
comprendió perfectamente lo que debe sentir la masa de un buñuelo al
entrar en contacto con el aceite hirviendo. Era tan real la sensación de
calor que invadía todo su cuerpo que ante el temor\footnote{die Furcht, die Bange} de que. Como a un
buñuelo, le empezaran a brotar\footnote{quellen, sprudeln, aufspritzen}
burbujas por todo el cuerpo - la cara, el
vientre, el corazón, los senos - Tita no pudo sostenerle esa mirada y
bajando la vista cruzó rápidamente el salón hasta el extremo opuesto\footnote{entgegengesetzt, gegensätzlich},
donde Gertrudis pedaleaba\footnote{in die Pedale treten, radeln} en la pianola El vals \textit{Ojos de juventud}.
Depositó la charola sobre un mesita de centro, tomó distraídamente una
copa de licor de noyó que encontró en su camino y se sentó junto a
Paquita Lobo, vecina del rancho. El poner distancia entre Pedro y ella
de nada le sirvió; sentía la sangre correr abrasadoramente\footnote{brennend, glühend, sengend} por sus
venas. Un intenso rubor\footnote{die Röte}~\sidenote{24}le cubrió las mejillas y por más esfuerzos
que hizo no pudo encontrar un lugar donde posar su mirada. Paquita notó
que algo raro le pasaba y mostrando gran preocupación la interrogó:

- Qué rico está el licorcito, ¿verdad? %
\\- ¿Mande usted? %
\\- Te veo muy distraída, Tita, ¿te sientes bien? %
\\- Si, muchas gracias. %
\\- Ya tienes edad suficiente como para tomar un poco de licor en %
ocasiones especiales, pilluela\footnote{der Schlingel, der Fratz}, pero dime, ¿cuentas con la autorización %
de tu mamá para hacerlo? Porque te noto a agitada y temblorosa - y %
añadió lastimeramente\footnote{jämmerlich, erbärmlich} -, mejor ya no tomes, no vayas a dar un %
espectáculo.\\

¡Nada más eso le faltaba! Que es Paquito Lobo pensara que estaba
borracha. No podía permitir que le quedara la menor duda o se exponía a
que fuera a llevarle el chisme a su mamá. El terror a su madre la hizo
olvidarse por un momento de la presencia de Pedro y trató por todos los
medios de convencer a Paquita de la lucidez\footnote{die Deutlichkeit}
de su pensamiento y de su agilidad mental. Platicó\footnote{sich unterhalten, plaudern}
de ella, de algunoschismes y bagatelas.
Inclusive le proporcionó la receta del noyó, que tanto la inquietaba.
Este licor se fabrica poniendo cuatro onzas de almendras de alberchigo y
media libra\footnote{das Pfund} de almendras del albaricoque
en un azumbre\footnote{Flüssigkeitsmaß von rund zwei Litern} de agua,
por 24 horas, para que aflojen\footnote{abflauen, etw. auflockern} la piel;
luego se pelan, se quebrantan\footnote{etw. brechen}
y se ponen en infusión\footnote{der Kräutertee, der Aufguss} en dos
azumbres de agua ardiente, por quince días. Después
se procede a la destilación. Cuando se han desleído\footnote{sich auflösen, zergehen, etw. anrühren, verrühren}
perfectamente dos libras y media de azúcar quebrantada en el agua se le añaden
cuatro onzas de flor de naranja, se forman la mezcla y se~\sidenote{25}filtra. Y para que no
quedara ninguna duda referente a su salud física y mental, le recordó a
Paquita, así como de refilón\footnote{schräg ansehen}, que
la equivalencia del azumbre es de 2.016 litros, ni más ni menos.
Así que cuando Elena se acercó a ellas para preguntarle a Paquita si
estaba bien atendida, esta entusiasmada respondió:
\\- ¡Estoy perfectamente! Tienes unas hijas maravillosas. ¡Y su %
conversación es fascinante!\\

Mamá Elena le ordenó a Tita que fuera a la cocina por unos bocadillos
para repartir entre todos los presentes. Pedro, que en ese momento pasaba
por ahí, no por casualidad, se ofreció a ayudarla. Tita caminaba
apresuradamente hacía la cocina, sin pronunciar una sola palabra. La
cercanía de Pedro la ponía muy nerviosa. Entró y se dirigió con rapidez
a tomar una de las charolas con deliciosos bocadillos que esperaban
pacientemente en la mesa de la cocina.

Nunca olvidaría el roce accidental\footnote{zufällig, unabsichtich, versehentlich}
de sus manos cuando ambos trataron torpemente\footnote{schwerfällig, ungeschickt}
de tomar la misma charola al mismo tiempo.

Fue entonces cuando Pedro le confesó su amor.
\\- Señorita Tita, quisiera aprovechar la oportunidad de poder hablarle a %
solas para decirle que estoy profundamente enamorado de usted. Sé que %
esta declaración es atrevida y precipitada, pero es tan difícil %
acercársele que tomé la decisión de hacerlo esta misma noche. Solo le %
pido que me diga si puedo aspirar a su amor. %
\\- No sé qué responderle; déme tiempo~\sidenote{26}para pensar. %
\\- No, no podría, necesito una respuesta en este momento; el amor no se %
piensa: se siente o no se siente. Yo soy hombre de pocas pero muy firmes %
palabras. Le juro que tendrá mi amor por siempre. ¿Qué hay del suyo? %
¿Usted también lo siente por mí? %
\\- ¡Si! \\
\rule{1em}{0pt}Si, si, y mil veces sí. Lo amó desde esa noche para siempre. Pero
ahora tenía que renunciar a él. No era decente desear al futuro esposo
de una hermana. Tenía que tratar de ahuyentarlo\footnote{jmdn./etw. verjagen, jmdn./etw. verscheuchen}
de su mente de alguna manera para poder dormir. Intentó comer la torta
de Navidad que Nacha le había dejado sobre su buró\footnote{der Schreibtisch, der Nachttisch},
junto con una vaso de leche. En muchas ocasiones le había dado excelentes
resultados. Nacha, con su gran experiencia, sabía que para Tita no había
pena alguna que no lograr desaparecer mientras comía una deliciosa torta
de Navidad. Pero no en esta ocasión. El vacío que sentía en el estómago
no se alivió. Por el contrario, una sensación de náusea\footnote{der Brechreiz}
la invadió. Descubrió que el hueco no era de hambre; más bien se trataba
de una álgida\footnote{eisig} sensación dolorosa. Era
necesario deshacerse de este molesto frío. Como primera medida se cubrió
con una pesada cobija\footnote{die Decke, die Bettwäsche}
y ropa de lana el frío permanecía inamovible\footnote{unverrückbar, unversetzbar}.
Entonces se puso zapatos de estambre\footnote{das Kammgarn} y
otras dos cobijas. Nada. Por último, sacó de su costurero\footnote{der Nähkasten}
una colcha\footnote{die Tagesdecke} que había empezado a tejer el
día en que Pedro le habló de matrimonio. Una colcha como esta, tejida a gancho\footnote{der Haken}, se termina aproximadamente en un año.
Justo el tiempo que Pedro y Tita habían pensado dejar pasar antes~\sidenote{27}de contraer nupcias\footnote{die Eheschließung, hochzeitlich}.
Decidió darle utilidad al estambre en lugar de desperdiciarlo y
rabiosamente tejió y lloró, y lloró y tejió, Hasta que en la madrugada
terminó la colcha y se la echó encima. De nada sirvió. Ni esa noche ni
muchas otras mientras vivió logró controlar el frío.

\clearpage

\section*{ Capítulo 2 }
\subsection*{ Febrero }
\subsection*{ Pastel Chabela }
\textbf{Ingredientes}

\begin{itemize}
\item 175 Gramos de azúcar granulada de primera
\item 300 Gramos de harina de primera, tamizada\footnote{gesiebt, überprüft} tres veces
\item 17 huevos
\item Raspadura\footnote{die Raspelspäne, das Abschaben} de un limón
\end{itemize}

\section*{ Manera de hacerse:}
En una cacerola se ponen~\sidenote{33}cinco yemas de huevo, cuatro huevos enteros y
el azúcar. Se baten hasta que la masa espesa y se anexan\footnote{etw. annektieren, etw. einverleiben} dos huevos
enteros más. Se sigue batiendo y cuando vuelve a espesar se le agregan\footnote{etw. hinzufügen, etw. ergänzen} otros dos
huevos completos, repitiendo este paso hasta que se terminan
incorporar todos los huevos, de dos en dos. Para elaborar el pastel de
boda de Pedro con Rosaura, Tita y Nacha habían tenido que multiplicar
por diez las cantidades de esta receta pues en lugar de un pastel dieziocho
personas tenían que preparar uno para ochenta. ¡Resultado da 170 huevos! Y eso
significaba que había tenido que tomar medidas para tener reunida\footnote{etw. ansammeln, etw. zusammensuchen,  jmdn./etw. vereinen, jmdn./etw. zusammenführen} esta
cantidad de huevos, de excelente calidad, en un mismo día.

Para lograrlo fueron poniendo en conserva desde hacía varias semanas
los huevos que ponían las gallinas de mejor calidad. Este método se
utilizaba en el rancho desde época inmemorial\footnote{uralt, seit jeher}
para proveerse\footnote{sich mit etw. ausrüsten, sich mit etw. versorgen}
durante el invierno de este nutritivo y necesario alimento.

El mejor tiempo para esta operación es por los meses de agosto y septiembre.
Los huevos que se destinen a la conservación deben ser muy~\sidenote{34}frescos.
Nacha prefería que fueran del mismo día. Se ponen los huevos en una vasija
que se llena de cebo\footnote{das Lockmittel, der Köder, das Mastfutter}
de carnero\footnote{der Hammel} derretido, próximo a enfriarse\footnote{erkalten, sich abkühlen}, hasta cubrirlo por
completo. Esto basta para garantizar su buen estado por varios meses.
Ahora, que si se desea conservarlos por más de un año, se colocan los
huevos en una orza\footnote{henkelloses Tongefäß für Honig, Marmelade und Konserve}
y se cubren con una lechada\footnote{die Tünche, die Kalkmilch} de un tanto de cal\footnote{der Kalk} por diez de agua. Después se tapan muy bien
para interceptar\footnote{etw. unterbrechen, etw. sperren, etw. aufhalten}
el aire y se guardan en la bodega. Tita y Nacha habían elegido la primera
opción pues no necesitaban conservar los huevos por tantos meses. Junto
a ellas, bajo la mesa de la cocina, tenían la vasija donde los habían
puesto y de ahí los tomaban para elaborar el pastel.

El esfuerzo fenomenal que representaba al batir tantos huevos empezó a
hacer estragos\footnote{die Verwüstung, die Zerstörung} en la
mente de Tita cuando iban apenas por los ciento huevos batidos. Le
parecía inalcanzable llegar a la cifra de cientosetenta.

Tita batía mientras Nacha rompía las cascarones\footnote{die Eierschale}
y los incorporaba. Un estremecimiento\footnote{das Frösteln, der Schauder}
recorría el cuerpo de Tita y, como vulgarmente\footnote{vulgär, ordinär}
se dice, se la ponía la piel de gallina cada vez que se rompía un huevo.
Asociaba los blanquillos con los testículos\footnote{der Hoden}
de los pollos a los que habían capado\footnote{jmdn./etw. einschränken, etw. beschneiden, etw. kastrieren}
un mes antes. Los capones son gallos castrados que se ponen a engordar.
Se eligió este platillo para la boda de Pedro cuando Rosaura por ser uno
de los más prestigiados\footnote{jmdm./etw. Ansehen verleihen}
en las buenas mesas, tanto por el trabajo que implica su preparación
como por el extraordinario sabor de los capones.

Desde que se fijó la boda para el doce~\sidenote{35}de entero se mandaron
comprar doscientos pollos a los que se les practicó la operación
y se pusieron a engorda inmediato.

Las encargadas de esta labor fueron Tita y Nacha. Nacha por su
experiencia y Tita como castigo por no haber querido estar presente al
día en que fueron a pedir la mano de su hermana Rosaura, pretextando\footnote{etw. als Vorwand gebrauchen, vorgeben} una jaqueca\footnote{die Migräne}.
\\
\\- No voy a permitir tus desmandadas\footnote{sich widersetzen, ausfallend werden} - le %
dijo Mamá Elena - me voy a permitir que le arruines a tu hermana su %
boda, con tu actitud de víctima. Desde ahora te vas a encargar de los %
preparativos para el banquete y cuidadito y yo te vea una mala cara o %
una lágrima, ¿me oíste?\\

Tita trataba de no olvidar esta advertencia mientras se disponía a
iniciar la primera operación. La capada consiste en hacer una incisión\footnote{der Schnitt, der Einschnitt}
en la parte que cubre los testículos del pollo: se mete el dedo para
buscarlos y se arrancan. Luego de ejecutado, se cose la herida y se
frota con mantequilla fresca o con enjundia\footnote{das Tierfett}
de aves. Tita estuvo a punto de perder el sentido cuando metió el dedo
y jaló\footnote{an etw. ziehen} los testículos
del primer pollo. Sus manos temblaban, sudaba copiosamente\footnote{reichlich}
y el estómago le giraba como un papalote\footnote{der Papierdrachen}
en vuelo. Mamá Elena le lanzó una mirada taladrante\footnote{etw. durchbohren}
y le dijo:
\\- ¿Qué te pasa? ¿Por qué tiemblas?, ¿vamos a empezar con problemas? - Tita %
levantó la vista y la miró. Tenía ganas de gritarle que sí, que %
había problemas, se había elegido mal al sujeto apropiado para capar, la %
adecuada era ella, de este manera habría menos una justificación real %
para que le estuviera negado el matrimonio y Rosaura tomara su lugar al %
lado del hombre que ella~\sidenote{36}amaba. Mamá Elena, leyéndole la mirada, enfureció\footnote{in Wut geraten, jmdn. wütend machen} %
y le propinó\footnote{jmdm. etw. versetzen} a %
Tita una bofetada fenomenal que la hizo rodar por el suelo, junto con el pollo, %
que pereció\footnote{zugrunde gehen, ums Leben kommen} por la mala operación.\\

Tita batiá y batía con frenesí\footnote{der Taumel, das Ungestüm, die Raserei},
como queriendo terminar una vez por todas con el martirio. Solo tenía que
batir dos huevos más y la masa para el pastel quedaría lista. Era lo único
que faltaba, todo lo demás, incluyendo los platillos para una comida de veinte
platos y los bocadillos de entrada, estaban listos para el banquete. En la
cocina solo quedaban Tita, Nacha y Mamá Elena. Chencha, Gertrudis y Rosaura
estaban dando los últimos toques al vestido de novia. Nacha, con un gran
alivio, tomó el penúltimo\footnote{vorletzter} para partirlo.
Tita, con un grito, impidió que lo hiciera.
\\- ¡No! \\
\rule{1em}{0pt}Suspendió la batida y tomó el huevo entre sus manos. Claramente
escuchaba piar a un pollo dentro del cascarón. Acercó el huevo a su oído
y escuchó con más fuerza los pillidos. Mamá Elena suspendió su labor y
con voz autoritaria preguntó:
\\- ¿Qué pasa? ¿Qué fue ese grito? %
\\- ¡Es que dentro de este huevo hay un pollo! Nacha de seguro no lo puede %
oír, pero yo sí. %
\\- ¿Un pollo? ¿Estás loca? ¡Nunca ha pasado algo parecido con los huevos %
de conserva!\\

De dos zancadas\footnote{langer Schritt}, llegó hasta donde
estaba Tita, le arrebató\footnote{wütend werden, außer sich geraten, jmdm. etw. entreißen, jmdn. mitreißen}
el huevo de las manos y lo partió.
Tita cerró los ojos con fuerza.
\\- ¡Abre los ojos y mira tú pollo! \\
\rule{1em}{0pt}Tita~\sidenote{37}abrió los ojos lentamente. Con sorpresa vio que lo que creía
un pollo no era más que un nuevo y bastante fresco, por cierto.
\\- Escúchame bien, Tita, me estás colmando\footnote{etw. überfüllen} %
la paciencia, no te voy a permitir que empieces con locuras. ¡Esta es la %
primera y la última! ¡O te aseguro que te arrepentirás!\\

Tita nunca pudo explicar qué fue lo que pasó esa noche, si lo que
escuchó fue producto del cansancio o una alucinación de su mente. Por lo
pronto lo más conveniente era volver a la batida, no quería investigar
cuál era el límite de la paciencia de su madre.

Cuando se baten los dos últimos huevos, se incorpora la ralladura\footnote{der Raspel} del
limón; cuando ha espesado bastante la masa, se deja de batir y se le
pone la harina tamizada, mezclándola poco a poco con una espátula de
madera, hasta incorporarla toda. Por último se engrasa un molde con
mantequilla, se espolvorea con harina y se le vacía la pasta. Se cuece\footnote{sieden, etw. kochen, etw. backen} en horno por treinta
minutos.

Nacha, después de preparar durante tres días veinte platillos diferentes, se
encontraba muerta de cansancio y no veía llegar la hora de meter el
pastel al horno para por fin poderse ir a descansar. Tita por esta vez
no era muy buena ayudante que digamos. En ningún momento se había
quejado, tal vez porque la mirada escrutadora\footnote{forschend}
de su madre no se lo permitía, pero en cuanto vio a Mamá Elena salir de la
cocina para dirigirse a sus habitaciones, lanzó un interminable suspiro.
Nacha, a su lado, le quitó suavemente la pala de las manos, la abrazó y le dijo:
\\- Ya no hay nadie en la~\sidenote{38}cocina, mi niña, llora ahora, porque mañana %
no quiero que te vean hacerlo. Mucho menos Rosaura.\\

Nacha suspendió la batida porque sentía que Tita estaba a punto de un
colapso nervioso, bueno, ella no lo conocía con ese nombre, pero con su
inmensa sabiduría comprendía que Tita no podía más. La verdad, ella
tampoco. Rosaura y Nacha nunca se habían llevado bien. A Nacha le
molestaba mucho que desde niña Rosaura fuera melindrosa\footnote{zimperlich, geziert} con la comida.
Siempre le dejaba intacta\footnote{heil, intakt, unberührt} en
el plato, o se la daba a escondidas a Tequila, el papá de Pulpe
(el perro del rancho). Nacha le ponía ejemplo a Tita, que siempre
comió muy bien y de todo. Bueno, solo había un
alimento que no era del agrado de Tita; se trataba del huevo tibio que
Mamá Elena la obligaba a comer. De ahí en fuera, como Nacha se había
encargado de su educación culinaria, Tita no solo comía lo
acostumbrado, sino que comía, además, jumiles\footnote{ein wanzenähnliches Insekt}, gusanos\footnote{der Wurm, die Made} de maguey\footnote{die Agave},
acociles, tepezcuintle\footnote{das Paka}, armadillo\footnote{das Gürteltier}, etc., ante el
horror de Rosaura. De ahí nació la aversión de Nacha para con Rosaura
y la rivalidad entre los dos hermanas, que culminaba\footnote{in etw. gipfeln}
con esta boda en la que Rosaura se casaba
con el hombre que Tita amaba. Lo que Rosaura no sabía, aunque lo
sospechaba, era que Pedro amaba a Tita con un amor inconmensurable\footnote{unermesslich}. Era
de entender entonces que Nacha tomara partido por Tita y tratara por
todos los medios de evitarle sufrimientos. Nacho le secaba con su
delantal las lágrimas que rodaban por la cara de Tita y le decía:
\\- Ya, mi niña, ya vamos a terminar. \\
\rule{1em}{0pt}Pero se tardaron más de lo acostumbrado~\sidenote{39}pues la masa no podía
espesar debido a las lágrimas de Tita.

Y así, abrazaras, permanecieron llorando hasta que a Tita no le quedaron
más lágrimas en los ojos. Entonces lloró en seco y dicen que eso duele
más, como el parto en seco, pero al menos no seguía mojando la masa del
pastel, pudiendo continuar con el paso siguiente, que es el del relleno.

\textbf{Relleno:}
\begin{itemize}
    \item 150 gramos de pasta de chabacano\footnote{die Aprikose}
    \item 150 g de azúcar granulado
\end{itemize}

\subsection*{ Manera de hacerse: }
Se ponen los chabacanos al fuego con muy poca agua, se dejan hervir y se
pasan por un cedazo\footnote{das Sieb} o tamiz\footnote{das Filtersieb, das Sieb};
si no se tiene, se puede usar una vulgar coladera\footnote{das Sieb, der Seiher}.
Se pone esta pasta en una cacerola, se le agrega el azúcar y se pone al
fuego sin dejar de moverla hasta que toma punto de mermelada.
Se retira del fuego y se deja enfriar un poco antes de ponerla en la
parte de en medio del pastel, que por supuesto se ha partido con
anterioridad.

Afortunadamente, un mes antes~\sidenote{40}de la boda, Nacha y Tita habían
preparado varios frascos\footnote{das Vorratsglas, das Einmachglas}
conservas de chabacano, higo y camote\footnote{die Süßkartoffel} con piña.
Gracias a eso se evitaron el trabajo de preparar la mermelada ese mismo
día.

Ellas estaban acostumbradas a preparar cantidades enormes de mermelada
en un gran cazo\footnote{die Schöpfkelle, der Kochtopf} que se ponía
en el patio, para aprovechar la fruta de la temporada. Lo colocaban
encima de una fogata\footnote{das Lagerfeuer, das Freudenfeuer} y
para menear\footnote{etw. schwenken, etw. schütteln, wackeln}
la mermelada tenían que cubrirse los brazos con sábanas viejas. Esto
evita para que un borbotón\footnote{das Sprudeln} de la misma brincara\footnote{hopsen, hüpfen} y les quemara la piel.

En cuanto Tita abrió el frasco, el olor de los chabacanos la hizo remitirse\footnote{sich auf etw. berufen} a la tarde en
que prepararon la mermelada. Tita venía del huerto cargando la fruta
sobre su falda pues había olvidado la canasta\footnote{der Korb}.
Traía recogida la falda cuándo entró y cuál no sería su sorpresa al
toparse con Pedro en la cocina. Pedro se dirigía al patio trasero a
preparar la carretela\footnote{die Kalesche mit Klappverdeck}.
Tenían que ir al pueblo a entregar unas invitaciones y como el caballerango\footnote{der Stallbursche} no se había
presentado ese día en el rancho, él mismo tenía que encargarse de esa
labor. En cuanto Nacha lo vio entrar en la cocina salió casi corriendo,
pretextando ir por epazote\footnote{Mexikanischer Drüsengänsefuß}
para los frijoles. Tita, de la sorpresa, dejó caer algunos chabacanos
sobre piso. Pedro rápidamente corrió a ayudarla a recogerlos. Y al
inclinarse pudo ver una parte de las piernas de Tita que quedaba al
descubierto.

Tita, tratando de evitar que Pedro la mirara, dejó caer su falda. Al
hacerlo, el resto de los chabacanos rodaron sobre la cabeza de~\sidenote{41}Pedro.
\\- Perdóneme, Pedro. ¿lo lastimé\footnote{jmdn. verletzen}? %
\\- No tanto como yo la he lastimado, déjeme decirle que mi propósito \ndots %
\\- No le he pedido ninguna explicación. %
\\- Es necesario que me permita dirigirle unas palabras \ndots %
\\- Una vez lo hice y resultaron una mentira, no quiero escucharlo más %
\ndots\\

Y diciendo esto, Tita salió rápidamente de la cocina, por la otra
puerta, hacía la sala, donde Chencha y Gertrudis bordaban la sábana
nupcial. Era una sábana de seda blanca a la que le estaban haciendo un
delicado bordaro en el centro. Este orificio estaba destinado a mostrar
únicamente las partes nobles de la novia en los momentos íntimos del
matrimonio. Realmente habían tenido suerte en haber podido conseguir
seda francesa en esas épocas de inestabilidad política. La revolución no
permitía que uno viajara de una manera segura por el país; así es que,
de no haber sido por un chino, que se dedicaba al contrabando\footnote{der Schmuggel}, no les
hubiera sido posible conseguir la tela, pues Mamá Elena no habría
permitido que ninguna de sus hijas se arriesgara yendo a la capital a
comprar lo necesario para el vestido y el ajuar\footnote{die Aussteuer, der Hausrat}
de Rosaura. Este chinito era bastante listo: vendía en la capital
aceptando billetes del ejército revolucionario del norte, ahí carecían
de valor y no eran negociables. Claro que los aceptaba a precios irrisorios\footnote{lachhaft, lächerlich} y con ese dinero
viajaba al norte, donde los billetes adquirirán su precio real y con
ellos compraba mercancía.

En el norte, por~\sidenote{42}supuesto, aceptaba billetes emitidos en la
capital a precios ínfimos\footnote{klein und mickrig, läppisch}
y así se la pasó toda a la revolución, hasta
que terminó millonario. Pero lo importante era que gracias a él Rosaura
pudo gozar de las telas más finas y exquisitas para su boda.

Tita quedó como hipnotizada, observando la blancura\footnote{die Weiße}
de la sabana, solo fueron algunos segundos, pero los suficientes como
para causarle una especie de ceguera\footnote{die Blindheit}. Dondequiera\footnote{irgendwo, überall, wo immer} que fijaba la
vista solo distinguía el color blanco. A Rosaura, que se encontraba
escribiendo a mano unas invitaciones, la percibía como un níveo\footnote{blütenweiß}
fantasma.  Disimuló también lo que pasaba que nadie lo notó.

No quería provocar otro regaño\footnote{die Schelte, jmdn. ausschimpfen, jmdn. rügen}
de Mamá Elena. Así que cuando los Lobo llegaron a entregar su regalo de bodas,
procuró agudizar sus sentidos
para descubrir a quienes estaba saludando pues para ella era como un espectáculo de sombras chinas\footnote{das Schattenspiel}
cubiertas por una blanca sábana.
Afortunadamente la voz chillante\footnote{kreischen, quieken} de Paquita
le dio la clave y los pudo saludar sin mayor problema.

Más tarde, cuando los acompañó a la entrada del rancho notó que hasta la
noche se mostraba ante ella como nunca la había visto: reluciente\footnote{strahlend, glänzend}
de albor\footnote{die Morgendämmerung}.

Le dio miedo que le fuera a pasar lo mismo en estos momentos, cuando
por más que trataba de concentrarse en la elaboración del fondant para
cubrir el pastel, no podía. La atemorizaba la blancura del azúcar
granulado, sentía que de un momento a otro el color blanco
se adueñaría\footnote{sich etw. aneignen, sich jmds./etw. bemächtigen}
de su mente, sin que ella no pudiera impedir, arrastrándo los cándidas\footnote{arglos, naiv, unschuldig}
imágenes de su niñez cuando en el mes de mayo~\sidenote{43}la llevaban vestida
de blanco a ofrecer flores blancas a la Virgen. Entraba caminando entre
una fila de niñas vestidas de blanco, hasta el altar lleno de velas y
flores blancas, iluminado por una celestial luz blanca proveniente del vitral\footnote{das Kirchenfenster} de la blanca parroquia\footnote{die Kirchengemeinde, die Pfarrei}.
No hubo una sola vez en que no entrara a la iglesia soñando en que algún
día lo haría de brazo de un hombre. Tenía que bloquear no solo este sino
todos aquellos recuerdos que la lastimaran: tenía que terminar el fondant
para el pastel de boda de su hermana. Haciendo un esfuerzo supremo\footnote{allerhöchst}
empezó a prepararlo:

\textbf{Cantidades para el fondant}

\begin{itemize}
    \item 800 gramos de azúcar granulado
    \item 60 gotas de limón y el agua suficiente para que remoje el azúcar
\end{itemize}

\section*{ Manera de hacerlo: }
Se ponen en una cacerola el azúcar y el agua al fuego sin dejar de
moverla, hasta que empieza a hervir. Se cuela en otra cacerola y se
vuelve a poner al fuego agregándole el limón hasta que tome punto de bola\footnote{die Kugel} floja,
limpiando de vez en cuando los bordes de la cacerola con un lienzo\footnote{das Leinen} húmedo para que la miel no se azucare;
cuando ha tomado el punto~\sidenote{44}anteriormente indicado se vacía en otra
cacerola húmeda, se rocía\footnote{etw. mit etw. besprengen, etw. mit etw. bestreuen, etw. mit etw. benetzen}
por encima y se deja enfriar un poco.

Después, con una espátula de madera, se bate hasta que empaniza\footnote{etwas panieren}.

Para aplicarlo, se le echa una cucharada de leche y se vuelve a poner al
fuego para que se deslíe,
se le añade después una gota de carmín\footnote{das Karmin} y se
cubre con él únicamente la parte superior del pastel.

Nacho se dio cuenta de que Tita estaba mal cuando está le preguntó si
no le iba a poner el carmín.
\\- Mi niña, se lo acabo de poner, ¿no ves el color rosada que tiene? %
\\- No \ndots %
\\- Vete a dormir, niña, yo termino el turrón. Solo las ollas saben los hervores\footnote{das Aufkochen, die Aufwallung} de su caldo, %
pero yo adivino los tuyos, y ya deja de llorar, %
que me estás mojando el fondant y no va a servir, anda, ya vete.\\

Nacha cubrió de besos a Tita y la empujó fuera de la cocina. No se
explicaba de dónde había sacado nuevas lágrimas, pero las había sacado y
alterado con ellas la textura del turrón. Ahora le costaría doble
esfuerzo dejarlo en su punto. Ya sola se dio a la tarea de terminar con
el turrón lo más pronto posible, para irse a dormir. El turrón que hace
con dez claras de huevo y 500 g de azúcar batidos a punto de hebra\footnote{der Faden}
fuerte.

Cuando terminó, se le ocurrió darle un dedazo\footnote{der Fingerhut} al fondant, para ver si
las lágrimas de Tita no habían alterado el sabor. Y no, aparentemente no
alteraron el sabor, pero, sin saber por~\sidenote{45}qué, a Nacha le entró de
golpe una gran nostalgia. Recordó uno a uno todos los banquetes de boda
que había preparado para la familia de la Garza con la ilusión de que el
próximo fuera el suyo. A sus 85 años no valía la pena llorar, ni
lamentarse de que nunca hubieran llegado ni el esperado banquete ni la
esperada boda, a pesar de que el novio si llegó, ¡vaya que había llegado!
Solo que la mamá de mamá Elena si había encargado de ahuyentarlo. Desde
entonces se había conformado con gozar de las bodas ajenas y así lo hizo
por muchos años sin repelar\footnote{mähen, zausen, kahl scheren}. No sabía por qué lo hacía ahora. Sentía que
era una reverenda\footnote{ehrwürdig} tontería, pero no podía
dejar de hacerlo. Cubrió el pastel con el turrón lo mejor que pudo y se
fue a su cuarto, con un fuerte dolor de pecho. Lloró toda la noche y a
la mañana siguiente no tuvo ánimos para asistir a la boda.

Tita hubiera dado cualquier cosa por estar en el lugar de Nacha, pues
ella no solo tenía que estar presente en la iglesia, se sintiera como se
sintiera, sino que tenía que estar muy pendiente\footnote{hängend, unerledigt, schwebend}
de que su rostro no revelara\footnote{etw. enthüllen, etw. offenlegen}
la menor emoción. Creía poder lograrlo siempre y cuando su
mirada no se cruzara con la de Pedro. Ese incidente podría destrozar
todo la paz y tranquilidad que aparentaba\footnote{etw. vorspielen, etw. vorgeben, etw. zur Schau tragen}.

Sabía que ella, más que su hermana Rosaura, era el centro de atención.
Los invitados, más que cumplir con un acto social, querían regodearse\footnote{sich an etw. ergötzen}
con la idea de su sufrimiento, pero no los complacería, no. Podía sentir
claramente cómo penetraban por sus espaldas los cuchicheos\footnote{das Getuschel}
de los presentes a su paso.
\\- ¿Ya~\sidenote{46}viste a Tita? ¡Pobrecita, su hermana se va a casar con su %
novio! Yo los vi un día en la plaza del pueblo, tomados de la mano. ¡Tan %
felices que se veían! %
\\- ¡No me digas! ¡Pues Paquita dice que ella cómo un día, en plena misa, %
Pedro le pasó a Tita una carta de amor, perfumado y todo! %
\\-¡Dicen que van a vivir en la misma casa! ¡Yo que Elena no lo %
permitiría! %
\\- No creo que lo haga. ¡Ya ves cómo son los chismes! No le gustaban nada %
esos comentarios. El papel de perdedora\footnote{der Versager, die Versagerin, der Verlierer, die Verliererin} %
no se había escrito para ella.\\

¡Tenía que tomar una clara actitud de triunfo! Como un gran actriz
representó su papel dignamente, tratando de que su mente estuviera
ocupada no en la marcha nupcial ni en las palabras del sacerdote ni en
el lazo\footnote{die Schleife, die Verschlingung, das Lasso} y
los anillos.

Se transportó al día en que a los nueve años se había ido de pinta con los
niños del pueblo. Tenía prohibido jugar con varones\footnote{der (junge) Kerl},
pero ya estaba harta de los juegos con sus hermanas. Se fueron a la
orilla del Río Grande para ver quién era capaz de cruzarlo a nado,
en el menor tiempo. Qué placer sintió esa día al ser ella la ganadora.

Otro de sus grandes triunfos ocurrió un tranquilo día domingo en el
pueblo. Ella tenía catorce años y paseaba en carretela acompañada de sus
hermanas cuando unos niños lanzaron un cohete. Los caballos salieron
corriendo espantadísimos\footnote{kopfscheu}. En las afueras
del pueblo se desbocaron\footnote{ausfallend werden, durchgehen}
y el cochero\footnote{der Kutscher} perdió el control del vehículo.

Tita lo hizo a un lado de un empujón y ella solo pudo dominar a los cuatro
caballos. Cuando algunos hombres del pueblo~\sidenote{47}a galope\footnote{galoppieren}
las alcanzaron para ayudarla, se admiraron de la hazaña\footnote{die Heldentat, der Kraftakt, das Wagnis}
de Tita.

En el pueblo la recibieron como a una heroína.
Estás y otras muchas remembranzas\footnote{das Rückbesinnen}
parecidas la tuvieron ocupada durante la ceremonia, haciéndola lucir
una apacible\footnote{ruhig sanft, beschaulich} sonrisa de gata
complacida, hasta que a la hora de los abrazos tuvo que felicitar a su
hermana. Pedro, que estaba junto a ella, le he dicho a Tita:
\\- ¿Y a mí no me va a felicitar? %
\\- Sí, cómo no. Que sea muy feliz.\\

Pedro, aplazando la más cerca de lo que las normas sociales permiten,
aprovechó la única oportunidad que tenía que poder decirle a Tita algo
al oído:
\\- Estoy seguro de que así será, pues logré con esta boda lo que tanto anhelaba\footnote{etw. ersehnen, nach etw. lechzen}: estar %
cerca de usted, la mujer que verdaderamente amo \ndots\\

Las palabras que Pedro acaba de pronunciar fueron para Tita como
refrescante brisa que enciende los restos de carbón a punto de apagarse.
Su cara por tantos meses forzar a no mostrar sus sentimientos
experimentó un cambio incontrolable, su rostro reflejó gran alivio y
felicidad. Era como si toda esa casi extinguida\footnote{ausgestorben} ebullición\footnote{das Aufkochen, das Sieden, die Wallung} interior
se viera reavivada\footnote{jmdn./etw. wiederbeleben}
de pronto por el fogoso\footnote{Feuer $\ndots$ , hitzig, ungestüm} aliento
de Pedro sobre su cuello, sus ardientes manos sobre su espalda, su impetuoso\footnote{heftig, ungestüm} pecho sobre
sus senos \ndots Pudo haberse quedado para siempre así, de no ser por la
mirada que Mamá Elena le lanzó y la hizo separarse de él rápidamente.
Mamá Elena se acercó a Tita y le preguntó:
\\- ¿Qué fue lo que Pedro te dijo? %
\\- Nada, mami. %
\\- A mí no me engañas, cuando~\sidenote{48}tú vas, yo ya fui y vine, así que no %
te hagas la mosquita muerta. Pobre de ti si te vuelvo a ver cerca de %
Pedro.\\

Después te estas amenazantes palabras de Mamá Elena, Tita procuró estar
lo más alejada de Pedro que pudo. Lo que le fue imposible fue borrar de
sus rostro una franca\footnote{aufrichtig, offenherzig, freigebig}
sonrisa de satisfacción. Desde ese momento la boda tuvo para ella otro
significado.

Ya no le molestó para nada ver cómo Pedro y Rosaura iban de mesa en
mesa brindando con los invitados, ni verlos bailar el vals, ni verlos
más tarde partir el pastel. Ahora ella sabía que era cierto: Pedro la
amaba. Se moría por que terminara el banquete para correr al lado de
Nacha a contarle todo. Con impaciencia esperó a que todos comieran su
pastel para poder retirarse. El manual de Carreño le impedía hacerlo
antes, claro no le vedaba\footnote{etw. verbieten, etw. hemmen, etw. untersagen}
a flotar entre nubes mientras comía apuradamente\footnote{sorgfältig, erschöpft, in einer heiklen Lage}
su rebanada\footnote{die Brotscheibe, die Schnitte}. Sus
pensamientos la tenían tan ensimismada que no le permitieron observar
que algo raro sucedía a su alrededor. Una inmensa nostalgia se adueñaba
de todos los presentes en cuanto le daban el primer bocado al pastel.
Inclusive Pedro, siempre tan propio, hacía un esfuerzo tremendo por
contener las lágrimas. Y Mamá Elena, que ni cuando su esposo murió
había derramado\footnote{etw. verbreiten, etw. vergießen}
una infeliz lágrima, lloraba silenciosamente. Y eso no fue todo, el
llanto fue el primer síntoma de una intoxicación rara que tenía algo que
ver con un gran melancolía y frustración que hizo presa a todos los
invitados y los hizo terminar en el patio, los corrales y los baños añorando\footnote{jmdn./etw. sehnlichst verlangen, sich nach jmdm./etw. sehnen}
cada uno el amor de su vida.
Ni uno solo escapó el hechizo\footnote{der Zauber, jmdn. bezaubern}~\sidenote{49}y solo algunos afortunados llegaron a tiempo a los baños; los que no,
participaron de la vomitona colectiva que se organizó en pleno patio.
Bueno, la única a quien el pastel le hizo lo que el viento a Juárez
fue a Tita. En cuanto terminó de comerlo abandonó la fiesta. Quería
notificarle a Nacha cuanto antes que estaba en lo cierto al decir que
Pedro lo amaba solo a ella. Por ir imaginando la cara de felicidad
que Nacha pondría no se percató\footnote{sich etw. bewusst werden, etw. bemerken}
de la desdicha\footnote{das Elend} que crecía a su paso
hasta llegar a alcanzar niveles patéticamente alarmantes.

Rosaura, entre arqueadas\footnote{gebogen, gekrümmt}, tuvo que
abandonar la mesa de honor.

Procura por todos los medios controlar la náusea, ¡pero esta era más
poderosa que ella! Tenía toda la intención de salvar su vestido de novia
de las deposiciones de los parientes y amigos, pero al intentar cruzar
el patio resbaló\footnote{abgleiten, abperlen, abrutschen} y no
hubo un solo pedazo de su vestido que quedara libre de vómito. Un voluminoso
río macilento\footnote{abgezehrt, fahl, hager} la envolvió y la
arrastró algunos metros, provocando que sin poderse resistir más lanzara como un
volcán en erupción estruendosas bocanadas de vómito ante la horrorizada
mirada de Pedro. Rosaura lamentó muchísimo este incidente que arruinó su
boda y no hubo poder humano que le quitara de la mente que Tita había
mezclado algún elemento en el pastel.

Pasó toda la noche entre quejidos y el tormento que le provocaba la
idea de deponer sobre las sábanas que tanto tiempo había tardado en
bordar. Pedro, apresuradamente, le sugirió dejar para otra día la
culminación de la noche de bodas. Pero pasaron meses antes de que Pedro
sintiera la obligación de hacerlo y de que Rosaura se atreviera a
decirle que ya se sentía perfectamente bien. Pedro hasta ese momento
comprendió que no podía rehusarse a realizar su labor de semental\footnote{ Saat$\ndots$ , Zucht$\ndots$ } por más tiempo y esa
misma noche, utilizando la sábana nupcial, se arrodilló frente a su
cama y a manera de rezo\footnote{das Gebet, beten} dijo:
\\- Señor, no es por vicio ni por fornicio sino por dar un hijo a tu %
servicio.\\

Tita nunca imaginó que había tenido que pasar tanto tiempo para que la mentada\footnote{etw. erwähnen, berühmt} boda se consumara\footnote{etw. vollenden, etw. vollbringen, etw. vollziehen}. Ni
siquiera le importó cómo fue y mucho menos si había sido el día de la
ceremonia religiosa o cualquier otro día.

Estaba más preocupada por salvar su pellejo que por otra cosa. Las
noches de la fiesta había recibido de manos de Mamá Elena una
paliza fenomenal, como nunca antes
la había recibido ni la volvería a recibir. Pasó dos semanas en cama
responiéndose de los golpes. El motivo de tan colosal castigo fue la certeza\footnote{die Gewissheit} que tenía Mamá Elena de que Tita, en contubernio\footnote{das Zusammenwohnen} con Nacha, había
planeado premeditadamente\footnote{vorsätzlich, vorbedacht, wissenlich}
arruinar la boda de Rosaura, mezclando algún vomitivo en el pastel. Tita
nunca la pudo convencer de que el único elemento extraño en él fueron las
lágrimas que derramó al prepararlo. Nacha no pudo atestiguar en su favor, pues cuando
Tita había llegado a buscarla el día de la boda la había encontrado
muerta, con los ojos abiertos, chiqueadores\footnote{gekniffen}
en las sienes\footnote{die Schläfe} y la foto de
un antiguo novia en las manos.

\clearpage

\section*{ Capítulo 3 }
\subsection*{ Marzo }
\subsection*{ Codornices en pétalos de rosas }
\textbf{Ingredientes}
\begin{itemize}
    \item 12 rosas, de preferencia rojas
    \item 12 castañas
    \item 2 cucharadas de mantequilla
    \item 2 cucharadas de fécula\footnote{das Stärkemehl, die Stärke} de maíz
    \item 2 gotas de esencia de rosas
    \item 2 cucharadas de anís
    \item 2 cucharadas de miel
    \item 2 ajos
    \item 6 codornices
    \item 1 pithaya
\end{itemize}

\subsection*{ Manera de hacerse: }
Se~\sidenote{55}desprenden\footnote{absplittern, abbröckeln, sich ablösen, etw. abgeben, etw. trennen, etw. losmachen}
con mucho cuidado los pétalos de las rosas,
procurando no pincharse los dedos, pues aparte de que es muy doloroso
(el piquete), los pétalos pueden quedar impregnados de sangre y esto,
aparte de alterar el sabor del platillo, puede provocar reacciones
químicas, por demás peligrosas.

Pero Tita era incapaz de recordar este pequeño detalle ante la intensa
emoción que experimentaba al recibir un ramo de rosas, de manos de
Pedro. Era la primera emoción profunda que sentía desde el día de la
boda de su hermana, cuando escuchó la declaración del amor que Pedro
sentía por ella y que trataba de ocultar a los ojos de los demás. Mamá
Elena, con esa rapidez y agudeza\footnote{die Grelle, die Schärfe, die Bauernschläue}
de pensamiento que tenía, sospechaba para lo que podría pasar si Pedro
y Tita tenían oportunidad de estar a solas. Por tanto, haciendo gala de
asombrosas artes de prestidigitación\footnote{die Gauklerei, die Taschenspielerei},
hasta ahora, se las había ingeniado\footnote{sich etw. ausdenken, etw. erfinden}
la maravilla para ocultar al uno de los ojos y el alcance del otro.
Pero se lo escapó un minúsculo detalle: a la muerte de Nacha, Tita era
entre todas las mujeres de la casa la más capacitada para ocupar el
puesto vacante de la cocina, y ahí escapaban de su riguroso control
los sabores, los olores, las texturas y lo que~\sidenote{56}estas pudieran
provocar.

Tita era el último eslabón\footnote{das Bindeglied, das Glied}
de una cadena de cocineras que desde la época prehispánica se habían
transmitido los secretos de la cocina de generación en generación y estaba
considerada como la mejor exponente de este maravilloso arte, el arte
culinario. Por tanto su nombramiento\footnote{die Ernennung, die Nominierung}
como cocinera oficial del rancho fue muy bien recibido por todo el mundo.
Tita aceptó el cargo con agrado, a pesar de la pena que sentía por la
ausencia de Nacha.

Está lamentable muerte tenía a Tita un estado de depresión muy grande.
Nacha, al morir, la había dejado muy sola. Era como si hubiera muerto su
verdadera madre. Pedro, tratando de ayudarla a salir adelante, pensó que
sería un buen detalle llevarle un ramo de rosas al cumplir su primer año
como cocinera del rancho. Pero Rosaura - que esperaba a su primer hijo -
no opinó lo mismo, y en cuanto lo vio entrar con el ramo en las manos y
dárselo a Tita en vez de a ella, abandonó la sala presa de un ataque de
llanto.

Mamá Elena, con solo una mirada, le ordenó a Tita salir de la sala y
deshacerse de las rosas. Pedro se dio cuento de su osadía\footnote{die Tollkühnheit, die Beherztheit, die Dreistigkeit}
bastante tarde. Pero Mamá Elena, lanzándole la mirada correspondiente,
le hizo saber que aún podía reparar el daño causado. Así que, pidiendo una
disculpa, salió en busca de Rosaura. Tita apretaba\footnote{etw. andrücken, etw. festklemmen, etw. herunterdrücken}
las rosas con tal fuerza contra su pecho que, cuando llegó a la cocina,
las Rosas, que en un principio eran de color rosada, ya se habían vuelto
rojas por la sangre de las manos y pecho de Tita. Tenía que
pensar~\sidenote{57}rápidamente qué hacer con ellas. ¡estaban tan hermosas!
No era posible tirarlas en la basura, en primera porque nunca antes había
recibido flores y en segunda porque se las había dado Pedro. De pronto
escuchó claramente la voz de Nacha, dictándole al oído una receta
prehispánica donde se utilizaban pétalos de rosas. Tita la tenía medio
olvidada, pues para hacerla se necesitaban faisanes\footnote{der Fasan} y en el rancho nunca se
habían dedicado a criar ese tipo de aves.

Lo único que tenían en ese momento eran codornices, así que decidió
alterar ligeramente la receta, con tal de utilizar las flores.

Sin pensarlo más salió al patio y se dedicó a perseguir\footnote{jmdn./etw. verfolgen, jmdm./etw. nacheilen, jmdm./etw. nachschleichen}
codornices. Después de atrapar a seis de ellas las metió a la cocina
y se dispuso a matarlas, lo cual no le era nada fácil después de
haberlas cuidado y alimentado por tanto tiempo.

Tomando una gran respiración, agarró a la primera y le retorció\footnote{etw. krümmen, etw. winden, etw. verdrehen} el pescuezo\footnote{das Genick} como había visto a Nacha hacerlo
tantas veces, pero con tan poca fuerza que la pobre codorniz no murió,
sino que se fue quejando lastimeramente por toda la cocina, con la cabeza
colgando de lada. ¡Esta imagen la horrorizó! Comprendió que no se podía
ser débil en esto de la matada: o se hacía con firmeza o solo se causaba
un gran dolor. En este momento pensó en lo bueno que sería tener la
fuerza de Mamá Elena. Ella mataba así, de tajo\footnote{das Holzklötzchen, der Schnitt, des Schneidebrett, etw. schneiden},
sin piedad. Bueno, aunque pensándolo bien, no. Con ella había hecho
una excepción, la había empezado a matar desde niña, poco a poquito,
y aún no le daba el golpe final. La boda de Pedro con Rosaura la había
dejado~\sidenote{58}como la codorniz, con la cabeza y el alma fracturadas, y
antes de permitir que la codorniz sintiera los mismos dolores que ella,
en un acto de piedad, con gran decisión, rápidamente la ultimó.

Con las demás todo fue más fácil. Solo trataba de imaginar que cada una
de las codornices tenía atorado\footnote{stocken, stecken bleiben, verstopfen}
un huevo tibio en buche\footnote{der Labmagen, der Magen, der Kropf}
y que ella piadosamente\footnote{barmherzig, fromm} las liberaba
de ese martirio dándoles un buen torzón.
Cuando niña, muchas veces deseó morir antes que desayunar el
consabido y
obligatorio huevo tibio. Mamá Elena una la obligaba a comerlo. Ella
sentía que el esófago\footnote{die Speiseröhre} se le
cerraba fuerte, muy fuerte, incapaz de poder deglutir\footnote{etw. schlucken, etw. hinunterschlingen}
alimento alguno, hasta que su madre le propinaba\footnote{jmdm. etw. versetzen, jmdm. etw. verabreichen}
un coscorrón\footnote{die Kopfnuss} que tenía el efecto milagroso
de desbaratarle el nudo en la
garganta, por la que entonces se deslizaba\footnote{ausrutschen, rutschen, herunterrutschen}
el huevo sin ningún problema. Ahora se sentía más tranquila y los siguientes
pasos los realizó con gran destreza\footnote{die Fingerfertigkeit, die Geschicklichkeit}.

Tal parecía que era la misma Nacha la que en el cuerpo de Tita realizaba
todas estas actividades: desplumar las aves en seco, sacarles las vísceras\footnote{die Eingeweide} y ponerlas a freír.

Después de desplumadas y vaciadas las codornices, se les recogen y atan
las patas, para que conserven una posición graciosa mientras se ponen a
dorar en la mantequilla, espolvoreadas con pimienta y sal al gusto.

Es importante que se desplume a las codornices en seco, pues el
sumergirlas en agua hirviendo altera el sabor de la carne. Este es uno
de los innumerables secretos de la cocina que solo se adquieren con la
práctica. Cómo Rosaura no había querido participar de las actividades
culinarias desde~\sidenote{59}que se quemó las manos en el comal, lógicamente
ignoraba este y muchos otros conocimientos gastronómicos. Sin embargo,
quién sabe si por querer impresionar a Pedro, su esposo, o por querer
establecer una competencia con Tita en sus terrenos, en una ocasión
intentó cocinar. Cuando Tita amablemente quiso darle algunos consejos,
Rosaura se molestó enormemente y le pidió que la dejara sola en la
cocina.

Obviamente el arroz se le batió, la carne de le saló\footnote{etw. salzen, etw. versalzen}
y el postre se le quemó. Nadie en la mesa se atrevió a mostrar ningún
gesto de desagrado, pues Mamá Elena a manera de sugerencia
había comentado:
\\- Esa primera vez que Rosaura cocina y opino que no lo hizo tan mal. %
¿Qué opina usted, Pedro?\\
\rule{1em}{0pt}Pedro, haciendo un soberano\footnote{eigenstaatlich, souverän} esfuerzo,
respondió sin ánimo de lastimar a su esposa:
\\- No, para ser la primera vez no está tan mal.\\

Por supuesto esa tarde toda la familia se enfermó del estómago.

Fue un verdadera tragedia, claro que no tanta como la que se suscitó\footnote{etw. hervorrufen, etw. auslösen, etw. erregen}
en el rancho ese día. La fusión de la sangre de Tita con los pétalos
de las rosas que Pedro le había regalado resultó ser de lo más explosiva.

Cuando se sentaron a la mesa había un ambiente ligeramente tenso, pero
no pasó a mayores hasta que se sirvieron las codornices. Pedro, no
contento con haber provocado los celos de su esposa, sin poderse
contener, al saborear el primer bocado del platillo, exclamó, cerrando
los ojos con verdadera lujuria:
\\- ¡Este es un placer de los dioses!\\

Mamá Elena~\sidenote{60}, aunque reconocía que se trataba de un guiso\footnote{das Gericht, das Geschmorte}
verdaderamente exquisito, molesta por el comentario dijo:
\\- Tiene demasiada sal.\\
\rule{1em}{0pt}Rosaura, pretextando náuseas y mareos, no pudo comer más que tres
bocados. En cambio Gertrudis algo raro le pasó
parecía que el alimento que estaba ingiriendo\footnote{etw. schlucken, etw. einnehmen}
producía en ella un efecto afrodisíaco pues empezó a sentir que un intenso
calor le invadía las piernas. Un cosquilleo en el centro de su cuerpo
no la dejaban estar correctamente sentada en su silla. Empezó a sudar y
a imaginar que se sentiría al ir sentado al lomo de un caballo, abrazada
por un villista, uno de esos que había visto una semana antes entrando
a la plaza del pueblo, oliendo a sudor, a tierra, amaneceres de peligro e incertidumbre\footnote{die Ungewissheit, die Unklarheit, die Unsicherheit},
a vida y a muerte. Ella iba al mercado en compañía de
Chencha la sirvienta cuando lo vio entrar por la calle principal de
Piedras Negras. Venía al frente de todos, obviamente capitaneando\footnote{jmdn./etw. anführen, jmdn. leiten}
a la tropa. Sus miradas se encontraron y lo que vio en los ojos de él la hizo
temblar. Vio muchas noches junto al fuego deseando la compañía de una
mujer a la cual pudiera besar, una mujer a la que pudiera abrazar, una
mujer \ndots como ella. Sacó su pañuelo y trató de que junto con el
sudor se fueran de su mente todos esos pensamientos pecaminosos\footnote{sündhaft}.

Pero era inútil, algo extraño le pasaba. Trató de buscar apoyo en Tita
pero ella estaba ausente, su cuerpo estaba sobre la silla, sentado, y
muy correctamente, por cierto, pero no había ningún signo de vida en sus
ojos. Tal parecía~\sidenote{61}que en un extraño fenómeno de alquimia su ser se
había disuelto en la salsa de las rosas, en el cuerpo de las codornices,
en el vino y en cada uno de los olores de la comida. De esta manera
penetraba en el cuerpo de Pedro, voluptuosa\footnote{wollüstig},
aromática, calurosa, completamente sensual\footnote{sinnlich}.

Parecía que habían descubierto un código nuevo de comunicación en el que
Tita era la emisora, Pedro el receptor y Gertrudis la afortunada en
quien se sintetizaba ese singularización sexual, a través de la comida.

Pedro no opuso resistencia, la dejó entrar hasta el último rincón
suceder sin poder quitarse la vista el uno del otro. Le dijo:
\\- Nunca había probado algo tan exquisito, muchas gracias. %
Es que verdaderamente este platillo es delicioso. Las rosas le %
proporcionan un sabor de lo más refinado\footnote{etw. verfeinern, etw. veredeln, raffiniert, kultiviert}.\\

Ya que se tienen los pétalos deshojados\footnote{etw. entblättern, etw. entlauben}
se muelen en el molcajete\footnote{der Mörser} junto
con el anís. Por separado, las castañas se ponen a dorar en el comal, descascaran\footnote{etw. schälen, etw. enthülsen} y se
cuecen en agua. Después, se hacen puré. Los ajos se
pican finamente y se doran en la mantequilla; cuando están acitronados\footnote{Zwiebeln glasig dünsten},
se les agregan el puré de castañas, la pitaya molida, la miel, los
pétalos de rosa y sal al gusto. Para que espese un poco la salsa se le
puede añadir dos cucharaditas de fécula de maíz. Por último, se pasa por
un tamiz y se agregan solo dos gotas de esencia de rosas, no más, pues se
corre el peligro de que quede muy olorosa y pasada el sabor. En cuanto
está sazonada\footnote{etw. reifen lassen, etw. abschmecken, etw. würzen}
se retira del fuego. Las codornices solo se sumergen
durante diez minutos en esta salsa para que se impregnen\footnote{etw. durchtränken, etw. imprägnieren}
y se sacan.

El aroma de la esencia de rosas es tan penetrante que el molcajete que
se utilizaba para moler los pétalos quedaba impregnado por varios días.

La encargada de lavarlo junto con los demás trastes\footnote{der Gittarenbund, der Kram, das Gerümpel, das Geschirr}
que utilizaban en la cocina era Gertrudis. Esta labor la realizaba después
de comer, en el patio, pues aprovechaba para echar a los animales la comida
que había quedado en las ollas además, como los trastes de cocina eran tan
grandes, los lavaba mejor en el fregadero\footnote{der Ausguss, das Spülbecken}.
Pero el día de las codornices no lo hizo, le pidió de favor a Tita que lo hiciera a por ella.
Gertrudis realmente se sentía indispuesta, sudaba copiosamente por todo
el cuerpo. Las gotas que le brotaban eran de color rosado y tenían un
agradable y penetrante olor a rosas. Sintió un imperiosa\footnote{dringend}
necesidad de darse un baño y corrió a prepararlo.

En la parte trasera del patio, junto a los corrales\footnote{der Geflügelhof, der Hof}
y el granero\footnote{die Scheune, der Getreidespeicher}, Mamá
Elena había mandado instalar una regadera\footnote{die Gießkanne, die Dusche}
rudimentaria. Se trataba de un pequeño cuarto construido con tablones\footnote{die Hinweistafel, das Brett}
unidos, solo que entre uno y otro quedaban hendiduras\footnote{der Spalt, die Kerbe}
lo suficientemente grandes como para ver, sin mayor problema, al que
estuviera tomando el baño. De cualquier manera era la
primera regadera de la que el pueblo tuvo noticia. La había inventado un
primo de Mamá Elena que vivía en San Antonio, Texas. Tenía un caja como
a dos metros de altura con capacidad para cuarenta litros, a la cual se le
tenía que depositar el agua con anterioridad, para que pudiera funcionar
utilizando la fuerza de gravedad. Costaba trabajo subir las cubetas
llenas de~\sidenote{63}agua por una escalera de madera, pero después era un
delicia solo abrir una llave y sentir correr el agua por todo el cuerpo
de un solo golpe y no en abonos\footnote{das Abonnement}, como sucedía
cuando uno se bañaba y jicarazos\footnote{etw übereilt erledigen}.
Años después los gringos le pagaron un bicoca\footnote{die Lappalie, die Belanglosigkeit}
al primo por su invento y lo perfeccionaron. Fabricaron miles de regaderas sin necesidad
del mentado\footnote{jmdn./etw. erwähnen, jmdn./etw. nennen} depósito\footnote{die Anzahlung, der Aufbewahrungsort},
pues utilizaron tubería para que funcionaran.

¡Si Gertrudis hubiera sabido! La pobre subió y bajó como diez veces
cargando las cubetas. Estuvo a punto de desfallecer\footnote{ermatten, schwach werden, in Ohnmacht fallen}
pues este brutal ejercicio intensificaba el abrasador calor que sentía.

Lo único que la animaba era la ilusión del refrescante baño que la
esperaba, pero desgraciadamente no lo pudo disfrutar pues las gotas que
caían de la regadera no alcanzaban a tocarle el cuerpo, se evaporaban
antes de rozarla\footnote{jmdn./etw. leicht berühren, jmdn./etw. streifen}
siquiera. El calor que despedía\footnote{sich von jmdm./etw. verabschieden, etw. aussenden, etw. ausstrahlen, etw. ausströmen}
su cuerpo era tan intenso que las maderas empezaron a tronar\footnote{donnern}
y a arder. Ante el pánico de morir abrasada por las llamas salió corriendo
del cuartucho\footnote{das Loch (Behausung)},
así como estaba, completamente desnuda.

Para entonces el olor a rosas que su cuerpo despedía había llegado muy,
muy lejos. Hasta las afueras del pueblo, en dónde revolucionarios y
federales libraban\footnote{jmdn./etw. von etw. befreien, jmdn./etw. von etw. erlösen, retten}
una cruel batalla. Entre ellos sobresalía\footnote{herausstehen, hervorragen}
por su valor el villista ese, el que había entrado una semana
antes a Piedras Negras y se había cruzado con ella en la plaza.

Una nube rosada llegó hasta él, lo envolvió y provocó que saliera a
todo galope hacía el rancho de Mamá Elena. Juan, que así se llamaba el sujeto\footnote{der Mensch, das Ding, die Person}, abandonó el
campo de batalla dejando atrás a un enemigo a medio morir~\sidenote{64}, sin saber
para qué. Una fuerza superior controlaba sus actos. Lo movía una poderosa
necesidad de llegar lo más pronto posible al encuentro de algo desconocido
en un lugar indefinido. No le fue difícil dar. Lo guiaba el olor del cuerpo
de Gertrudis. Llegó justo a tiempo para descubrirla corriendo en medio del
campo. Entonces supo para qué había llegado hasta allí. Esta mujer necesitaba
imperiosamente que un hombre le apagará el fuego abrasador que nacía en
sus entrañas\footnote{die Eingeweide}.

Un hombre igual de necesitado de amor que ella, un hombre como él.

Gertrudis dejó de correr en cuanto lo vio venir hacia ella. Desnuda
como estaba, con el pelo suelto cayéndole hasta la cintura y irradiando\footnote{etw. verstrahlen, etw. durchstrahlen}
una luminosa energía, representaba lo que sería una síntesis entre una
mujer angelical\footnote{engelhaft} y una infernal\footnote{höllisch, infernal}.
La delicadeza de su rostro y la perfección de su inmaculado y virginal
cuerpo contrastaban con la pasión y la lujuria que le salían atropelladamente
por los ojos y los poros.
Estos elementos, aunados\footnote{etw. verbinden, etw. vereinen}
al deseo sexual que Juan por tanto tiempo había
contenido por estar luchando en la sierra, hicieron que el encuentro
entre ambos fuera espectacular.

Él, sin dejar de galopar para no perder tiempo. Se inclinó, la tomó de
la cintura, la subió al caballo delante de él, pero acomodándola\footnote{etw. in Ordnung bringen, sich auf etw. einstellen } frente
a frente, Y se la llevó. El caballo, aparentemente\footnote{anscheinend, scheinbar}
siguiendo también órdenes superiores, siguió galopando como si supiera perfectamente cuál
era su destino final, a pesar de que Juan había soltado las riendas\footnote{der Zügel} para
poder abrazar y besar apasionadamente a Gertrudis. El movimiento del
caballo se confundía con el de sus cuerpos mientras se realizaban su
primera copulación a todo galope y con alto grado de~\sidenote{65}dificultad.

Todo fue tan rápido que la escolta\footnote{die Eskorte, das Geleit}
que seguía a Juan tratando de interceptarlo
nunca lo logró. Decepcionados dieron media vuelta y el
informe que llevaron fue que el capitán había enloquecido repentinamente
durante la batalla y que por esta causa había desertado del ejército.

Generalmente, esa es la manera en que se escribe la historia, a través
de las versiones de los testigos presenciales, que no siempre
corresponden a la realidad. Pues el punto de vista de Tita sobre lo acontecido\footnote{geschehen, vorkommen, vorfallen, betrübt, traurig}
era totalmente diferente al de estos revolucionarios. Ella
había observado todo de este espacio donde estaba lavando los trastes.
No perdió detalle a pesar de que le interferían\footnote{stören, interferieren}
la visión una nube de vapor rosado y las llamas del cuarto de baño.
A su lado, Pedro también tuvo la suerte de contemplar el espectáculo,
pues había salido el patio por su bicicleta para ir a dar un paseo.

Y como mudos\footnote{stumm} espectadores de una película, Pedro y
Tita se emocionaron hasta las lágrimas al ver a sus héroes realizar el
amor que para ellos estaba prohibido. Hubo un momento, un solo instante
en que Pedro pudo haber cambiado el curso de la historia. Tomando a Tita
de la mano alcanzó a pronunciar: \glqq{}Tita \ndots.\grqq{}. Solo eso. No tuvo tiempo
de decir más. La sucia realidad se lo impidió. Se escuchó un grito de
Mamá Elena preguntando qué era lo que pasaba en el patio. Si Pedro le
hubiera pedido a Tita huir con él, ella no no hubiera pensado ni tantito,
pero no lo hizo, sino que montando rápidamente en la bicicleta se fue
pedaleando su~\sidenote{66}rabia.

No podía borrar de su mente la imagen de Gertrudis corriendo por el
campo \ndots ¡completamente desnuda! Su grandes senos bamboleándose\footnote{jmdn./etw. schaukeln, jmdn./etw. schwingen} de
un lado a otro lo habían dejado hipnotizado. Èl nunca había visto a una
mujer desnuda. En la intimidad con Rosaura no había sentido deseos de
verle el cuerpo ni de acariciárselo. En estos casos siempre utilizaban
la sábana nupcial, que solo dejaba visibles las partes nobles de su
esposa. Terminado el acto, se alejaba de la recámara antes de que esta
se descubriera. En cambio, ahora, se había despertado en él la
curiosidad de ver a Tita por largo rato así, sin ninguna ropa.
 Indagando\footnote{etw. erforschen, etw. ermitteln}, husmeando\footnote{etw. schnüffeln, etw. schnuppern}, averiguando
cómo era hasta el último centímetro de piel de su monumental y atractivo
cuerpo. De seguro que se parecía al de Gertrudis, no en balde eran hermanas.

La única parte del cuerpo de Tita que conocía muy bien, aparte de la
cara y las manos, era el redondo trozo de pantorrilla\footnote{die Wade}
que había alcanzado a verle en una ocasión. Ese recuerdo lo atormentaba
por las noches. Qué antojo\footnote{die Laune, das Belieben} sentía
de poner su mano sobre eso trozo de piel y luego por todo el cuerpo tal
y como había visto hacerlo al hombre que se llevó a Gertrudis:
¡con pasión, con desenfreno\footnote{die Hemmungslosigkeit, die Zügellosigkeit, sich nicht zügeln, hemmungslos, zügellos},
con lujuria!

Tita, por su parte, intentó gritarle a Pedro que la esperara, que se la
llevara lejos, adonde los dejaran amarse, adonde aún no hubieran
inventado reglas que seguir y respetar, adonde no estuviera su madre,
pero su garganta no emitió ningún sonido. Las palabras se le hicieron
nudo y se ahogaron unas a otras antes de salir.

¡Se sentía~\sidenote{67}tan sola y abandonada! Un chile en nogada\footnote{Soße aus Walnüssen und Gewürzen, in der Fisch gekocht wird}
olvidado en una charola después de un gran banquete no se sentiría peor
que ella.
Cuántas veces sola en la cocina se había tenido que comer una de estas
delicias antes de permitir que se echara a perder. El que nadie se coma
el último chile de una charola generalmente sucede cuando la gente no
quiere demostrar su gula\footnote{die Völlerei} y aunque les
encantaría devorarlo\footnote{etw. auffressen, etw. verschlingen},
nadie se atreve. Y es así como se rechaza a un chile relleno que contiene
todos los sabores imaginables, lo dulce de acitrón, lo picoso del chile, lo sutil\footnote{feinsinnig, subtil} de la nogada, lo refrescante de la granada,
¡un maravilloso chile en nogada! Que contiene\footnote{etw. zügeln, sich etw. verkneifen, etw. beherrschen}
el interior todos los secretos del amor, pero que nadie podrá desentrañar\footnote{etw. ergründen, etw. erfassen, etw. herausbekommen}
a causa de la decencia\footnote{die Anständigkeit, der Anstand}.

¡Maldita decencia! ¡Maldito Manual de Carreño! Por su culpa su cuerpo
quedaba destinado a marchitarse\footnote{ermatten, welk werden}
poco a poco, sin remedio alguno. ¡Y maldito Pedro tan decente, tan correcto,
tan varonil\footnote{männlich}, tan \ndots, tan amado.

Sí Tita hubiera sabido entonces que no tendrían que pasar muchos años
para que su cuerpo conociera el amor no se habría desesperado tanto en
ese momento.

El segundo grito de mamá Elena la sacó de sus cavilaciones\footnote{der Kniff, das Nachdenken, die Grüblerei}
y la hizo buscar rápidamente una respuesta. No sabía qué era lo que le iba
a decir a su mamá, sí primero la decía que estaba ardiendo la parte trasera
del patio, o que Gertrudis se había ido con un villista al lomo de caballo
\ndots y desnuda.

Se decidió cortar una versión en la cual los federales, a los que Tita aborrecía\footnote{jmdn. hassen, jmdn./etw. verabscheuen},
habían entrado en tropel\footnote{das Durcheinander, das Wirrwarr, das Getümmel, die Eile, die Hast},
habían prendido\footnote{etw. anmachen, etw. anstecken} fuego a
los paños y habían raptado a Gertrudis. Mamá Elena se creyó~\sidenote{68}toda la historia y
enfermó de la pena, pero estuvo a punto de morir cuando se enteró una
semana después por boca de padre Ignacio, el párroco\footnote{der Pfarrer}
del pueblo - que quién sabe cómo se enteró -, que Gertrudis estaba trabajando en un burdel\footnote{das Bordell} en la frontera. Prohibió volver a mencionar
el nombre de su hija y mandó quemar sus fotos y su acta de nacimiento.

Sin embargo ni el fuego ni el paso de los años han podido borrar el
penetrante olor de rosas que despide el lugar donde antes estuvo la
regadera y que ahora es el estacionamiento\footnote{das Parken, der Parkplatz, die Stationierung}
un edificio de departamentos.
Tampoco pudieron borrar de la mente de Pedro y de Tita las imágenes que
observaron y que los marcaron para siempre. Desde ese día las codornices
en pétalos de rosas se convirtieron en un muro recuerdo de esta
experiencia fascinante.

Tita lo preparaba cada año como ofrenda a la libertad que su hermana
había alcanzado y ponía especial esmero\footnote{die Sorgfalt}
en el decorado de las codornices.

Estás se ponen en un platón\footnote{die Schüssel}, se
les vacía la salsa encima y se
decoran con una rosa completa en el centro y pétalos a los lados, o se
pueden servir de una vez en un plato individual en lugar de utilizar el
platón. Tita así lo prefería, pues de esta manera no corría el riesgo de
que a la hora de servir la codorniz se perdiera el equilibrio del
decorado. Precisamente así lo especificó en el libro de cocina que
empezó a escribir esa misma noche, después de tejer un buen tramo de su
colcha, como diariamente lo hacía. Mientras la tejía, en su cabeza daban
vueltas y vueltas las imágenes de Gertrudis corriendo por el campo~\sidenote{69}junto con otras que ella imaginaba sobre lo que habría pasado más tarde,
cuando se le perdió de vista su hermana. Claro que su imaginación era en
este aspecto bastante limitada, por su falta de experiencia.

Tenía curiosidad de saber si ya tendría algo de ropa encima, o seguiría
así de \ndots ¡desabrigada\footnote{ungeschützt}! Le preocupaba
que pudiera sentir frío, al igual que ella, pero llegó a la conclusión de
que no. Lo más probable era que estuviera cerca del fuego, en los brazos
de su hombre, y eso definitivamente debería dar calor.

De pronto una idea que cruzó por su mente la hizo levantarse a mirar al
cielo estrellado. Ella conocía, pues lo había sentido en carne propia,
lo poderoso que puede ser el fuego de una mirada.

Es capaz de encender al mismo sol. Tomando esto en consideración, ¿qué
pasaría si Gertrudis miraba una estrella? De seguro que el calor de su
cuerpo, inflamado por el amor, viajaría con la mirada a través del
espacio infinito sin perder su energía, hasta depositarse en
el lucero\footnote{der Stern, die Blesse}
de su atención. Estos grandes astros\footnote{das Gestirn} han
sobrevivido millones de años gracias a que se cuidan mucho de no absorber
los rayos ardientes que los amantes de todo el mundo les lanzan noche
tras noche. De hacerlo, si generaría tanto calor que estallarían en mil
pedazos. Por lo que al recibir una mirada, la rechazan de inmediato,
reflejándola hacia la tierra como en un juego de espejos. Es por eso
que brillan tanto en las noches. Y es por eso que a Tita le entró la
esperanza de que si ella pudiera descubrir entre todos las estrellas del
firmamento cuál era la que su hermana veía en ese momento, recibiría
por reflejo un poco~\sidenote{70}del calor que a ella le sobraba.

Bueno, esa era su ilusión, pero por más que observó una a una todas las
estrellas del cielo no sintió absolutamente nada de calor, sino más bien
todo lo contrario. Estremecida regresó a su cama plenamente convencida
de que Gertrudis dormía plácidamente con los ojos bien cerrados y que
por eso no funcionó el experimento. Se cubrió entonces con su colcha,
que ya para entonces se doblaba en tres, revisó la receta que había
escrito para ver si no se le olvidaba apuntar algo y añadió: \glqq{}hoy que
comimos este platillo, huyó de la casa Gertrudis \ndots\grqq{}.

\clearpage

\section*{ Capítulo IV }
\subsection*{ Abril }
\subsection*{ Mole de guajolote con almendra y ajonjolí\footnote{der Sesam} }
\textbf{ingredientes}

\begin{itemize}
    \item ¼ de chile mulato
    \item 3 chiles pasilla
    \item 3 chiles anchos
    \item un puño\footnote{die Faust} de almendras
    \item un puño de ajonjolí
    \item caldo de guajolote\footnote{der Truthahn}
    \item un bizcocho\footnote{der Biskuit, der Sandkuchen, der Rührkuchen} (/s de concha)
    \item cacahuetes
    \item ½ cebolla
    \item dos tablillas\footnote{das Armtragetuch, das Täfelchen, das Brettchen} de chocolate
    \item vino
    \item anís
    \item manteca
    \item clavo\footnote{der Nagel, die Gewürznelke, die Nelke}
    \item canela
    \item pimienta
    \item azúcar
    \item semilla de los chiles
    \item 5 dientes de ajo
\end{itemize}

\section*{ Manera de hacerse: }
Después~\sidenote{75}de dos días de matado el guajolote, se limpia y se pone a
cocer con sal. La carne de los guajolotes es sabrosa y aún exquisit si
se ha cebado\footnote{etw. beködern, etw. nähren, etw. auffüllen}
cuidadosamente. Esto se logra teniendo a las aves en corrales limpios,
con grano y agua en abundancia.

Quince días antes de matar a los guajolotes, se les empieza a alimentar con
nueces pequeñas. Comenzando el primer día con una, al siguiente se les
echan en el pico y así sucesivamente se les va aumentando la ración,
hasta la víspera\footnote{der Vorabend, der Vortag} de matarse,
sin importar el maíz que coman voluntariamente en ese tiempo.

Tita tuvo mucho cuidado en cebar a los guajolotes apropiadamente, pues
le interesaba mucho quedar bien en la fiesta tan importante a celebrarse
en el rancho: el bautizo de su sobrino, el primer hijo de Pedro y
Rosaura. Este acontecimiento ameritaba\footnote{etw. verdienen, etw. auszeichnen}
una gran comida con mole\footnote{großer Umfang, massige Gestalt, die Masse}.
Para la ocasión se había mandado a hacer una vajilla\footnote{das Essgeschirr}
de barro con el nombre de Roberto, que así se llamaba el agraciado\footnote{anmutig, begünstigt, auserwählt} bebé,
quién no paraba de recibir las atenciones los regalos de familiares y amigos.
En especial de parte de Tita, quien en contra de lo que se esperaba,
sentía~\sidenote{76}un inmenso cariño por este niño, olvidando por completo que era el
resultado del matrimonio de su hermana con Pedro, el amor de su vida.

Con verdadero entusiasmo se dispuso a preparar con un día de
anterioridad el mole para el bautizo. Pedro la escuchaba desde la sala
experimentando una nueva sensación para él. El sonido de las ollas al
chocar unas contra otras, el olor de las almendras dorándose en el
comal, la melodiosa voz de Tita, que cantaba mientras cocinaba, habían
despertado su instinto sexual. Y así como los amantes saben que se
aproxima el momento de una relación íntima, ante la cercanía, el olor de
ser amado, o las caricias recíprocas\footnote{gegenseitig, beiderseiting, reziprok}
en un previo juego amoroso, así estos sonidos y olores, sobre todo el de
ajonjolí dorado, le anunciaban a Pedro la proximidad de un verdadero
placer culinario.

Las almendras y el ajonjolí se tuestan\footnote{etw. toasten, sich bräunen} en comal.
Los chiles anchos, desvenados\footnote{Chilischoten putzen},
también se tuestan, pero no mucho para que no se amarguen\footnote{bitter schmecken}.
Esto se tiene que hacer en un sartén aparte, pues se les pone un poco de
manteca para hacerlo. Después se muelen en metate\footnote{die Reibmühle} junto con los
almendras y el ajonjolí.

Tita, de rodillas, inclinada sobre el metate, se movía rítmica y cadenciosamente\footnote{rhythmisch, harmonisch, wolhlklingend} mientras
molía las almendras y el ajonjolí.

Bajo su blusa sus senos se meneaban libremente pues ella nunca usó
sostén alguno. De su cuello escurrían\footnote{ablaufen, abtropfen} gotas
de sudor que rodaban\footnote{rollen, kullern} hacia abajo siguiendo el surco\footnote{die Ackerfurche, die Furche, die Falte} piel entre sus
pechos redondos y duros.

Pedro,~\sidenote{77}no pudiendo resistir los olores que emanaban\footnote{etw. abgeben, etw. ausdünsten}
de la cocina, se dirigió hacia ella, quedando petrificado\footnote{versteinert}
en la puerta ante la sensual postura\footnote{die Körperhaltung} en que
encontró a Tita.

Tita levantó la vista sin dejar de moverse y sus ojos se encontraron con
los de Pedro. Inmediatamente, sus miradas enardecidas\footnote{entfachen, erregen, begeistern}
se fundieron\footnote{schmelzen, zerfließen} de
tal manera que quien lo hubiera visto solo habría notado una sola
mirada, un solo movimiento rítmico y sensual, una sola respiración
agitada y un mismo deseo.

Permanecieron en éxtasis amoroso hasta que Pedro bajó la vista y la
clavó en los senos de Tita. Esta dejó de moler, se enderezó\footnote{sich aufrichten, etw. gerade biegen, etw. gerade machen}
y orgullosamente y irguió\footnote{sich aufrichten, sich erheben, etw. hochhalten, etw. aufrecken, etw. heben}
su pecho, para que Pedro lo observara plenamente. El examen de que fue objeto
cambió para siempre la relación entre ellos. Después de esa escrutadora
mirada que penetraba más la ropa llamar a volvería a ser igual. Tita supo
en carne propia por qué el contacto con el fuego altera los elementos,
por qué un pedazo de masa se convierte en tortilla, por qué un pecho sin
haber pasado por el fuego del amor es un pecho inerte\footnote{reglos, leblos},
una bola de masa sin ninguna utilidad. En solo unos instantes Pedro había
transformado los senos de Tita, de castos\footnote{keusch} a voluptuosos,
sin necesidad de tocarlos.

De no haber sido por la llegada de Chencha, que había ido al mercado
por los chiles anchos, quién sabe qué hubiera pasado entre Pedro y Tita;
tal vez Pedro hubiera terminado amasando\footnote{etw. kneten, etw. anmachen}
sin descanso los senos que Tita le ofrecía pero, desgraciadamente, no
fue así. Pedro, fingiendo haber ido por un vaso de agua de limón con Chía,
lo tomó rápidamente y salió de la cocina.

Tita, con~\sidenote{78}manos temblorosas, trató de continuar con la elaboración
del mole como si nada hubiera pasado.

Cuando ya están bien molidas las almendras y el ajonjolí, se mezclan
con el caldo donde se coció el guajolote y se le agrega sal al gusto. En
un molcajete se muelen el clavo, la canela, el anís, la pimienta y, por
último, el bizcocho, que anteriormente se ha puesto a freír en manteca
junto con la cebolla picada y el ajo.

Enseguida se mezclan con el vino y se incorporan. Mientras molía las
especias, Chencha trataba en vano\footnote{vergeblich, umsonst} de
capturar el interés de Tita. Pero por más que le exageró los incidentes\footnote{der Zwischenfall, der Vorfall}
que había presenciado en la plaza y le narraba con lujo de detalles la
violencia de las batallas que tenían lugar en el pueblo, solo alcanzaba
a interesar a Tita por breves momentos.

Esta, por hoy, no tenía cabeza para otra cosa que no fuera la emoción
que acababa de experimentar. Además de que Tita conocía perfectamente
cuáles eran los móviles de Chencha al decirle estas cosas. Como ella ya
no era la niña que se asustaba con las historias de la llorona, la bruja
que chupaba a los niños, el coco y demás horrores, ahora Chencha trataba
de asustarla con historias de colgados\footnote{aufgehängt}, fusilados\footnote{etw. (schlecht) nachahmen, etw. plagiieren, jmdn. standrechtlich erschießen, jmdn. niederschießen}, desmembrados\footnote{jmdn./etw. zergliedern, zerstückeln, zerlegen}, degollados\footnote{jmdm./etw. die Kehle aufschneiden, jmdm./etw. enthaupten}
e inclusive sacrificados a los que se les sacaba el corazón
¡en pleno campo de batalla! En otro momento le hubiera gustado caer en
el sortilegio\footnote{die Hexerei, die Zauberei} de la graciosa
narrativa de Chencha y termina por creerle sus mentiras, inclusive la de que
a Pancho Villa le llevaban los corazones sangrantes de sus enemigos para que
se los comiera, pero no ahora.

La mirada~\sidenote{79}de Pedro la había hecho recuperar confianza en el amor
que este le profesaba\footnote{etw. ausüben, sich zu etw. bekennen}.
Había pasado meses envenenada\footnote{vergiftet} con la
idea de que o Pedro le había metido el día de la boda al declararle
su amor solo para no hacerla sufrir, o que con el tiempo Pedro realmente
se había enamorado de Rosaura. Esta inseguridad había nacido cuando él, inextricablemente\footnote{unentwirrbar, unlösbar, undurchdringlich, verwicket},
había dejado de festejarle sus platillos. Tita se esmeraba\footnote{etw. putzen, etw. polieren, sich um etw. bemühen, sich bei etw. (größte) Mühe geben}
con angustia en cocinar cada día mejor. Desesperada, por las
noches, obviamente después de tejer un buen tramo de su colcha,
inventaba una nueva receta con la intención de recuperar la relación que
entre ella y Pedro había surgido a través de la comida. De está épocas
de sufrimiento nacieron sus mejores recetas.

Y así como un poeta juega con las palabras, así ella jugaba a su antojo
Con los ingredientes y con las cantidades, obteniendo resultados
fenomenales. Pero nada, todos sus esfuerzos eran en vano. No lograba
arrancar de los labios de Pedro una sola palabra de aprobación. Lo que
no sabía es que mamá Elena le había \glqq{}pedido\grqq{} a Pedro que se abstuviera\footnote{sich enthalten, auf etw. verzichten}
de elogiar\footnote{jmdn./etw. loben, anpreisen} la
comida, pues Rosaura de por si sufría de inseguridad, por
estar gorda y deforme a causa de su embarazo, como para encima de todo
tener que soportar los cumplidos que él le hacía a Tita so pretexto de
lo delicioso que ella cocinaba.

Qué sola se sintió Tita en esa época. ¡Extrañaba tanto a Nacha! Odiaba a
todos, inclusive a Pedro. Estaba convencida de que nunca volvería a
querer a nadie mientras viviera. Claro que todas estas convicciones se esfumaron\footnote{verschwimmen, verlaufen, verblassen, verduften}
en cuanto recibió en sus propias manos el hijo de Rosaura.

Fue una mañana fría de marzo~\sidenote{80}, ella estaba en el gallinero
recogiendo los huevos que las gallinas acababan de poner, para
utilizarlos en el desayuno. Algunos aún estaban calientes, así que se
los metía bajo de la blusa, pegándoselos al pecho, para mitigar\footnote{etw. abschwächen, mildern, lindern}
el frío crónico qué sufría y que últimamente se le había agudizado.
Se había levantado antes que nadie, como de costumbre.

Pero hoy lo había hecho media hora antes de lo acostumbrado, para empacar\footnote{etw. etw. einpacken, etw. packen, etw. verpacken}
una maleta con la ropa de Gertrudis. Quería aprovechar que
Nicolás salía de viaje a recoger un granado para pedirle que por favor
se la hiciera llegar a su hermana. Por supuesto, esto lo hacía a
escondidas de su madre. Tita decidió enviársela pues no se le quitaba de
la mente la idea de que Gertrudis seguía desnuda. Claro que Tita se
negaba a aceptar como cierto que esto fuera porque el trabajo de su
hermana en el burdel de la frontera así lo requiería, sino más bien
porque no tenía ropa que ponerse.

Rápidamente le dio a Nicolás la maleta con la ropa y un sobre con las señas\footnote{die Anschrift} del antro\footnote{der Abgrund, die berüchtige Bar, die Bude}
donde posiblemente encontraría a Gertrudis y regresó a
hacerse cargo de sus labores.

De pronto escuchó a Pedro preparar la carretela. Le extrañó que lo
hiciera a tan temprana hora, pero al ver la luz del sol se dio cuenta de
que ya era tardísimo y que empacarle a Gertrudis, junto con su ropa,
parte de su pasado le había tomado más tiempo del que se había
imaginado. No le fue fácil meter en la maleta el día en que hicieron su
primera comunión las tres juntas. La vela, el libro y la foto afuera de
la iglesia cupieron muy bien, pero no así el sabor de los tamales y del
atole que Nacha les había preparado y que habían comido después en
compañía de sus amigos y~\sidenote{81}familiares. Cupieron los huesitos de
chabacano de colores, pero no así las risas cuando jugaban con ellos en
el patio de la escuela, ni la maestra Jovita, ni el columpio, ni el olor
de su recámara, ni el del chocolate recién batido. Lo bueno es que
tampoco cupieron las palizas y los regaños de Mamá Elena, pues Tita
cerró muy fuerte la maleta antes de que se fueran a colar.

Salió el patio justo en el momento en que Pedro gritaba buscándola con
desesperación. Tenía que ir a Eagle Pass por el doctor Brown, que era el
médico de la familia, y no la encontraba por ningún lado. Rosaura había
empezado con los dolores del parto.

Pedro le encargó que por favor le la atendiera mientras él volvía.

Tita era la única que es podía hacerlo. En casa no quedaba nadie: Mamá
Elena y Chencha ya se habían ido al mercado, con el propósito de abastecer\footnote{jmdn. mit etw. versorgen}
la despensa\footnote{die Speisekammer, die Vorratskammer} pues
esperaban el nacimiento de un momento a otro y no querían que faltara en
casa ningún artículo que fuera indispensable en estos casos. No habían podido
hacerlo antes pues la llegada de los federales y su peligrosa estancia
en el pueblo se los había impedido.
No supieron al salir que el arribo
del niño ocurriría más pronto de los que pensaban, pues en cuanto se
fueron Rosaura había empezado con el trabajo de parto.

A Tita entonces no le quedó otra que ir al lado de su hermana para
acompañarla, con la esperanza de que fuera por poco tiempo.

No~\sidenote{82}tenía ningún interés en conocer al niño o niña o lo que fuera.

Pero lo que nunca se esperó es que a Pedro lo capturaran los federales injustamente\footnote{fälschlich, zu Unrecht} impidiéndole
llegar por el doctor y que Mamá Elena y Chencha no pudieran regresar
a causa de una balacera\footnote{die Schießerei}
que se entabló\footnote{etw. beginnen, etw. einleiten, etw. unternehmen}
en el pueblo y que las obligó a refugiarse en casa de los Lobo, y que de
esta manera la única presente en el nacimiento de su sobrina fuera ella,
¡precisamente ella! En las horas que pasó al lado de su hermana aprendió
más que en todos sus años de estudio en la escuela del pueblo. Renegó\footnote{etw. ableugnen, etw. verfluchen, etw. abschwören}
como nunca sus maestros y de su mamá por no haberle dicho en ninguna
ocasión lo que se tenía que hacer en un parto.
De qué le servía en ese momento saber los nombres de los planetas y el
Manual de Carreño de pe a pa si su hermana estaba a punto de morir y
ella no podía ayudarla.
Rosaura había engordado treinta kilogramos durante el embarazo, lo cual
dificultaba aún más su trabajo de parto como primeriza\footnote{anfängerisch}.
Dejando de lado la excesiva gordura\footnote{die Korpulenz, die Fettheit}
de su hermana, Tita notó que a Rosaura se le estabaetw. aufblähen, etw. aufblasen, etw. aufpumpen|Verb\footnote{hinchar algo} descomunalmente\footnote{ungeheuer, außerordentlich, ungemein} el
cuerpo. Primero fueron los pies y después la cara y manos. Tita le limpiaba
el sudor de la frente y trataba de animarla, pero Rosaura parecía no
escucharla.

Tita había visto en hacer algunos animales, pero esas experiencias de
nada le servían en estos momentos. En aquellas ocasiones solo había
estado de espectadora. Los animales sabían muy bien lo que tenían que
hacer, en cambio ella no sabía nada de nada. Tenía preparadas sábanas,
agua caliente y unas tijeras esterilizadas. Sabía que tenía que cortar~\sidenote{83}el cordón umbilical, pero no sabía cómo ni cuándo ni a qué altura.
Sabía que había que darle una serie de atenciones a la criatura en
cuanto arriba a este mundo, pero no sabía cuáles. Lo único que sabía es
que primero tenía que nacer, ¡y no tenía para cuándo! Tita se asomaba
entre las piernas de su hermana con frecuencia y nada. Solo un túnel
obscuro, silencioso, profundo. Tita, arrodillada frente a Rosaura, con
gran desesperación pidió a Nacha que la iluminara en estos momentos.

¡Si era posible que le dictara algunas recetas de cocina, también era
posible que le ayudara en este difícil trance\footnote{die Trance, schwierige Situation, kritische Lage}!
Alguien tenía que asistir a Rosaura desde el más allá, porque los del más
acá no tenían manera.

No supo por cuánto tiempo rezó de hinojos\footnote{der Dill, der Fenchel, sich hinknien}, pero cuando por fin despegó los párpados\footnote{das Augenlid},
el obscuro túnel de un momento a otro se transformó por completo en un río
rojo, en un volcán impetuoso, en un desgarramiento\footnote{der Einriss, das Einreissen} de
papel. La carne de su hermana se habría para dar paso a la vida. Tita no
olvidaría nunca ese sonido ni la imagen de la cabeza de su sobrino
saliendo triunfante de su lucha por vivir. No era una cabeza bella, más
bien tenía forma de un piloncillo\footnote{der Zuckerhut},
debido a la presión a que sus huesos estuvieron sometidos\footnote{unterworfen, zensiert}
por tantas horas. Pero a Tita le pareció la más hermosa de todas las que
había visto en su vida.

El llanto del niño invadió todos los espacios vacíos dentro del corazón
de Tita. Supo entonces que amaba nuevamente: a la vida, a ese niño, a
Pedro, inclusive a su hermana, odiada por tanto tiempo. Tomó al niño
entre sus manos, se~\sidenote{84}lo llevó a Rosaura, y juntas lloraron un rato,
abrazadas a él.

Después, siguiendo las instrucciones que Nacha le daba al oído, supo
perfectamente todos los pasos que tenía que seguir: cortar el cordón
umbilical en el lugar y momento precisos, limpiar el cuerpo de niño con
aceite de almendras dulces, fajarle\footnote{jmdn. umwickeln}
el ombligo\footnote{der Bauchnabel} y vestirlo. Sin ningún
problema supo cómo ponerle primero la camiseta y la camisa, luego el fajero\footnote{die Nabelbinde} en el ombligo, luego el pañal\footnote{die Windel}
de manta\footnote{die Schlafdecke, die Stalldecke} de cielo, luego
y el de ojo de pájaro, luego la franela\footnote{das Flanelltuch}
para cubrirle las piernas, luego la chambrita\footnote{die Hausjacke},
luego los calcetines y los zapatos y, por último, utilizando
una cobija\footnote{die Bettwäsche} de felpa\footnote{der Plüsch} le cruzó las manos sobre el
pecho para que no fuera a rasguñar\footnote{etw. anritzen, etw. aufkratzen}
la cara. Cuando por la noche llegaron Mamá Elena y Chencha
acompañaras de los Lobo, se admiraron del profesional trabajo que Tita
realizó. Envuelto como taco, el niño dormía tranquilamente.

Pedro llegó con el doctor Brown hasta el día siguiente, después de que lo
dejaron en libertad. Su retorno tranquilizó a todos.

Temían por su vida. Ahora solo les quedaba la preocupación por la salud
de Rosaura, que aún estaba muy delicada e hinchada\footnote{aufgedunsen, geschwollen}.
El doctor Brown la examinó exhaustivamente. Fue entonces que supieron
lo peligroso que había estado el parto. Según el doctor, Rosaura sufrió
un ataque de eclampsia que la pudo haber matado. Se mostró muy sorprendido
de que Tita la hubiera asistido con tanto aplomo\footnote{der Ernst, die Sicheheit, sicheres Auftreten, die Souveränität}
y decisión en condiciones tan poco favorables. Bueno, quién sabe qué le
llamó más la atención, si el que Tita la hubiera atendido sola y sin
tener ninguna experiencia o el descubrir de pronto que Tita, la niña
dientona~\sidenote{85}que él recordaba, se había transformado en una bellísima
mujer sin que él lo hubiera notado.

Desde la muerte de su esposa, cinco años atrás, nunca había vuelto a
sentirse atraído hacía ninguna mujer. El dolor de haber perdido a su cónyuge\footnote{der Gatte, die Gattin}, prácticamente
de recién casados, lo había dejado insensible para el amor todos estos años.
Qué extraña sensación le producía el observar a Tita. Un hormigueo le recorría
todo el cuerpo, despertando y activando sus dormidos sentidos. La observaba
como si fuera la primera vez que lo hiciera. Qué agradable le parecían ahora
sus dientes, habían tomado su verdadera proporción dentro de la armonía perfecta
de las facciones\footnote{die Gesichtszüge, die Fraktion, die Parteigruppe}
finas y delicadas de su rostro.

La voz de Mamá Elena interrumpió sus pensamientos.
\\- Doctor, ¿no sería molesto para usted venir dos veces al día, hasta que %
mi hija salga del peligro? %
\\- ¡Claro que no! En primera es mi obligación y en segunda es un placer %
frecuentar su agradable casa.\\

Fue verdaderamente una fortuna que Mamá Elena estuviera muy preocupada
por la salud de Rosaura y no detectara el brillo de aspiración\footnote{das Saugen, das Streben} que John
tenía en la mirada mientras observaba a Tita, pues de haberlo hecho no
le hubiera abierto tan confiadamente las puertas de su hogar.

Por ahora el doctor no le representaba ningún problema a Mamá Elena; lo
único que la tenía muy preocupada era que Rosaura no tenía leche.

En el pueblo, afortunadamente, encontraron a una nodriza que se encargó
de amamantar\footnote{jmdn. stillen, jmdn. saugen} al
niño. Era pariente de Nacha, acababa de tener su octavo
hijo y aceptó con agrado el honor de alimentar al nieto de Mamá Elena.
Durante un mes lo hizo de maravilla, hasta que una mañana, cuando se
dirigía al pueblo a visitar a su familia, fue alcanzada por una bala\footnote{der Ballen, die Gewehrkugel} perdida que se escapó
de una balacera entre rebeldes y federales y la hirió\footnote{jmdn./etw. verletzen, jmdn./etw. verwunden}
de muerte. Una de sus parientes y llegó a dar la noticia al
rancho, justamente cuando Tita y Chencha estaban mezclando en una olla
de barro grande todos los ingredientes del mole.

Este es el último paso y se realiza cuando ya se tienen todos los
ingredientes molidos tal y como se indicó. Se mezclan en una olla, se le
añaden las piezas del guajolote, las tablillas de chocolate y azúcar al
gusto. En cuanto espesa, se retira del fuego.

Tita terminé sola de preparar el mole, pues Chencha, en cuanto supo la
noticia, se fue inmediatamente al pueblo a tratar de conseguir otra
nodriza para el niño. Regresó hasta la noche y sin haberlo logrado. El
bebé lloraba exasperado\footnote{wütend werden, aufbrausen, jmdn. in Wut versetzen}.
Trataron de darle leche de vaca y la rechazó.
Tita trató entonces de darle té, tal y como Nacha lo había hecho con
ella pero fue inútil: el niño igualmente lo rechazó. Se le ocurrió
ponerse el rebozo\footnote{mexikanisches Kleidungsstück für Frauen, eine Art Tuch}
que Lupita la nodriza había dejado olvidado, pensando
que el niño se tranquilizaría al percibir el olor familiar que este
despedía, pero por el contrario, el niño lloró con más fuerza, pues el
olor le indicaba que ya pronto recibiría su alimento y no comprendía el
motivo de su retraso. Buscaba desesperado su leche entre los senos de
Tita.

Si hay algo en la vida que Tita no resistía~\sidenote{87}era que una persona
hambrienta le pidiera comida y que ella no pudiera dársela. Le provocaba
mucha angustia. Y sin poderse contener por más tiempo, Tita se abrió la
blusa y le ofreció al niño su pecho. Sabía que estaba completamente
seco, pero al menos le serviría de chupón y lo mantendría ocupado
mientras ella decidía qué hacer para calmarle el hambre.

El niño se pescó\footnote{etw. angeln, etw. fischen, etw. fangen}
del pezón\footnote{die Zitze, die Brustwarze} con desesperación
y succionó y succionó, con fuerza tan descomunal que logró sacarle leche
a Tita. Cuando ella vio que el niño recuperaba poco a poco la tranquilidad
en su rostro y les escuchó deglutir sospechó que algo extraño estaba pasando.
¿Sería posible que el niño se estuviera alimentando de ella? Para comprobarlo,
separó al niño de su pecho y vio cómo la brotaba un chisguete\footnote{der Schluck}
de leche.
Tita no alcanzaba a comprender lo que sucedía. No era posible una mujer
soltera tuviera leche, se trataba de un hecho sobrenatural y sin
explicación en esos tiempos. En cuanto el niño sintió que lo separaban
de su alimento empezó a llorar nuevamente. Tita inmediato lo dejó que se
pescara a ella, hasta que sació por completo su hambre y se quedó
plácidamente dormido, como un bendito. Estaba tan absorta\footnote{versonnen, andächtig} en la contemplación del
niño que no sintió cuando Pedro entró a la cocina.
Tita era en ese momento la misma Ceres personificada, la diosa de la
alimentación en pleno.

Pedro no se sorprendió en lo más mínimo ni necesitó recibir una
explicación. Embelesado\footnote{jmdn. bezaubern, jmdn. entzücken}
y sonriente, se acercó a ellos, se inclinó y le dio un beso a Tita en la frente.
Tita le quitó~\sidenote{88}al niño el pecho, ya estaba satisfecho. Entonces los ojos
de Pedro contemplaron realmente lo que antes habían visto a través de la
ropa: los senos de Tita.

Tita intentó cubrirse con la blusa, Pedro la ayudó en silencio y con
gran ternura. Al hacerlo, una serie de sentimientos encontrados se
apoderaron de ellos: amor, deseo, lujuria\footnote{die Lüsternheit, die Wollust},
vergüenza \ndots, temor de verse descubiertos. El sonido de los pasos
de Mamá Elena sobre la duela\footnote{die Daube} de madera les
advirtieron oportunamente del peligro. Tita alcanzó a ajustarse correctamente
la blusa y Pedro a tomar distancia de ellas antes de que Mamá Elena entrara
a la cocina. De tal manera que cuando abrió la puerta no pudo encontrar,
dentro de lo que las normas sociales permiten, nada de qué preocuparse.
Pedro y Tita aparentaban gran serenidad.

Sin embargo, algo olío en el ambiente que la hizo agudizar todos sus
sentidos y tratar de descubrir que era lo que la inquietaba.
\\- Tita, ¿qué pasa con ese niño? ¿Lograste hacerlo comer? %
\\- Sí, mami, tomó su té y se durmió. %
\\- ¡Bendito sea Dios! Entonces, Pedro, ¿qué esperas para llevar al niño %
con tu mujer? Los niños no deben estar lejos de su madre.\\

Pedro salió con el niño en brazos, Mamá Elena no dejaba de observar
detenidamente a Tita, había en sus ojos un destello\footnote{der Blitz, das Glitzern}
de turbación\footnote{die Bestürzung, die Verwirrung} que no
le gustaba para nada.
\\- ¿Ya está listo el champurrado\footnote{Mixgetränk mit Maisstärke und Schokolade} %
para tu hermana? %
\\- Ya, mami. %
\\- Dámelo para que se lo lleve, necesitas tomarlo día y noche, para que %
le~\sidenote{89}baje la leche.\\

Pero por más champurrado que tomó, nunca le bajó la leche. En cambio
Tita tuvo desde ese día leche suficiente como para alimentar no solo a
Roberto sino a otros dos niños más, si así lo hubiera deseado. Como
Rosaura estuvo dedicada algunos días, a nadie le extrañó que Tita se
encargara de darle de comer a su sobrino; lo que nunca descubrieron fue
la manera en que lo hacía, pues Tita, con la ayuda de Pedro, puso mucho
cuidado en que nadie la viera.
El niño, por tanto, en lugar de ser un motivo de separación entre ambos,
terminó por unirlos más. Tal parecía que la madre del niño era Tita y
no Rosaura. Ella así lo sentía y así lo demostraba. El día del bautizo,
¡con qué orgullo cargaba a su sobrino y lo mostraba a todos los
invitados! Rosaura no pudo estar presente más que en la iglesia pues aún
se sentía mal. Tita entonces tomó su lugar en el banquete.

El doctor John Brown miraba a Tita embelesado. No podía quitarle los
ojos de encima. John había asistido al bautizo solo para ver si podía
conversar con ella a solas. A pesar de que se veían a diario durante las
visitas médicas que John le hacía a Rosaura, no habían tenido la
oportunidad de platicar libremente y sin ninguna otra persona presente.
Aprovechando que Tita caminaba cerca de la mesa donde él se encontraba,
se levantó y se le acercó con el pretexto de ver al niño.
\\- ¡Qué bien se ve este niño, al lado de una tía tan bella! %
\\- Gracias, doctor. %
\\- Eso que no es su~\sidenote{90}propio hijo, ya me imagino lo bonita que se va a %
ver cuando el niño que cargue sea el suyo.\\

Una nube de tristeza cruzó por el semblante\footnote{die Miene, der Gesichtsausdruck}
de Tita. John la detectó y dijo:
\\- Perdón, parece que dije a incorrecto. %
\\- No, no es eso. Lo que pasa es que yo no me puedo casar, ni tener %
hijos, porque tengo que cuidar a mi mamá hasta que muera. %
\\- ¡Pero cómo! Eso es una tontería. %
\\- Pero es así. Ahora le ruego\footnote{die Bitte, etw. erbitten} %
me disculpe, voy a atender a mis invitados.\\

Tita se alejó rápidamente, dejando a John completamente desconcertado.
Ella también lo estaba, pero se recuperó de inmediato al sentir en sus
brazos a Roberto.

Qué le importa para su destino mientras pudiera tener cerca de ese
niño, que era más suyo que de nadie. Realmente ella ejercía el puesto de
madre sin el título oficial. Pedro y Roberto le pertenecían y ella no
necesitaba nada más en la vida.

Tita estaba tan feliz que no se dio cuenta de que su madre, lo mismo
que John, aunque por otra razón, no la perdía de vista un solo instante.
Estaba convencida de que algo se traían entre manos Tita y Pedro.
Tratando de descubrirlo, ni siquiera comió y estaba tan concentrada en
su labor de vigilancia que le pasó desapercibido el éxito o de la
fiesta. Todos estuvieron de acuerdo en que gran parte del mismo se debía
a Tita, ¡el mole que había preparado estaba delicioso! Ella no paraba de
recibir felicitaciones por sus méritos como cocinera y todos querían
saber~\sidenote{91}cuál era su secreto.

Fue verdaderamente lamentable que en el momento en que Tita respondía a
esta pregunta diciendo que su secreto era que había preparado el mole
con mucho amor, Pedro estuviera cerca y los dos se miraran por fracción
de segundo con complicidad recordando el momento en que Tita molía en el
metate, pues la vista de águila de Mamá Elena, a veinte metros de distancia,
detectó el destello y le molestó profundamente.

Entre todos los invitados ella era realmente la única molesta, pues
curiosamente, después de comer el mole, todos habían entrado en un
estado de euforia que los hizo tener reacciones de alegría poco comunes.
Reían y alborotaban como nunca lo habían hecho y pasaría a bastante
tiempo antes de que lo volvieran a hacer. La lucha revolucionaria
amenazaba con acarrear\footnote{etw. transportieren, etw. einfahren}
hambre y muerte por doquier\footnote{überall, weit und breit}.
Pero en esos momentos parecía que todos trataban de olvidar que en el
pueblo había muchos balazos.

La única que no perdió la compostura\footnote{der Anstand}
fue Mamá Elena, que estaba muy ocupada en buscar una solución a su resquemor\footnote{der Kummer}, y aprovechando un momento
en que Tita estaba lo suficientemente cerca como para no perder una sola
de las palabras que ella pronunciaria, le comentó al Padre Ignacio en
voz alta:
\\- Por cómo se están presentando las cosas, padre, me preocupa que un día %
mi hija Rosaura necesite un médico y no lo podamos traer, como el día en %
que dio a luz. Creo que lo más conveniente sería que en cuanto tenga más %
fuerzas~\sidenote{92}se vaya junto con su esposo y su hijito a vivir a San %
Antonio, Texas, con mi primo. Ahí tendrá mejor atención médica. %
\\- Yo no opino lo mismo, doña Elena, precisamente por cómo está la %
situación política, usted necesita de un hombre en casa que la defienda. %
\\- Nunca lo he necesitado para nada, sola he podido con el rancho y con %
mis hijas. Los hombres no son tan importantes para vivir, padre - recalcó\footnote{etw. unterstreichen, etw. bekräftigen, etw. betonen} -. Ni %
la reproducción es tan peligrosa como la pintan, ¡peor es %
el chile y el agua lejos! %
\\- ¡No, pues eso sí! - respondió riéndose -. ¡Ah, qué doña Elena! Siempre %
tan ocurrentes\footnote{einfallsreich}. Y, dígame, ¿ya pensó donde %
trabajaría Pedro en San Antonio? %
\\- Puede entrar a trabajar como contador\footnote{der Buchhalter, der Gebührenzähler} %
en la compañía de mi primo, no tendrá problema, pues habla el inglés a %
la perfección.\\

Las palabras que Tita escuchó resonaron como cañonazos\footnote{der Kanonendonner, der Kanonenschlag} dentro
de su cerebro. No podía permitir que eso pasara. No era posible que ahora le
quitaran al niño. Tenía que impedirlo a como diera lugar. Por lo pronto,
Mamá Elena logró arruinarle la fiesta. La primera fiesta que gozaba en
su vida.


\clearpage

\section*{ Capítulo 5 }
\subsection*{ Mayo }
\subsection*{ Chorizo norteño }
\textbf{Ingredientes}

\begin{itemize}
    \item 8 kg de lomo de puerco
    \item 2 kg de retazo\footnote{der Rest, das Stück} o cabeza de lomo
    \item un kilo de chile ancho
    \item 60 gramos de comino
    \item 60 gramos de orégano
    \item 30 gramos de pimienta
    \item 60 gramos de clavo
    \item 2 tazas de ajos
    \item 2 litros de vinagre de manzana
    \item ¼ kg de sal
\end{itemize}

\subsection*{ Manera de hacerse: }
El vinagre~\sidenote{97}se pone en la lumbre\footnote{die Flamme, die Glut}
y se le incorporan los chiles, a
los que previamente se les han quitado las semillas. En cuanto suelta el
hervor, se retira del fuego y se pone a la olla una tapadera encima,
para que los chiles se ablanden.

Chencha puso la tapa y corrió a la huerta a ayudar a Tita en su
búsqueda de lombrices. De un momento a otro llegaría a la cocina Mamá
Elena a supervisar la elaboración de chorizo y la preparación del agua
para su baño y estaban bastante atrasadas en ambas cosas. El motivo era
que Tita, desde que Pedro, Rosaura y el niño se habían ido a vivir a San
Antonio, Texas, había perdido todo interés en la vida, exceptuando el
que le despertaba un indefenso pinchón al que alimentaba con lombrices.
De ahí en fuera, la casa podía caerse, que a ella no le importaba.

Chencha no quería ni imaginar lo que pasaría si Mamá Elena se enteraba
que Tita no quería participar en la elaboración de chorizo.

Habían decidido prepararlo por ser uno de los mejores recursos para
utilizar la carne de cerdo de una manera económico y que les aseguraba
un buen alimento por mucho tiempo, sin peligro de que se~\sidenote{98}descompusiera.
También habían dispuesto una gran cantidad de cecina\footnote{das Dörrfleisch, das Rauchfleisch},
jamón, tocino y manteca. Tenían que sacarle el mejor provecho posible
a este cerdo, uno de los pocos animales sobrevivientes de la visita
que miembros del ejército revolucionario les habían hecho unos días
antes.

El día que llegaron los rebeldes, solo estaban en el rancho Mamá Elena,
Tita, Chencha y dos peones\footnote{der Hilfsarbeiter}: Rosalío y
Guadalupe . Nicolás, el capataz, aún no regresaba con el ganado que por
imperiosa necesidad había ido a comprar, pues ante la escasez de alimentos
había tenido que ir matando a los animales con que contaban y era preciso
reponerlos. Se había llevado con él a dos de los trabajadoras de más
confianza para que lo ayudaran.
Había dejado a su hijo Felipe al cuidado del rancho, pero Mamá Elena lo
había relevado del cargo, tomando ella el mando en su lugar, para que
Felipe pudiera irse a San Antonio, Texas, en busca de noticias sobre
Pedro y su familia. Temían que algo malo les hubiera pasado, ante su
falta de comunicación desde su partida.

Rosalío llegó a galope a informar que una tropa se acercaba al rancho.
Inmediatamente Mamá Elena tomó su escopeta y mientras la limpiaba pensó
en esconder de la voracidad\footnote{die Gefräßigkeit} y el
deseo de estos hombres los objetos más valiosos que poseía. Las referencias
que le habían dado de los revolucionarios no era nada buenas, claro que
tampoco era nada confiables\footnote{zuverlässig} pues provenían
del Padre Ignacio y del presidente municipal de Piedras Negras. Por ellos
tenía conocimiento de cómo entraban a la casa, cómo arrasaban\footnote{etw. einebnen, etw. abreißen, etw. zerstören, etw. ausradieren}
con todo y como violaban a las muchachas que encontraban en su~\sidenote{99}camino.
Así pues, ordenó que Tita, Chencha y el cochino permanecieran escondidos
en el sótano.

Cuando los revolucionarios llegaron, encontraron a Mamá Elena en la
entrada de la casa. Bajo las enaguas\footnote{der Unterrock} escondía
su escopeta, su lado estaban Rosalío y Guadalupe. Su mirada se encontró con
la del capitán que venía al mando y este supo inmediatamente, por la dureza
de esa mirada, que estaba ante una mujer de cuidado.
\\- Buenas tardes, señora, ¿es usted la dueña de este rancho? %
\\- Así es. ¿Qué es lo que quieren? %
\\- Venimos a pedirle, por las buenas, su cooperación para la causa. %
\\- Y yo, por las buenas, les digo que se lleven lo que quieran de %
provisiones que encuentran en el granero y los corrales. Pero eso sí, %
las que tengo dentro de mi casa no las tocan, ¿entendido? Esas son para %
mí causa particular.\\

El capitán, bromeando, se le cuadró\footnote{strammstehen}
y respondió:
\\- Entendido, mi general.\\

A todos los soldados les cayó en gracia el chiste, y lo festejaron,
pero el capitán se dio cuenta de que con Mamá Elena no valían las chanzas\footnote{der Scherz, der Witz}, ella hablaba en serio,
muy en serio.

Tratando de no amedrentarse\footnote{sich erschrecken, sich ängstigen}
por la dominante y severa\footnote{streng, schwerwiegend} mirada que
recibía ella, ordenó que revisaran el rancho. Lo que encontraron no fue
gran cosa, un poco de maíz para desgranar\footnote{etw. auskörnen, etw. aushülsen}
y ocho gallinas. Uno de los sargentos, muy molesto, se acercó al capitán
y le dijo:
\\- Este vieja~\sidenote{100}ha de tener todo escondido dentro de la casa, ¡déjeme %
entrar a supervisar!\\

Mamá Elena, poniendo el dedo en el gatillo\footnote{der Abzug, der Hahn, der Drücker},
respondió:
\\- ¡Yo no estoy bromeando y ya dije que a mi casa no entra nadie! \\
\rule{1em}{0pt}El sargento, riéndose y columpiando unas gallinas que llevaba en la mano,
trató de caminar hacía la entrada. Mamá Elena levantó la escopeta, se recargó\footnote{etw. (neu) laden, etw. aufladen, etw. nachladen}
en la pared para no caer al piso por el impulso que iba a
recibir, y les disparó\footnote{der Schuss, abdrücken, schießen}
a las gallinas. Por todos lados se esparcieron\footnote{sich verbreiten, sich verstreuen}
pedazos de carne y olor a plumas quemadas.

Rosalío y Guadalupe sacaron sus pistolas templando\footnote{etw. abschwächen, etw. mildern, etw. dämpfen}
y plenamente convencidos de que ese era su último día en la tierra.
El soldado que estaba junto al capitán intentó dispararle a Mamá Elena,
pero que el capitán con un gesto se lo impidió. Todos esperaban una
orden suya para atacar.

- Tengo muy buen tino\footnote{die Geschicklichkeit, die Treffsicherheit} %
y muy mal carácter, capitán. El próximo tiro es para usted y le aseguro que %
puedo dispararle antes de que me maten, así es que mejor nos vamos %
respetando, porque si nos morimos, yo no le voy a hacer falta a nadie, %
pero de seguro la nación sí sentiría mucho su pérdida, ¿o no es así?\\

Realmente era difícil sostener la mirada de Mamá Elena, hasta para un
capitán. Tenía algo que atemorizaba. El efecto que provocaba en quienes
la recibían era de un temor indescriptible: se sentían enjuiciados\footnote{jmdn./etw. beurteilen, jmdn./etw. bewerten} y sentenciados\footnote{etw. entscheiden, etw. zu Protokoll geben, jmdn. zu etw. verurteilen}
por faltas cometidas. Caía uno preso de un miedo pueril\footnote{jungenhaft, kindisch, naiv}
a la autoridad materna.
\\- Sí, tiene~\sidenote{101}razón pero no se preocupe, nadie va a matarla, ni a %
faltarle al respeto, ¡faltaba más! Una mujer así de valiente siempre %
tendrá mi admiración. - y dirigiéndose a sus soldados dijo -: nadie va a %
entrar a esta casa, vean qué más fue encontrar aquí y vámonos.\\

Lo que descubrieron fue el gran palomar que formaba todo el techo de
dos aguas de la enorme casa. Para llegar a él se tenía que trepar una
escalera de siete metros de altura. Subieron tres rebeldes hice quedaron pasmados\footnote{sprachlos sein, verblüfft sein} un buen rato
antes de poder moverse. Imponían el tamaño, la obscuridad y el canturreo\footnote{(halblaut vor sich hin) trällern, vor sich hin singen}
de las palomas ahí reunidas que entraban y salían por pequeñas ventanas
laterales. Cerraron la puerta y las ventanas para que ninguna pudiera escapar
y se dedicaron a atrapar pichones y palomas.
 Juntaron\footnote{etw. verbinden, etw. zusammentragen, etw. vereinigen}
tal cantidad que pudieron alimentar a todo el batallón por una
semana. Antes de retirarse, el capitán recorrió a caballo el patio
trasero, inhaló profundamente el indeleble olor a rosas que aún
permanecía en ese lugar. Cerró los ojos y así permaneció un buen rato.
Regresando al lado de Mamá Elena le preguntó:
\\- Tengo entendido que tiene tres hijas, ¿dónde están? %
\\- La mayor y la menor viven en Estados Unidos, la otra murió.\\

La noticia pareció conmover\footnote{jmdn. erschüttern, jmdn. rühren}
al capitán. Con voz apenas perceptible respondió:
\\- Es una lástima, una verdadera lástima. %
Se despidió de Mamá Elena con una reverencia. Se fueron tranquilamente, %
tal y como vinieron y Mamá Elena quedó muy desconcertada~\sidenote{102}antes la %
actitud que habían tenido para con ella; no correspondía a la de los matones\footnote{der Schläger, der Raufbold} desalmados\footnote{herzlos} %
que esperaba. %
Desde ese día prefirió no opinar sobre los revolucionarios. De lo que nunca %
se enteró es de que es capitán y era el mismo Juan Alejandres que meses antes %
se había llevado a su hija Gertrudis.\\

Estaban a mano, pues el capitán también ignoró que en la parte trasera
de la casa Mamá Elena tenía, enterrada en ceniza, una gran cantidad de
gallinas. Habían logrado matar a 20 antes de que ellos llegaran. Las
gallinas se rellenan con granos de trigo\footnote{der Weizen}
o avena\footnote{der Hafer } y con todo y plumas se
meten dentro de una olla de barro barnizado\footnote{etw. beizen, etw. lackieren, etw. lasieren, etw. glasieren}.
Con un lienzo se tapa bien la olla y de esta manera se puede conservar
la carne en buen estado por más de una semana.

Esta era una práctica común en el rancho desde tiempos remotos\footnote{fern, abgelegen, weit entfernt},
cuando tenían que conservar los animales después de una cacería.

Al salir de su escondite, lo primero que Tita extrañó fue el canturreo
constante de las palomas, el cual, desde que nació, formaba parte de su
cotidianidad. Este súbito silencio hizo que sintiera de golpe la
soledad. Fue en ese momento cuando más sintió la partida de Pedro,
Rosaura y Roberto del rancho. Subió rápidamente los peldaños de la
enorme escalera que terminaba en el palomar y lo único que encontró fue
la alfombra de plumas y la suciedad característica del lugar.

El viento se colaba por la puerta abierta y levantaba algunas plumas
que caía sobre una alfombra de silencio. De pronto~\sidenote{103}escuchó un
leve sonido: un pequeño pichón recién nacido se había salvado de la
masacre. Tita lo tomó y se dispuso a bajar, pero antes se detuvo a mirar
por un momento la polvareda\footnote{die Staubwolke}. Se
preguntaba extrañada el porqué no le habían hecho ningún daño a su madre.
Mientras estaba en su escondite rezaba por que nada malo le pasara a Mamá
Elena, pero inconscientemente tenía la esperanza de que al salir la
encontraría muerta.

Avergonzada de tales pensamientos metió al pichón entre sus pechos para
tener las manos libres y poder agarrarse bien de la peligrosa escalera.
Luego bajó del palomar. Desde ese día su mayor preocupación y era la de
alimentar al escuálido\footnote{verwahrlost, ausgemergelt, abgemagert}
pichón. Solo de esta manera la vida tenía cierto sentido. No se comparaba
con la plenitud que proporciona al amamantar a un ser humano, pero de alguna
manera se le parecía.

Sus pechos se habían secado de un día para otro, por la pena que le
causó la separación de su sobrino. Mientras buscaba lombrices, no podía
dejar de pensar en quién y cómo estaría alimentando a Roberto. Este
pensamiento la atormentaba día y noche. En todo el mes no había podido
conciliar el sueño ni un instante. Su único logro\footnote{der Erfolg, die Errungenschaft, die Leistung}
durante ese periodo había sido el quintuplicar\footnote{verfünffachen}
el tamaño de su enorme colcha. Chencha llegó a sacarla de sus pensamientos
de conmiseración\footnote{das Mitleid} y se la llevó a
empujones a la cocina. La sentó frente al metate y la puso a moler las especias
junto con los chiles. Para que se facilite esta operación es bueno poner
de vez en cuando unos chorritos de vinagre mientras se muele. Por
último, se mezcla la carne muy picada o molido con los chiles y las~\sidenote{104}especias y se deja reposar largo rato, de preferencia todo una
noche.

No acababan de empezar a moler cuando Mamá Elena entró a la cocina,
preguntando por qué no estaba llena la tina\footnote{die Wanne, der Zuber}
para su baño. No le gustaba bañarse demasiado tarde, pues el cabello
no se le alcanzaba a secar adecuadamente.

Preparar el baño para Mamá Elena ir a lo mismo que preparar una
ceremonia. El agua se tenía que poner a hervir con flores de espliego\footnote{der Lavendel},
el aroma preferido de Mamá Elena. Después se pasaba la \glqq{}decocción\grqq{}\footnote{die Abkochung, der Absud}
por un paño limpio y se le añadían unas gotas de aguardiente\footnote{der Branntwein}.
Por último había que llevar, una tras otra, cubetas con esta agua caliente
hasta el cuarto obscuro. Un pequeño cuarto que estaba al final de la
casa, junto a la cocina. Este cuarto, como su nombre lo indica, no recibía rayo de
luz alguno pues carecía de ventanas. Solo tenía una angosta\footnote{eng, etw. verengen}
puerta. Dentro, a mitad del cuarto, se encontraba un cantina donde se depositaba
el agua. Junto a ella, en una vasija de peltre\footnote{das Bleizinn}
se ponía agua shi-shi para el lavabo del pelo de Mamá Elena.

Solo Tita, cuya misión era la de atenderla hasta su muerte, era la única
que podía estar presente en el ritual y ver a su madre desnuda. Nadie
más. Por eso se había construido este cuarto a prueba de mirones\footnote{der Gaffer, der Spanner}.
Tita le tenía que lavar a su mamá primero el cuerpo, luego el cabello y por
último la dejaba unos momentos descansado, gozando del agua, mientras
ella planchaba la ropa que se pondría Mamá Elena al salir de la tina.

A una orden de su madre, Tita la ayudaba a secarse y a ponerte lo más
pronto posible la ropa bien caliente para evitar un~\sidenote{105}resfrío.
Después, entreabría\footnote{etw. halb öffnen} un milímetro
la puerta, para que el cuarto se fuera enfriando y el cuerpo de Mamá
Elena no sufriera un cambio brusco de temperatura. Mientras tanto le cepillaba\footnote{etw. bürsten} el pelo, alumbrada únicamente
por el débil rayo de luz que se filtraba por la rendija\footnote{der Lichtspalt, der Spalt}
de la puerta y que creaba un ambiente de sortilegio
al revelar las formas caprichosas del vapor de agua. La cepillaba el
cabello hasta que este quedaba seco por completo, entonces le hacía
una trenza y daban por terminada la liturgia. Tita siempre daba gracias
a Dios de que su mamá solo se bañara una vez por semana, porque si no
su vida sería un verdadero calvario\footnote{die Qual}.

En opinión de Mamá Elena, con el baño pasaba lo mismo que con la comida:
formas que Tita seesforzarse::esforzar a alguien|sich abmühen, sich anstrengen::jmdn. bestärken, jmdn. ermutigen\footnote{esforzaba}
siempre cometía infinidad de errores. O la camisa tenía una arruguita
o no estaba suficiente caliente en agua o la raya de la trenza estaba chueca\footnote{krumm, schief}. En fin,
parecía que la única virtud\footnote{die Fähigkeit, die Tugend} de
Mamá Elena y era la de encontrar defectos. Pero nunca encontró tantos
como ese día. Y es que Tita verdaderamente había descuidado todos los
detalles de la ceremonia. El agua estaba tan caliente que Mamá Elena se
quemó los pies al entrar, había olvidado el shi-shi para el lavado del
pelo, había quemado el fondo y la camiseta, había abierto la puerta
demasiado, en fin, que ahora sí se había ganado a pulso el que Mamá
Elena la reprendieria\footnote{jmdn. tadeln, zurechtweisen}
y la expulsara del cuarto de baño.

Tita caminaba a prisa hacia la cocina, llevando bajo el brazo la ropa
sucia, lamentándose del regañó y de sus garrafales\footnote{großfrüchtig, riesig, ungeheuer}
fallas. No que más le dolía era el trabajo extra que significaba haber
quemado la ropa. Era la segunda vez en su vida~\sidenote{106}que le ocurría este
tipo de desgracia. Ahora iba a tener que humedecer las manchas rojizas
en una solución de clorato de potasa con agua pura y con lejía\footnote{die Bleichlauge, die Lauge}
alcalina suave, restregando\footnote{etw. rubbeln, etw. scheuern}
repetidas veces, hasta lograr que la mancha desapareciera,
aunando ese penoso\footnote{mühsam, leidvoll} trabajo al de lavar la ropa negra
con que se vestía su madre. Para hacerlo tenía que disolver hiel\footnote{die Galle, die Gallseife}
de vaca en una pequeña cantidad de agua hirviendo, sumergir una esponja\footnote{der Schwamm}
suave en esta agua y con ella mojar toda la ropa, enseguida aclarar
con agua limpia los vestidos y sacarlos al aire libre.

Tita fregaba\footnote{etw. wischen, schrubben abspülen} y fregaba
la ropa como tantas veces lo hizo con los pañales de Roberto para
quitarlas las manchas. Lo lograba poniendo a
cocer un porción de orina, en ella sumergía la mancha por un momento,
lavándola después con agua. Así de simple, las manchas se esfumaban.
Pero ahora, por más que sumergía los pañales en la orina, no podía
quitarles es horroroso color negro. De pronto se dio cuenta que no se
trataba de los pañales de Roberto, sino de la ropa de su madre. La había
estado sumergiendo en la bacinica\footnote{der Nachttopf} que
desde la mañana había dejado olvidada sin lavar junto al fregadero.
Apenada, se dispuso a corregir su falla.

Ya instalada en la cocina, Tita se propuso poner más atención en lo que
hacía. Tenía que poner coto\footnote{das Gehege, das Revier, jmdm./etw. Einhalt gebieten}
a los recuerdos que la atormentaban o la furia de Mamá Elena podría
estallar de un momento a otro.

Desde que empezó a preparar el baño de Mamá Elena dejó reposando el
chorizo, por tanto ya había pasado tiempo suficiente como para proceder
a rellenar las tripas\footnote{die Kutteln, der Darm}.

Tienen que ser tripas~\sidenote{107}de res\footnote{das Vieh}, limpias y curadas\footnote{trocken, ausgehärtet, gepökelt}. Para rellenarlas
se utiliza un embudo\footnote{der Fülltrichter, der Trichter}.
Sí están muy bien a distancia de cuatro dedos y se pican con una aguja
para que salga el aire, que es lo que puede perjudicar el chorizo. Es
muy importante comprimirlo muy bien mientras se rellena, para que no quede
ningún espacio.

Por más empeño\footnote{die Beharrung, das Bestreben, jmdn. zu etw. verpflichten, jmdn. zu etw. zwingen, auf etw. beharren, unbedingt etw. tun müssen}
que Tita ponía en evitar que los recuerdos acudieran a
ella y la hicieran cometer más errores, no pudo evitarlos al tener en
las manos un trozo grande de chorizo y rememorar la noche de verano en
que todos salieron a dormir al patio. En la época de canícula\footnote{die Hitze, die Hundstage, die Sommerhitzt} se
colgaban en el espacio grandes hamacas, pues el calor se hacía
insoportable. En una mesa se ponía un tinaja\footnote{der Tonkrug}
con hielo y dentro se colocaba una sandía partida por si alguien a media
noche se levantaba acolorado con deseos de refrescarse comiendo una
rebanada.
Mamá Elena era especialista en partir la sandía: tomando un cuchillo filoso,
encajaba a la punta de la manera que solo penetraba hasta donde
terminaba la parte verde de la cáscara, dejando sin tocar el corazón de
la sandía.

Hacía varios cortes en la cáscara de una perfección matemática tal que
cuando terminaba tomaba entre sus manos la sandía y le daba un solo
golpe sobre una piedra, pero en el lugar exacto y mágicamente la cáscara
de la sandía se abría como pétalo en flor, quedando sobre la mesa el
corazón intacto. Indudablemente\footnote{zweifellos},
tratándose de partir, desmantelar\footnote{etw. demontieren, etw. niederreißen, etw. abbauen}, desmembrar\footnote{etw. aufteilen, etw. spalten, etw. trennen}, desolar\footnote{etw. verheeren, etw. verwüsten}, destetar\footnote{jmdn. abstillen, jmdn. entwöhnen}, desjarretar\footnote{jmdn. entnerven},
desbaratar o desmadrar\footnote{ein Tier von der Mutter trennen} algo,
Mamá Elena era una maestra. Desde que Mamá Elena murió nunca nadie ha
podido volver a realizar esa proeza\footnote{das Bravourstück, die Meisterleistung}
(con la sandía).

Tita escuchó~\sidenote{108}desde su hamaca como alguien se había levantado a
comer un pedazo de sandía. A ella la habían despertado las ganas de ir
al baño. Todo el día había tomado cerveza, no para aminorar
el calor sino para tener más leche para amamantar a su sobrino.

Este dormía apaciblemente junto
a su hermana. Se levantó a tientas\footnote{blindlings, jmdn./etw. betasten},
no podía distinguir nada, era una noche de completa obscuridad. Se fue
caminando hacía el baño, tratando de recordar dónde estaban las hamacas,
No quería tropezar con nadie.

Pedro, sentado en su hamaca, comía su sandía y pensaba en Tita. Su
cercanía le producía un gran agitación. No podía dormir imaginándola ahí
a unos pasos de él. Y de Mamá Elena, por supuesto. Su respiración se
detuvo unos instantes al escuchar el sonido de unos pasos en las tinieblas\footnote{die Finsternis}. Tenía que tratarse de
Tita, la fragancia\footnote{der Wohlgeruch, der Duft}
peculiar que se esparció
por el aire, entre jazmín y olores de la cocina, solo podía
pertenecerle a ella. Por un momento pensó que Tita se había levantado
para buscarlo. El ruido de sus pasos acercándose a él se confundía con
el de su corazón, que latía violentamente. Pero no, los pasos ahora se
alejaban, en dirección al baño. Pedro se levantó como un felino\footnote{die Katze}
y sin hacer ruido la alcanzó.

Tita se sorprendió al sentir que alguien la jalaba y le tapaba la boca,
pero inmediatamente se dio cuenta de a quién pertenecía esa mano y
permitió sin ninguna resistencia que la mano se deslizara primero por su
cuello hasta sus senos y después en un reconocimiento total por todo su
cuerpo.

Mientras recibía un beso en la boca, la mano de Pedro, tomando la suya,
la invitó a recorrerle el~\sidenote{109}cuerpo. Tita tímidamente palpó\footnote{etw. abfühlen, etw. betasten, etw. ertasten}
los duros músculos de los brazos y el pecho de Pedro. Más abajo, un tizón\footnote{der Getreidebrand, die Kohle}
encendido, que palpitaba\footnote{klopfen, schlagen, pochen}
bajo la ropa. Asustada, retiró la mano, no por el descubrimiento, sino por
un grito de Mamá Elena.
\\- Tita, ¿dónde estás? %
\\- Aquí, mami, vine al baño. \\
\rule{1em}{0pt}Temerosa de que su madre sospechara algo, Tita regresó rápidamente y
pasó una noche de tortura aguantando las ganas de orinar acompañada de
otra sensación parecida. Pero de nada sirvió su sacrificio: al día
siguiente Mamá Elena, y por un tiempo parecía haber cambiado de opinión
en cuanto a que Pedro y Rosaura se fueran a vivir a San Antonio, Texas,
aceleró la partida y en tres días más logró que se fueran del rancho.

La entrada de Mamá Elena a la cocina ahuyentó sus recuerdos. Tita dejó
caer el chorizo de entre sus manos. Sospechaba que su madre podía leerle
el pensamiento. Tras ella, entró Chenga llorando desconsoladamente.
\\- ¡No llores, niña! Me choca verte llorar. ¿Qué es lo que te pasa? %
\\- Es q’el Felipe ya’stá aquí y dice ¡que si petatió! %
\\- ¡Pos el niño! %
\\-¿Cuál niño? %
\\- ¡Pos cuál iba’ser! Pos su nieto, todo lo que comía le caía mal !Y pos %
si petatió! \\
\rule{1em}{0pt}Tita sintió en su cabeza~\sidenote{110}un trastero cayéndose. Después del
golpe, el sonido de un vajilla rota
en mil pedazos. Como impelida\footnote{etw. fortschieben, jmdn. zu etw. anregen, jmdn. zu etw. veranlassen}
por un resorte\footnote{die Triebfeder, die Feder} se
levantó.
\\- ¡Siéntate a trabajar! Y no quiero lágrimas. Pobre criatura, espero que %
el Señor lo tenga en su gloria, pero no podemos dejar que la tristeza %
nos gane, hay mucho trabajo que hacer. Primero terminas y luego haces lo %
que quieras, menos llorar, ¿me oíste?\\

Tita sintió que un violenta agitación se posesionaba de su ser:
enfrentó firmemente la mirada de su madre mientras acariciaba el chorizo
y después, en lugar de obedecerla, tomó todos los chorizos que encontró
y los partió en pedazos, gritando enloquecida.
\\- ¡Mire lo que hago con sus órdenes! ¡Ya me cansé! ¡Ya me cansé de %
obedecerla! \\
\rule{1em}{0pt}Mamá Elena se acercó, tomó una cuchara de madera y le cruzó la cara con
ella.
\\- ¡Usted es la culpable de la muerte de Roberto! \\
\rule{1em}{0pt}Le gritó Tita fuera de sí y salió corriendo, secándose la sangre que
le escurría de la nariz; tomó al pichón, la cubeta de lombrices y se
subió al palomar.

Mamá Elena ordenó que quitaran la escalera y que la dejaran pasar toda
la noche ahí. Mamá Elena y Chencha terminaron en silencio de rellenar
los chorizos. Con lo perfeccionista que era Mamá Elena y el cuidado que
siempre ponía para que nos quedara aire dentro de los chorizos, fue
verdaderamente inexplicable para todos que una semana después
encontraran los chorizos invadidos de gusanos en la bodega donde los
había puesto a secar.

A la mañana siguiente~\sidenote{111}mandó que Chencha bajara Tita. Mamá Elena
no podía hacerlo pues solo había una cosa que tenía en la vida y era el
miedo a las alturas. No soportaba ni el pensamiento de tener que subir
por la escalera, que medía siete metros, y abrir hacia fuera la pequeña
puerta, para poder entrar. Por lo tanto le convenía fingir más orgullo
del que tenía y mandar a otra persona para que bajara a Tita, aunque
ganas no le faltaban de subir personalmente y bajarla arrastrándola de
los cabellos.

Chencha la encontró en el pichón en las manos. Tita parecía no darse
cuenta de que estaba muerto. Intentaba darle de comer más lombrices. El
pobre tal vez murió de indigestión\footnote{die Magenverstimmung, die Verdauungsstörung}
porque Tita le dio demasiadas. Tita tenía la mirada perdida y miraba a
Chencha como si fuera la primera vez que la viera en su vida.

Chencha bajó diciendo que Tita estaba como loca y que no quería
abandonar el palomar.
\\- Muy bien, si está como loca va a ir a dar al manicomio\footnote{die Irrenanstalt}. %
¡En esta casa no hay lugar para dementes! \\
\rule{1em}{0pt}Y efectivamente, de inmediato
mandó a Felipe por el doctor Brown para que se llevara a Tita a un manicomio.
El doctor llegó, escuchó la versión de la historia de parte de Mamá Elena y se
dispuso a subir al palomar.

Encontró a Tita desnuda, con la nariz rota y llena de suciedad de
palomas en todo el cuerpo. Algunas plumas se le habían pegado en la piel
y el pelo. En cuanto vio al doctor corrió a un rincón y se puso en posición
fetal.

Nadie supo que tanto le dijo el doctor Brown durante las horas que pasó
con ella, pero al atardecer bajó con Tita ya vestida, la subió a su
carretela y se la llevó.

Chencha,~\sidenote{112}corriendo y llorando a su lado, apenas alcanzó a ponerle
a Tita en los hombros la enorme colcha que había tejido en sus
interminables noches de insomnio. Era tan grande y pesada que no cupo
dentro del carruaje\footnote{die Kutsche}. Tita se aferró\footnote{sich an jmdn./etw. klammern}
a ella con tal fuerza que no hubo más remedio que llevarla arrastrando
como un enorme y caleidoscópica cola de novia que alcanzaba a cubrir
un kilómetro completo. Debido a que Tita utilizaba en su colcha cuánto
están caía en sus manos, sin importarle el color, las colchas mostraba
una amalgama de colores, texturas y formas que aparecían y desaparecían
como por arte de magia entre la monumental polvareda que levantaba a su
paso.

\clearpage

\section*{ Capítulo VI }
\subsection*{ Junio }
\subsection*{ Masa para hacer fósforos }
\textbf{ingredientes}
\begin{itemize}
    \item 1 onza de nitro\footnote{der Salpeter} del polvo
    \item ½ onza de minio\footnote{die Mennige}
    \item ½ concha de goma arábiga en polvo
    \item 1 dracma de fósforo
    \item azafrán
    \item cartón
\end{itemize}

\subsection*{ Manera de hacerse: }
Disuélvase la~\sidenote{117}goma arábiga en agua caliente hasta que se haga una
masa no muy espesa; estando preparada se le une el fósforo y se disuelve
en ella, al igual que el nitro. Se le pone después en minio suficiente
para darle color.

Tita observaba al doctor Brown realizar estas acciones en silencio.

Estaba sentada junto a la ventana de un pequeño laboratorio que el
doctor tenía en la parte trasera del patio de su casa. La luz que se
filtraba por la ventana le daba en la espalda y le proporcionaba una
pequeña sensación de calor, tan sutil que era casi imperceptible. Sufrió
crónico no le permitía calentarse, a pesar de estar cubierta con su
pesada colcha de lana. Por uno de sus extremos continuaba tejiéndola por
las noches, con un estambre que John le había comprado.

De toda la casa, esa era el lugar preferido de ambos. Tita lo había
descubierto a la semana de haber llegado a la casa del doctor John
Brown. Pues John, en contra de lo que Mamá Elena le había pedido, en
lugar de depositarla en un manicomio la llevó a vivir con él. Tita
nunca dejaría de agradecérselo. Tal~\sidenote{118}vez en un manicomio hubiera
terminado realmente loca. En cambio, aquí, con las cálidas palabras y
las actitudes de John para con ella se sentía cada día mejor. Como en
sueños recordaba su llegada a la casa. Entre imágenes borrosas\footnote{trübe, unscharf}
guardaba en su memoria el intenso dolor que sintió cuando el doctor
le puso la nariz en su lugar.

Después las manos de John, grandes y amorosas, quitándole la ropa y
bañándola; luego con cuidado le había desprendido de todo el cuerpo la
suciedad de las palomas, dejándola limpia y perfumada. Por último le
había cepillado el cabello tiernamente y la habían acostado en una cama
con sábanas almidonadas\footnote{gestärkt, geschniegelt}.

Esas manos la habían rescatado del horror y nunca lo olvidaría.

Algún día, cuando tuviera ganas de hablar le gustaría hacérselo saber a
John; por ahora prefería el silencio. Tenía muchas cosas que ordenar en
su mente y no encontraba palabras para expresar lo que se estaba
cocinando en su interior desde que dejó el rancho. Se sentía muy desconcertada\footnote{ratlos, bestürzt, fassungslos, verblüfft, verwirrt}.
Los primeros días inclusive no quería salir de su cuarto,
ahí le llevaba sus alimentos Caty, una señora norteamericana de setenta años,
que aparte de encargarse de la cocina tenía la misión de cuida de Alex,
el pequeño hijo del doctor. La madre este se había muerto cuando él
nació. Tita escuchaba a Alex reír y corretear\footnote{herumlaufen, herumtollen}
por el patio, sin ánimos de conocerlo.

A veces Tita ni siquiera probada la comida, y era un comida insípida\footnote{abgeschmackt, schal, geschmacklos}
que le desagradaba. En lugar de comer prefería pasarse horas enteras
viéndose las manos. Como un bebé las analizaba y las reconocía como~\sidenote{119}propias.

Las podía mover a su antojo, aún no sabía qué hacer con ellas, aparte de
tejer. Nunca había tenido tiempo de detenerse a pensar en estas cosas.
Al lado de su madre, lo que sus manos tenían que hacer estaba fríamente
determinado, no había dudas. Tenía que levantarse, vestirse, prender el
fuego en la estufa, preparar el desayuno, alimentar a los animales,
lavar los trastes, hacer las camas, preparar la comida, lavar los
trastes, planchar la ropa, preparar la cena, lavar los trastes, día tras
día, año tras año. Sin detenerse un momento, sin pensar si eso era lo
que les correspondía. Al verlas ahora libres de las órdenes de su madre
no sabía qué pedirles que hicieran, nunca lo había decidido por sí
misma. Podían hacer cualquier cosa convertirse en cualquier cosa. ¡Si
pudieran transformarse en aves y elevarse volando! Le gustaría que la
llevaran lejos, lo más lejos posible. Acercándose a la ventana que
daba al patio, elevó sus manos al cielo, quería huir de sí misma, no
quería pensar en tomar una determinación, no quería volver a hablar. No
quería que sus palabras gritaran su dolor.

Deseó con todo el alma que sus manos se elevaran. Permaneció un buen
rato así, viendo el fondo azul del cielo a través de sus inmóviles
manos. Tita pensó que el milagro se estaba convirtiendo en realidad
cuando observó que sus dedos se empezaban a transformar en un tenue\footnote{dünn, fein, zart}
vapor que se elevaba al cielo. Se preparó para subir atraída\footnote{angezogen} por una
fuerza superior, pero nada de eso sucedió. Decepcionada, descubrió que
el humo no le pertenecía.
Provenía de un pequeño cuarto~\sidenote{120}al fondo del patio. Una fumarola\footnote{vulkanische Dampfquelle} desperdigaba\footnote{etw. verstreuen, etw. zerstreuen}
por el ambiente un olor tan agradable y a la vez tan
familiar que la hizo abrir la ventana para poder inhalarlo
profundamente. Con sus ojos cerrados se vio sentada junto a Nacha en el
piso de la cocina mientras hacían tortillas de maíz: vio la olla donde
se cocinaba un puchero\footnote{der Eintopf} de lo más aromático,
junto a él los frijoles soltaban el primer hervor \ndots sin dudarlo
decidió ir a investigar quién cocinaba. No podía tratarse de Caty. La
persona que producía ese tipo de olor con la comida si sabía cocinar.
Sin haberla visto, Tita sentía reconocerse en esa persona, quienquiera
que fuera.

Cruzó el patio con determinación, abrió la puerta y se encontró con una
agradable mujer como ochenta años de edad. Era muy parecida a Nacha. Una
larga trenza cruzada le cubría la cabeza, estaba limpiándose el sudor de
la frente con el delantal. Su rostro tenía claro rasgos indígenas.
Hervía té en un cazo de barro.

Levantó la vista y le sonrío amablemente invitándola a sentarse junto a
ella. Tita así lo hizo. Inmediatamente le ofreció una taza de
delicioso té.
Tita lo tomó despacito, disfrutando al máximo el sabor de esas hierbas
desconocidas y conocidas al mismo tiempo. Qué es sensación más agradable
le producían el calor y el sabor de este infusión.

Permaneció un buen rato al lado de esta señora. Ella tampoco hablaba,
pero no era necesario. Desde un principio se estableció entre ellas una
comunicación que iba más allá de las palabras.

Desde~\sidenote{121}entonces diariamente la había visitado. Pero poco a poco, en
lugar de ella, fue apareciendo el doctor Brown. La primera vez que
sucedió le causó extrañeza, no esperaba encontrarlo ahí, ni tampoco los
cambios que había hecho en la decoración del lugar.

Ahora había muchos aparatos científicos, tubos de ensayo, lámparas,
termómetros, etcétera \ndots la pequeña estufa había perdido el lugar preponderante\footnote{überwiegen, vorherrschen, überwiegend, vorwiegend},
para ocupar un pequeño en el sitio en un rincón de la
habitación. Sentía que no era justa esta relegación\footnote{die Abschiebung, die Verweisung},
pero como no deseaba que sus labios emitieran sonido alguno, se guardó
para más tarde su opinión al respecto junto con la pregunta sobre el paradero\footnote{der Aufenthaltsort, der Verbleib} y la
identidad de esta mujer. Además tenía que reconocer que también disfrutaba
enormemente de la compañía de John. La única diferencia y era que él sí
hablaba y en el lugar de cocinar se dedicaba a poner a prueba sus
teorías de una manera científica.

Esta afición por experimentar la había heredado de su abuela. Una india
kikapú a la que su abuelo había raptado\footnote{jmdn. entführen, jmdn. verschleppen}
y llevado a vivir con él lejos de su tribu. Con todo y que se casó con
ella, la orgullosa y netamente\footnote{klar, eindeutig, netto}
norteamericana familia del abuelo nunca la aceptó oficialmente como su
esposa. Entonces el abuelo le había construido este cuarto al fondo de
la casa, donde la abuela podía pasar la mayor parte del día dedicándose
a la actividad que más le interesaba: investigar las propiedades
curativas de las plantas.

Al mismo tiempo este cuarto le servía de refugio en contra de las
agresiones de su familia. Una de las primeras que recibió fue que le
pusieron el mote\footnote{der Spitzname, die Devise, das Motto}
de \glqq{}la kikapú\grqq{}, en lugar de llamarla~\sidenote{122}por su
verdadero nombre, creyendo que con esto la iban a molestar enormemente.
Para los Brown, la palabra kikapú encerraba lo más desagradable de este
mundo, para no así para \glqq{}luz del amanecer\grqq{}.
Para ella significaba todo lo contrario y era un motivo enorme de
orgullo.

Este era solo un pequeño ejemplo de la gran diferencia de opiniones y
conceptos que existía entre estos representantes de dos culturas tan
diferentes y que hacía imposible que entre los Brown surgiera el deseo
de un acercamiento a las costumbres y tradiciones de \glqq{}luz del amanecer\grqq{}.
Tuvieron que pasar años antes de que se adentraran\footnote{in etw. hineingehen, in etw. eindringen, etw. durchdringen}
un poco en la cultura de \glqq{}la kikapú\grqq{}. Fue cuando el bisabuelo de John,
Petro, estuvo muy enfermo de un mal en los bronquios\footnote{die Bronchie}.
Los accesos de tos lo hacían ponerse morado constantemente. El aire no
odía entrarle libremente a sus pulmones. Su esposa, Mary, conocedora\footnote{der Kenner, die Kennerin}
de nociones\footnote{die Idee, der Begriff, die Vorstellung} sobre
medicina, pues era hija de un médico, sabía que en estos casos el organismo
del enfermo producía mayor cantidad de glóbulos\footnote{das Blutkörperchen, das Kügelchen}
rojos; para contrarrestar\footnote{etw. aufhalten, etw. entgegenwirken, etw. eindämmen}
esta insuficiencia era recomendable aplicar una sangría para prevenir
y un exceso de estos glóbulos produjera un infarto o un trombo, ya que
cualquiera de ellos podía ocasionar la muerte del enfermo.

La bisabuela de John, Mary, entonces empezó a preparar las sanguijuelas\footnote{der Blutegel} con las que aplicaría
la sangría a su esposo. Mientras lo hacía se sentía de lo más orgullosa
de estar al tanto de los mejores conocimientos científicos, que le
permitían cuidar la salud de su familia de una manera moderna y adecuada,
¡no con hierbas como \glqq{}la kikapú\grqq{}.

Las sanguijuelas~\sidenote{123}se ponen dentro de un vaso con medio dedo de
agua, por espacio de una hora. Las partes del cuerpo donde se van a
aplicar se lava con agua tibia azucarada. Entre tanto se colocan las
sanguijuelas en un lienzo limpio y se cubren con él. Después se colocan
sobre la parte en que han de agarrar, sujetándolas bien con el paño y
procurando comprimirlas, para que no vayan a picar por otro lado. Si
después de desprenderlas conviniera\footnote{angemessen sein, nutzlich sein}
la evacuación de sangre, esta se favorece por medio de fricciones de agua
caliente. Para contener la sangre y cerrar las fisuras se cubren con yesca\footnote{der Zunder}
de álamo\footnote{die Pappel} o trapo y luego
se aplica una cataplasma\footnote{der Breiumschlag}
de miga\footnote{der Brösel, der Krümmel, etw. zerbröseln}
de pan y leche, que se retira hasta que las fisuras estén enteramente cicatrizadas\footnote{narbig}.

Mary hizo esto al pie de la letra, pero el caso es que cuando retiraron
las sanguijuelas del brazo de Pedro se empezó a desangrar\footnote{verbluten, ausbluten} y no podían
contenerle la hemorragia\footnote{der Blutfluss, die Blutung}.
Cuando \glqq{}la kikapú\grqq{} escuchó los gritos de desesperación provenientes de
la casa corrió a ver qué era lo que pasaba. Al momento se acercó al enfermo
y al poner una de sus manos sobre las heridas logró inmediato contener
el sangrado. Todos quedaron asombradísimos. Entonces les pidió que por
favor la dejaran solas con el enfermo. Nadie se atrevió a decirle que
no después de lo que acababan de presenciar. Se pasó toda la tarde al lado
de su suegro cantándole melodías extrañas y poniéndole cataplasma de hierbas
entre los humos del incienso y copal\footnote{der Kopal (ein Baumharz)}
que había puesto a quemar. Hasta muy entrada la noche abrió la puerta
de la recámara y salió rodeada de nubes de incienso; tras ella, Petro
hizo~\sidenote{124}su aparición, completamente restablecido.

A partir de ese día \glqq{}la kikapú\grqq{} se convirtió en el médico de la familia
y fue plenamente a desconocida como curandera milagrosa entre la
comunidad norteamericana. El abuelo quiso construirle un sitio más
grande para que practicara sus investigaciones, pero ella se negó. No
podía haber en toda la casa un lugar superior a su pequeño laboratorio.
En el John había pasado la mayor parte de su niñez y adolescencia.
Cuando entró a la universidad dejó de frecuentarlo, pues las modernas
teorías médicas que ahí le enseñaban se contraponían\footnote{etw. dagegenhalten, etw. entgegenstellen, etw. gegenüberstellen}
enormemente con las de su abuela y con lo que él aprendía de ella.
Conforme la medicina fue avanzando, fue llevando a John de regreso a
los conocimientos que su abuela le había dado en sus inicios, y ahora,
después de muchos años de trabajo y estudio, regresaba al laboratorio
convencido de que solo ahí encontraría lo último en medicina. Mismo
que podría llegar a ser del conocimiento público si es que él lograba
comprobar científicamente todas las curaciones milagrosas que
\glqq{}luz del amanecer\grqq{} había realizado.

Tita gozaba enormemente el verlo trabajar. Con él siempre había cosas
que aprender y descubrir, como ahora, que mientras preparaba los cerillos\footnote{das Streichholz}
le estaba dando toda una cátedra\footnote{der Lehrstuhl} sobre
el fósforo y sus propiedades.

- En 1669, Brandt, químico de Hamburgo, buscando la piedra filosofal %
descubrió el fósforo. Él creía que al unir el extracto de la orina con %
un metal conseguiría transmutarlo\footnote{etw. in etw. umwandeln, etw. in etw. verwandeln} %
en~\sidenote{125}oro. Lo que obtuvo fue un cuerpo luminoso por sí mismo, que %
ardía con una vivacidad\footnote{die Aufgewecktheit, die Lebendigkeit, die Lebhaftigkeit} %
desconocida hasta entonces. Por mucho tiempo se obtuvo el fósforo calcinando\footnote{etw. abbrennen, etw. verkohlen} %
fuertemente el residuo %
de la evaporación de la orina en una retorta de tierra cuyo cuello se %
sumergía en el agua. Hoy se extrae de los huesos de los animales, que %
contienen ácido fosfórico y cal.\\

El doctor no por hablar descuidaba la preparación de los fósforos. Sin
ningún problema disociaba\footnote{etw. abspalten, zerfallen, sich auflösen}
la actividad mental de la física. Podría inclusive filosofar sobre aspectos
muy profundos de la vida sin que sus manos cometieran errores o pausas.
Por tanto prosiguió manufacturando los cerillos mientras platicaba con Tita.
\\- Ya teniendo la masa ahora los fósforos, el paso que sigue es preparar %
el cartón para las cerillas. En una libra de agua se disuelve una de %
nitro y se le agrega un poco de azafrán para darle color y en esta %
solución se baña el cartón. Al secarse se corta en pequeñas tiritas y a %
estas se les pone un poco de masa en las puntas. Poniéndolas a secar, %
enterradas en arena.\\

Mientras se secaban las tiras, el doctor le mostró un experimento a
Tita.
\\- Aunque el fósforo no hace combustión en el oxígeno a la temperatura %
ordinaria, es susceptible de arder con gran rapidez a una temperatura %
elevada, mire \ndots \\
\rule{1em}{0pt}El doctor introdujo un pequeño pedazo de fósforo bajo un tubo cerrado
por uno de sus extremos y lleno de mercurio. Hizo fundir el fósforo
acercando el tubo a la llama de una vela. Después, por medio de una
pequeña campana de ensayos llena de gas oxígeno, hizo pasar el gas a la
campana muy poco a~\sidenote{126}poco. En cuanto el gas oxígeno llegó a la parte
superior de la campana, donde se encontraba el fósforo fundido, se
produjo una combustión viva instantánea, que los deslumbró\footnote{jmdn./etw. blenden}
como si fuese un relámpago.

- Como ve, todos tenemos en nuestro interior los elementos necesarios %
para producir fósforo. Es más, déjeme decirle algo que a nadie le he %
confiado. Mi abuela tenía una teoría muy interesante, decía que si bien %
todos nacemos con una caja de cerillos en nuestro interior, no los %
podemos encender solos, necesitamos, como en el experimento, oxígeno y %
la ayuda de una vela. Solo es que en este caso el oxígeno tiene que %
provenir, por ejemplo, del aliento de la persona amada; la vela puede %
ser cualquier tipo de alimento, música, caricia, palabra o sonido que %
haga disparar el detonador\footnote{der Zünder, die Sprengkapsel} %
y así encender uno de los cerillos. Por un momento nos sentiremos deslumbrados %
por una intensa emoción. Se producirá en nuestro interior un agradable %
calor que irá desapareciendo poco a poco conforme pase el tiempo, hasta %
que venga una nueva explosión a reavivarlo. %
Cada persona tiene que descubrir cuáles son los detonadores para poder %
vivir, pues la combustión que se produce al encenderse uno de ellos es %
lo que nutre la energía al alma. En otras palabras, esta combustión es %
su alimento. Si uno no descubre a tiempo cuáles son sus propios detonadores, %
la caja de cerillos se humedece y nunca podremos encender un solo fósforo. %
Si eso llega a pasar el alma huye de nuestro cuerpo, camina errante\footnote{irrend, verfehlend, ziellos} %
por las tinieblas más profundas tratando vanamente\footnote{vergeblich, unbegründet} %
de encontrar alimento por~\sidenote{127}sí misma, ignorante de que solo el cuerpo %
que ha dejado inerme\footnote{wehrlos}, lleno de frío, es el único que %
podría dárselo.\\

¡Qué ciertas eran estas palabras! Si alguien lo sabía era ella.

Desgraciadamente, tenía que reconocía que sus cerillos estaban llenos
de moho y humedad. Nadie podría volver a encender uno solo.

Lo más lamentable era que ella sí conocía cuáles eran sus detonadores,
pero cada vez que había logrado encender un fósforo se lo habían apagado inexorablemente\footnote{unerbittlich, unweigerlich}.

John, como leyéndole el pensamiento, comentó:
\\- Por eso hay que permanecer alejados de personas que tengan un aliento gélido\footnote{eiskalt, gefroren}. Su sola presencia podría apagar %
el fuego más intenso, con los resultados que ya conocemos. Mientras más %
distancia tomemos de estas personas, será más fácil protegernos de su %
soplo. - tomando una mano de Tita entre las suyas, fácil añadió -: %
hay muchas maneras de poner a secar una caja de cerillos húmeda, pero %
puedes estar segura de que tiene remedio.\\

Tita dejó que unas lágrimas se deslizaran por su rostro. Con dulzura\footnote{die Warmherzigkeit, die Sanftheit}
John se las secó con su pañuelo.
\\- Claro que también hay que poner mucho cuidado en ir encendiendo los %
cerillos uno por uno. Porque si por una emoción muy fuerte se llegan a %
encender todos de un solo golpe producen un resplandor\footnote{der Lichtschein, der Lichschimmer} %
tan fuerte que ilumina más allá de lo que podemos ver normalmente y %
entonces ante nuestros ojos aparece un túnel esplendoroso\footnote{glanzvoll, leuchtend, strahlend} %
que nos muestra el camino que~\sidenote{128}olvidamos al momento de nacer y que %
nos llama a reencontrar nuestro perdido origen divino\footnote{göttlich}. %
El alma desea reintegrarse al lugar de donde proviene, dejando al cuerpo %
inerte \ndots Desde que mi abuela %
murió he tratado de demostrar científicamente esta teoría. Tal vez algún %
día lo logre. ¿Usted qué opina?\\

El Dr Brown guardó silencio, para darle tiempo a Tita de comentar algo
si así lo deseaba. Pero su silencio era como de piedra.
\\- Bueno, no quiero aburrirla con mi plática. Vamos a descansar, pero %
antes de irnos quisiera enseñarle un juego que mi abuela y yo %
practicábamos con frecuencia. Aquí pasábamos la mayor parte del día y %
entre juegos me transmitió todos sus conocimientos. Ella era una mujer %
muy callada, así como usted. Se sentaba frente a esa estufa, con su gran %
trenza cruzada sobre la cabeza y solía adivinar lo que yo pensaba. Yo %
quería aprender a hacerlo, así que después de mucho insistirle me dio la %
primera lección. Ella escribía utilizando una sustancia invisible, y sin %
que yo la viera, una frase en la pared. Cuando por la noche yo veía la %
pared, adivinaba lo que ella había escrito. ¿Quiere que hagamos la %
prueba?\\

Con esta información Tita se enteró de que la mujer con la que tantas
veces había estado era la difunta\footnote{verstorben, tot}
abuela de John. Ya no tenía que preguntarlo.
El doctor tomó con un lienzo un pedazo de fósforo y se lo dio a
Tita.
\\- No quiero romper la ley del silencio que se ha impuesto, así que como %
un secreto entre los dos, le voy a pedir~\sidenote{129}que en cuanto yo salga %
usted me escriba en esta pared las razones por las que no habla, ¿de %
acuerdo? Mañana yo las adivinaré ante usted.\\

El doctor, por supuesto, omitió decirle a Tita que una de las
propiedades del fósforo era la que haría brillar por la noche lo que
ella hubiera escrito en la pared. Obviamente, él no necesitaba este subterfugio\footnote{die Ausflucht, die Ausrede, der Vorwand}
para conocer lo que ella pensaba, claro confiaba en que este
sería un buen comienzo para que Tita entablara
nuevamente una comunicación consciente con el mundo, aunque esta fuera por escrito.
John percibía que ya estaba lista para ello. En cuanto el doctor salió,
Tita tomó el fósforo y se acercó al muro.

En la noche, cuando John Brown entró al laboratorio, sonrío complacido
al ver escrito en la pared con letras firmes y fosforescentes: \glqq{}porque
no quiero\grqq{}. Tita con estas tres palabras había dado el primer
paso hacia la libertad.

Mientras tanto, Tita, con los ojos fijos en el techo, no podía dejar
de pensar en las palabras de John: ¿Sería posible hacer vibrar su alma
nuevamente? Deseó con todo su ser que así fuera.

Tenía que encontrar a alguien que lograra encenderle este anhelo\footnote{die Begierde, die Sehnsucht, der Herzenswunsch}.

¿Si esa persona fuera John? Recordaba la placentera sensación que le
recorrió el cuerpo cuando él la tomó de la mano en laboratorio. No. No
lo sabía. De lo único que estaba convencida es de que no quería volver
al rancho. No quería vivir cerca de Mamá Elena nunca más.

\clearpage

\section*{ Capítulo VII }
\subsection*{ Julio }
\subsection*{ Caldo de colita de res }
\textbf{Ingredientes}
\begin{itemize}
    \item 2 colitas de res
    \item 1 cebolla
    \item 2 dientes de ajo
    \item 4 jitomates
    \item ¼ de kilo de ejotes
    \item 2 papas
    \item 4 chiles moritas
\end{itemize}

\subsection*{ Manera de hacerse: }
Las~\sidenote{135}colitas partidas se ponen a cocinar con un trozo de cebolla,
un diente de ajo, sal y pimienta al gusto. Es conveniente poner un poco
más agua de la que normalmente se utiliza para un cocido, teniendo en
cuenta que vamos a preparar un caldo. Y un buen caldo que se respete
tiene que ser caldoso, sin caer en lo aguado.

Los caldos pueden curar cualquier enfermedad física o mental, bueno, al
menos esa era la creencia de Chencha y Tita, y por mucho tiempo no le
había dado el crédito suficiente. Ahora no podía menos que aceptarla
como cierta.

Hacía tres meses, al probar una cucharada del caldo que Chencha le
preparó y le llevó a la casa del Dr John Brown, Tita había recobrado\footnote{etw. wiedererlangen, etw. wiederbekommen}
toda su cordura\footnote{der Verstand}.

Estaba recargada en el cristal,
viendo a través de la ventana a Alex, el hijo de John, en el patio corriendo
tras unas palomas.

Escuchó los pasos de John subiendo las escaleras, esperaba con ansia su
acostumbrada visita. Las palabras de John eran su único enlace con el
mundo. Si pudiera hablar y decirle lo importante que era para ella su presencia
y su plática. Si~\sidenote{136}pudiera bajar y besar a Alex como al hijo que no
tenía y jugar con él hasta el cansancio, si pudiera recordar cómo
cocinar tan siquiera un par de huevos, si pudiera gozar de un platillo
cualquiera que fuera, si pudiera \ndots volver a la vida. Un olor que
percibió la sacudió\footnote{jmdn./etw. durchrütteln, jmdn./etw. schütteln}.
Era un olor ajeno a esta casa. John abrió la puerta y apareció ¡con una
charola en las manos y un plato con caldo de colita de res!

¡Un caldo de colita de res! No podía creerlo. Atrás John entró Chencha
bañada en lágrimas. El abrazo que se dieron fue breve, para evitar que
el caldo se enfriara. Cuando dio el primer sorbo Nacha llegó a su lado y
le acarició la cabeza mientras comía, como lo hacía cuando de niña ya se
enferma, y la besó repetidamente en la frente. Ahí estaban, junto a
Nacha, los juegos de su infancia en la cocina, las salidas al mercado,
las tortillas recién cocidas, los huesitos de chabacano de colores, las
tortas de Navidad, su casa, el olor a leche hervida, a pan de natas, a
champurrado, a comino, a ajo, a cebolla. Y como toda la vida, al sentir
el olor que despedía la cebolla, las lágrimas hicieron su aparición.
Lloró como no lo hacía desde el día en que nació. Qué bien le hizo
platicar largo rato con Nacha. Igual que en los viejos tiempos, cuando
Nacha aún vivía y juntas habían preparado infinidad de veces caldo de
colita. Rieron al revivir esos momentos y lloraron al recordar los pasos
a seguir en la preparación esta receta. Por fin había logrado recordar
una receta, al rememorar como primer paso la picada de la cebolla.

La cebolla y el ajo se pican finamente y se ponen a freír en un poco de
aceite; una vez que se acitronan se les~\sidenote{137}incorporan las papas, los
ejotes y el jitomate picado hasta que se sazonen\footnote{reifen, etw. reifen lassen, etw. abschmecken, etw. würzen}.

John interrumpió estos recuerdos al entrar bruscamente en el cuarto,
alarmado por el riachuelo\footnote{das Flüsschen, das Bächlein}
que corría escaleras abajo.

Cuando se dio cuenta de que se trataba de las lágrimas de Tita, John bendijo\footnote{jmdn./etw. segnen, jmdn./etw. weihen}
a Chencha y a su caldo de colita por haber logrado lo que
ninguna de sus medicinas había podido: que Tita llorara de esa manera. Apenado\footnote{jmdn. bekümmern, sich um jmdn./etw. sorgen }
por la intromisión\footnote{die Einmischung}, se dispuso
a retirarse. La voz de Tita se lo impidió. Esa melodiosa voz que no había
pronunciado palabra en seis meses.
\\- ¡John! ¡No se vaya, por favor!\\

John permaneció a su lado y fue testigo de cómo pasó Tita de las
lágrimas a las sonrisas al escuchar por boca de Chencha todo tipo de
chismes e infortunios\footnote{das Unglück, das Ungemach}.
Así se enterró el doctor de que Mamá Elena tenía
prohibidas las visitas a Tita. En la familia de la Garza se podía
perdonar algunas cosas, pero nunca la desobediencia\footnote{der Ungehorsam}
ni el cuestionamiento\footnote{das Infragestellen} de
las actitudes de los padres. Mamá Elena no le perdonaría jamás a Tita,
loca o no loca, la hubiera culpado de la muerte de su nieto. Y al igual
que con Gertrudis tenía vetado\footnote{gegen etw. ein Veto einlegen}
inclusive el que se pronunciara su nombre. Por cierto, Nicolás había
regresado hacía a poco con noticias de ella.

Efectivamente la había encontrado trabajando en un burdel. Le había
entregado su ropa y ella le había mandado una carta a Tita. Chencha se
la dio y Tita la leyó en silencio:
\\[1em]
\begin{itshape}
Querida Tita:
\\

No~\sidenote{138}sabes cómo te agradezco el que me hayas enviado mi ropa. Por
fortuna aún me encontraba aquí y la pude recibir. Mañana voy a dejar
este lugar, pues no es el que me pertenece. Aún no sé cuál sea, pero sé
que en alguna parte tengo que encontrar un sitio adecuado para mí. Si
caí aquí fue porque sentía que un fuego muy intenso me quemaba por
dentro, el hombre que me recogió en el campo prácticamente me salvó la
vida. Ojalá no vuelva a encontrar algún día. Me dejó porque sus fuerzas
se estaban agotando a mi lado, sin haber logrado aplacer\footnote{jmdn./etw. beruhigen, jmdn./etw. besänftigen}
mi fuego interior. Por fin ahora después de que infinidad de hombres han pasado
por mí, siento un gran alivio. Tal vez algún día regrese a casa y te lo
puedo explicar.
\\~\\
Te quiere, tu hermana Gertrudis.
\end{itshape}
\\[1em]

Tita guardó la carta en la bolsa de su vestido y no hizo el menor
comentario. El que Chencha no le preguntara nada sobre el contenido de
la carta indicaba claramente que ya la había leído al derecho y al
revés.

Más tarde, entre Tita, Chencha y John secaron la recámara y las
escaleras y la planta baja.

Al despedirse, Tita le comunicó a Chencha su decisión de no regresar
nunca más al rancho y le pidió que se lo hiciera saber a su madre.
Mientras Chencha cruzaba por enésima\footnote{zum x-ten Mal}
vez el puente entre Eagle Pass y Piedras Negras, sin darse cuenta, pensaba
cuál sería la mejor manera de darle la noticia a Mamá Elena. Los celadores\footnote{der Aufseher, die Aufseherin} de
ambos países la dejaron hacerlo, pues la conocían desde niña. Además
resultaba de lo más divertido verla caminar de un lado a otro hablando
sola y mordisqueando\footnote{etw. anknabbern, etw. abknabbern}
su rebozo. Sentía que su ingenio para inventar estaba paralizado por el
terror.

Cualquier versión que diera de seguro iba a enfurecer a Mamá Elena .
Tenía que inventar una en la cual al menos ella saliera bien librada.
Para lograrlo tenía que encontrar una excusa que disculpara la visita
que le había hecho a Tita. Mamá Elena no se la tragaría ninguna. ¡Como
si no la conociera! Envidiaba a Tita por haber tenido el valor de no
regresar al rancho. Ojalá y pudiera hacer lo mismo, pero no se atrevía.
Triste niña había oído hablar de lo mal que les va a las mujeres que
desobedecen a sus padres o a sus patrones y se van de la casa. Acaban revolcadas\footnote{jmdn./etw. zu Boden werfen, jmdn./etw. zu Fall bringen}
en el arroyo inmundo\footnote{schmutzig, unrein} de la vida
galante. Nerviosa daba vueltas y vueltas a su rebozo, tratando de
exprimirle la mejor de sus mentiras para estos momentos. Nunca antes
le había fallado. Al llegar a las cien retorcidas\footnote{gekrümmt, verdreht}
al rebozo siempre encontraba el embuste\footnote{die Flunkerei} apropiado\footnote{geeignet, maßgeschneidert, tauglich}
para la ocasión. Para ella mentir era una práctica de sobrevivencia que
había aprendido desde su llegada al rancho. Era mucho mejor decir que
el Padre Ignacio la había puesto a recoger las limosnas\footnote{die Spende, der Almosen}
que reconocer que se le había tirado la leche por estar platicando
en el mercado. El castigo al cual una se hacía merecedora era
completamente diferente.

Total, todo podía ser verdad o mentira, dependiendo de que uno se
creyera las cosas verdaderamente o no.

Por ejemplo, todo lo que había imaginado~\sidenote{140}sobre la suerte de Tita
no había resultado cierto.

Todos estos meses se los había pasado angustiada pensando en los
horrores por los que estaría pasando fuera de la cocina de su casa.
Rodeada de locos gritando obscenidades, atada por una camisa de fuerza y
comiendo qué sabe qué tipo de comida horrenda fuera de casa. Imaginaba
la comida de un manicomio, y gringo, para acabarla de amolar, como lo
peor del mundo. Y la verdad, a Tita le había encontrado bastante bien,
nunca había puesto un pie en un manicomio, se veía que la trataban de lo
más bien en casa del doctor y no ha de haber comido tan mal, pues le
notaba hasta unos kilitos de más. Pero eso sí, por mucho que hubiera
comido nunca le habían dado algo como el caldo de colita. De eso sí
podía estar bien segura, si no, ¿por qué había llorado tanto cuando lo
comió?

Pobre Tita, de seguro ahora que la había dejado estaría llorando
nuevamente, atormentada por los recuerdos y la idea de no volver a
cocinar al lado de Chencha nunca más. Si, de seguro estaría sufriendo
mucho. Nunca se le hubiera ocurrido imaginarla como realmente estaba,
bellíssima, luciendo un vestido de raso\footnote{das Ödland, das Satin} tornasol\footnote{das Changieren} en encajes\footnote{das Spitzenkleid},
cenando a la luz de la luna y recibiendo una declaración de amor. Para
la mente sufridora y exagerada de Chencha esto hubiera sido demasiado.
Tita estaba sentada cerca de una fogata asando un malvavisco\footnote{Echter Eibisch}.
A su lado John Brown le proponía matrimonio. Tita había aceptado acompañar a
John a una lunada en un rancho vecino para festejar que la acaba de
dar de alta. John~\sidenote{141}le había regalado un hermoso vestido que desde
hacía tiempo había comprado en San Antonio, Texas, para este momento. Su
color tornasol la hacía recordar el plumaje que las palomas tienen el
cuello, pero ya sin ninguna asociación dolorosa con el lejano día en que
se encerró en el palomar. Francamente, estaba completamente recuperada y
dispuesta a iniciar una nueva vida al lado de John. Con un tierno beso
en los labios sellaron\footnote{etw. stempeln, etw. siegeln} su
compromiso. Tita no sintió lo mismo que cuando Pedro la había besado,
pero espera para que su alma por tanto tiempo enmohecida\footnote{schimmelig} lograra poco a poco a encenderse
con la cercanía de este hombre tan maravilloso.

¡Por fin, después de haber caminado tres horas, Chencha tenía ya la
respuesta! Como siempre, había encontrado la mentira idónea\footnote{geeignet}.
Le diría a Mamá Elena que paseando por Eagle Pass se había encontrado
en un esquina a una limosnera con la ropa sucia y desgarrada\footnote{etw. zerrißen, etw. zerfetzen}.
Que la compasión la había hecho acercársele para darle diez centavos, y
que azorada\footnote{jmdn./etw. ängstigen, jmdn./etw. erschrecken}
descubrió que se trataba de Tita. Se había escapado del manicomio y vagaba\footnote{herumbummeln, umherstreifen} por el
mundo pagando la culpa de haber insultado a su madre. Ella la había
invitado a regresar, claro Tita se había negado. No se sentía merecedora\footnote{verdienstlich, wert, würdig} vivir nuevamente
al lado de tan buena madre y le había pedido que por favor le dijera a su
mamá que la quería mucho y que nunca olvidaría lo mucho que siempre había
hecho por ella, prometiendo que en cuanto se hiciera una mujer de bien
regresaría a su lado para darle todo el amor y el respecto que Mamá Elena
se merecía.

Chencha~\sidenote{142}pensaba cubrirse de gloria con esta mentira, pero
infortunadamente no lo pudo lograr. Esa noche, al llegar a la casa, un
grupo de bandoleros atacó el rancho. A Chencha la violaron y Mamá Elena,
a tratar de defender su honor, recibió un fuerte golpe en la espalda y
este le provocó una paraplejia que la paralizó de la cintura para abajo.
En esas condiciones no estaba para recibir ese tipo de noticias, ni
Chencha como para darlas.

Por otro lado estuvo bien que no le hubiera dicho nada, pues con el retorno\footnote{die Rückkehr} de Tita al rancho a conocer
la desgracia, su piadosa mentira se
habría venido a pique ante la esplendorosa belleza y energía que Tita
irradiaba. Su madre la recibió en silencio. Y por primera vez Tita le
sostuvo firmemente la mirada y Mamá Elena retiró la suya. Había en la
mirada de Tita una luz extraña.

Mamá Elena desconocía a su hija. Sin palabras hicieron mutuos\footnote{beiderseitig, gegenseitig}
reproches y con esto se rompió entre ellas el hasta entonces fuerte
lazo de sangre y obediencia\footnote{der Gehorsam} que las
unía y que ya nunca se restablecería. Tita sabía muy bien que su madre
se sentía profundamente humillada al tener que aceptarla nuevamente en
casa, y no solo eso, sino al necesitar sus cuidados para su restablecimiento.
Por tanto intentó de todo corazón atenderla lo mejor posible. Con mucho
cuidado preparaba la comida para su madre y en especial el caldo de colita,
con la sana intención de que le sirviera como ella para recuperarse
totalmente.

Vació el caldillo ya sazonado con las papas y los ejotes en la olla
donde había puesto a cocer las colitas de res.

Ya que~\sidenote{143}se vacía, solo hay que dejar hervir por media hora todos
los ingredientes juntos. Enseguida se retira del fuego y se sirve bien
caliente.

Tita sirvió el caldo y lo subió a su madre en una hermosa charola de
plata cubierta con una servilleta de algodón bellamente deshilada y
perfectamente blanqueada y almidonada.

Tita esperaba con ansiedad la reacción positiva de su madre en cuanto
diera el primer sordo, pero por el contrario Mamá Elena escupió el
alimento sobre la colcha y a gritos le pidió a Tita que inmediatamente
le retirara de su vista esa charola.
\\- Pero, ¿por qué? %
\\- Porque está asquerosamente amargo, no lo quiero. ¡Llévatelo! ¿No me %
oíste?\\

Tita en lugar de obedecerla dio media vuelta tratando de ocultar a los
ojos de su madre el sentimiento de frustración que experimentaba. No
entendía la actitud de Mamá Elena. Nunca la había entendido. Escapaba a
su comprensión el que un ser, independientemente parentesco\footnote{die Verwandtschaft}
que pudiera tener con otro, así nomás\footnote{kaum, lediglich},
con la mano en la cintura rechazara de una manera tan brutal una atención.
Porque estaba segura de que el caldo estaba exquisito. Ella misma lo había
probado antes de subirlo. No podía ser de otra manera, pues había puesto
mucho cuidado al prepararlo.

Se sentía verdaderamente una estúpida por haber regresado al rancho
para atender a su madre. Lo mejor hubiera sido quedarse en casa de John
sin pensar nunca más en la suerte que hoy que pudiera correr Mamá Elena.
Pero los~\sidenote{144}remordimientos no la hubieran dejado. La única manera de
liberarse realmente de ella sería con la muerte y Mamá Elena aún no
tenía para cuándo.

Sentía ganas de correr lejos, muy lejos para proteger de la gélida
presencia de su madre el pequeño fuego interior que John con trabajos
había logrado encender. Era como si el escupitajo\footnote{die Spucke}
de Mamá Elena hubiera caído justo en el centre de la incipiente\footnote{einsetzend, beginnend}
hoguera y la hubiera extinguido. Sufría dentro de si los efectos del apagón\footnote{der Stromausfall, der Blackout}; el humo le subía
a la garganta y se le arremolinaba\footnote{sich zusammendrängen, sich zusammenscharen}
en un nudo espeso, que le nublaba la vista y le producía lagrimeo.
Bruscamente abrió la puerta y corrió, en el preciso momento en que
John llegaba a realizar su visita médica. Chocaron intempestivamente\footnote{zur Unzeit, ungelegen}.
John la sostuvo en sus brazos justo a tiempo para evitar que cayera.
Su cálido abrazo salvó a Tita de una congelación\footnote{das Gefrieren, die Unterkühlung}.
Fueron solo unos instantes los que estuvieron unidos pero los suficientes
como para reconfortarle
el alma. Tita estaba empezando a dudar si está sensación
de paz y seguridad que John le daba era el verdadero amor, y no el ansia
y el sufrimiento que experimentaba al lado de Pedro. Con verdadero
esfuerzo se separó de John y salió de la recámara.
\\- ¡Bendito, ven acá! ¡Te dije que te llevaras esto! %
\\- Doña Elena, no se altere\footnote{sich beunruhigen, sich (über etw.) aufregen, etw. abändern, etw. verändern} %
por favor, le hace daño. Yo le quito esa charola, pero, dígame, ¿no tiene deseos de comer?\\

Mamá Elena le pidió al doctor que cerrara la puerta con llave y casi en
secreto le externó su inquietud respecto a lo amargo de la~\sidenote{145}comida.
John le respondió que tal vez debía al efecto de las medicinas que
estaba tomando.
\\- De ninguna manera, doctor, si fuera la medicina todo el tiempo tendría %
esa sabor en la boca y no es así. Algo me están dando con la comida. %
Curiosamente desde que Tita regresó. Necesito que lo investigue.\\

John, sonriendo ante la maliciosa insinuación\footnote{die Andeutung, die Anspielung},
se acercó a probar el caldo de colita que le habían llevado y que estaba
intacto en la charola.
\\- A ver, vamos a descubrir que le están poniendo en la comida. ¡Mmmmm! %
Qué delicia. Esto tiene ejotes, papas, chile y \ndots no logro %
distinguir bien \ndots qué tipo de carne es. %
\\- No estoy para juegos, ¿no siente un sabor amargo? %
\\- No, doña Elena, para nada. Pero si quiere lo mando analizar. No quiero %
que se preocupe. Pero mientras me dan los resultados tiene que comer. %
\\- Entonces mándame una buena cocinera. %
\\- ¡Pero cómo! Si tiene en casa a la mejor. Tengo entendido que su hija %
Tita es una cocinera excepcional. Un día de estos voy a pedirle su mano. %
\\- ¡Ya sabe que ella no se puede casar! - exclamó presa de una furiosa %
agitación.\\

John guardó silencio. No le convenía irritar más a Mamá Elena. Y tenía
caso puesto que estaba plenamente convencido de que él se casaría con
Tita con o sin la autorización de ella. Sabía también que ahora a Tita
la tenía muy sin cuidado su absurdo destino y que en cuanto cumpliera 18
años se casarían. Dio por terminada la visita, pidiéndole calma a Mamá
Elena y prometiéndole que al día~\sidenote{146}siguiente le mandaría una nueva
cocinera. Y así lo hizo, pero Mamá Elena ni siquiera se dignó\footnote{geruhen etw. zu tun, sich herablassen etw. zu tun}
recibirla. El comentario del doctor sobre la idea de pedir la mano de
Tita le había abierto los ojos.

De seguro que entre los dos había surgido una relación amorosa.

Desde hacía tiempo sospechaba Tita deseaba que ella desapareciera de
este mundo para así poderse casar libremente, no una sino mil veces si
le daba la gana. Este deseo lo percibía como una presencia constante
entre ellas, en cada roce, en cada palabra, en cada mirada. Pero ahora
no le cabía la menor duda de que Tita intentaba envenenarla\footnote{jmdn./etw. vergiften}
poco a poco para poder casarse con el doctor Brown. Por tanto, desde ese día se negó
terminantemente a comer nada que Tita hubiera cocinado. Le ordenó a
Chencha que se hiciera cargo de la preparación de su comida. Solo ella y
nadie más podía llevársela y la tenía que probar en su presencia antes
de que Mamá Elena se animara a comerla.

La nueva disposición no afectó para nada a Tita, es más, fue para ella
un alivio a delegar en Chencha la penosa obligación de atender a su
madre y así tener libertad para empezar a bordar las sábanas para su
ajuar de novia. Había decidido casarse con John, en cuanto su madre
estuviera mejor.

La que sí se vio muy afectada por la orden fue Chencha. Aún se estaba restableciendo\footnote{sich erholen,  etw. wiederherrichten, etw. wiederherstellen}
física y emocionalmente del brutal ataque del que fue
objeto. Y aunque aparentemente se veía beneficiada al no tener que
realizar ninguna otra~\sidenote{147}tarea más que la de hacer la comida y
llevársela a Mamá Elena, no era así. Al principio recibió con gusto la
noticia, pero en cuanto empezaron los gritos y los reproches se dio
cuenta de que no hay pan que no cueste una torta.

Un día en que había ido a que doctor John Brown le quitara las costuras\footnote{das Nähen, die Naht} que le
había tenido que hacer, pues había sufrido un desgarre\footnote{der Riss, etw. zerreißen, etw. zerfetzen}
la violación, Tita preparó la comida en su lugar.

Creyeron que podrían engañar a Mamá Elena sin mayor problema. A su
regreso Chencha la llevó la comida y la probó como siempre lo hacía,
pero al dársela a comer a ella, Mamá Elena de inmediato detectó el sabor
amargo. Con enojo lanzó la charola al piso y corrió a Chencha de la
casa, por intentarlo por burlarse de ella.

Chencha se aprovechó de este pretexto para irse a pasar unos días a su
pueblo. Necesitaba olvidarse del asunto de la violación y de la
existencia de Mamá Elena. Tita trató de convencerla de que no le hiciera
caso a su mamá.

Tenía muchos años de conocerla y ya sabía muy bien cómo manejarla.
\\- ¡Sí niña, pero ‘orita pa‘ qué quiero más agrura, si con el mole %
tengo! Déjame ir, no seas ingrata.\\

Tita la abrazó y la consoló como lo había hecho todas las noches desde
su regreso. No veía la manera de sacar a Chencha de su depresión y de la
creencia de que ya nadie se casaría con ella después del violento ataque
que sufrió por parte de los bandoleros.

- Ya ves cómo son los hombres~\sidenote{148}. Toditos dicen que plato de segunda %
mesa ni en otra vida, ¡menos en esta!\\

Al ver su desesperación, Tita decidió dejarla ir. Por experiencia sabía
que si permanecía en el rancho y cerca de su madre no tendría salvación.
Solo la distancia podría hacerla sanar. Al otro día la mandó con Nicolás
a su pueblo.

Tita entonces se vio en la necesidad de contratar una cocinera. Pero
esta se fue de la casa a los tres días de haber llegado. No soportó las exigencias\footnote{die Anforderungen, die Erfordernisse}
ni los malos modos de Mamá Elena. Entonces buscaron a otra,
que solo duró dos días, y a otra y a otra hasta que no quedó ninguna en el
pueblo que quisiera trabajar en la casa. La que más duró fue una
muchacha sordomuda\footnote{der Gehörlose, die Gehörlose, der Taubstumme, die Taubstumme}:
aguantó quince días, pero se fue porque mamá Elena le había dicho a
señas que era una mensa\footnote{dumm, einfältig}.

Entonces a Mamá Elena no le quedó otra que comer lo que Tita cocinaba,
pero lo hacía con las debidas precauciones. Aparte de exigir que Tita
probar a la comida antes que ella, siempre pedía que le llevara un vaso
de leche tibia con cada comida y se lo tomaba antes de ingerir los
alimentos, para contrarrestar los efectos del amargo veneno que, según
ella, percibía disuelto en la comida. Algunas veces solo esta medida era
suficiente, pero en ocasiones sentía vivos dolores en el vientre,
entonces se tomaba, además, un trago de vino de ipecacuana\footnote{die Brechwurz}
y otro de cebolla de albarrana como vomitivo. No fue por mucho tiempo. Al mes
murió Mamá Elena presa de unos dolores espantosos\footnote{entsetzlich, grässlich, traurig}
acompañados de espasmos y convulsiones\footnote{der Krampf}
intensas. En un principio, Tita y John no se
explicaban esta extraña muerte~\sidenote{149}, pues aparte de la paraplejia\footnote{die Querschnittslähmung} Mamá
Elena clínicamente no tenía ninguna otra enfermedad. Pero al revisar su buró\footnote{der Schreibtisch, der Nachttisch}
encontraron el frasco de vino de ipecacuana y dedujeron que de
seguro Mamá Elena lo había estado tomando a escondidas. John le hizo
saber a Tita que este vomitivo es tan fuerte que puede provocar la
muerte.

Tita no podía quitarle la vista al rostro de su madre durante el velorio\footnote{die Totenwache}. Hasta ahora, después de
muerta, la veía por primera vez y la empezaba a comprender. Quien la
viera podría fácilmente confundir esa mirada de reconocimiento con una
mirada de dolor, pero Tita no sentía dolor alguno. Hasta ahora comprendía
el significado de la frase \glqq{}fresca como una lechuga\grqq{}, así de extraña y
lejana se debería sentir una lechuga ante su repentina\footnote{plötzlich, jäh, schlagartig}
separación de otra lechuga con la que hubiera crecido. Sería ilógico\footnote{unlogisch, widersinning} esperar que sufriera
por la separación de esa lechuga con la que nunca había podido hablar
ni establecer ningún tipo de comunicación y de la que solo conocía las
hojas exteriores, ignorando que en su interior había muchas otras escondidas\footnote{heimlich}.

No podía imaginar a esa boca con rictus\footnote{der Gesichtsausdruck, das Zwangslachen}
amargo besando con pasión, ni esas mejillas ahora amarillentas, sonrosadas\footnote{rosenfarbig, rosig} por el calor de una noche
de amor. Y sin embargo así había sido alguna vez. Y Tita lo había
descubierto hasta ahora, demasiado tarde y de una manera meramente\footnote{bloß, lediglich}
circunstancial. Cuando Tita la estaba vistiendo, ahora el velorio, le
quitó de la cintura el enorme llavero\footnote{der Schlüsselanhänger, der Schlüsselbund}
que como una cadena la había acompañado desde que ella recordaba. En
la casa todo estaba bajo llave y bajo estricto~\sidenote{150}control. Nadie
podía sacar ni una taza de azúcar de la despensa sin autorización de
Mamá Elena. Tita conocía las llaves de todas las puertas y escondrijos\footnote{der Schlupfwinkel, der Unterschlupf}.
Pero además del enorme llavero, tenía colgado al cuello un pequeño dije\footnote{der Anhänger - Schmuckstück} en forma
de corazón y dentro de él había una pequeña llave que le llamó la atención.

De inmediato relacionó la llave con la cerradura indicada. De niña, un
día jugando en las escondidillas se había metido en el ropero\footnote{der Kleiderschrank, der Kleiderhändler}
de Mamá Elena. Entre las sábanas había descubierto un pequeño cofre\footnote{die Truhe}.
Mientras Tita esperaba que la fueran a buscar trató inútilmente de abrirlo,
pues estaba bajo llave. Mamá Elena a pesar de no estar jugando a las
escondidas fue quien la encontró al abrir el ropero. Había ido por una
sábana o algo así y la cogió con las manos en la masa. La castigó en el
granero y la pena consistió en desgranar cien elotes. Tita sintió que la
falta no ameritaba el castigo tan grande, esconderse con zapatos entre
las sábanas limpias no era para tanto. Hasta ahora, muerta su madre,
mientras leía las cartas que contenía el cofre, se daba cuenta de que no
había sido castigada por eso, sino por haber intentado ver el contenido
del cofre, y que el castigo sí era para tonto.

Tita abrió el cofre con morbosa\footnote{ungesund, krankhaft, auf etw. heiß sein, auf etw. scharf sein}
curiosidad. Contenía un paquete de cartas de un tal José Treviño y un
diario. Las cartas estaban dirigidas a Mamá Elena. Tita las ordenó por
fechas y se enteró de la verdadera historia de amor de su madre. José
había sido el amor de su vida. No le habían permitido casarse con él
pues tenía en sus venas sangre negra.
Una colonia de negros, huyendo de la guerra civil en~\sidenote{151}U.S.A y del
peligro que corrían de ser linchados\footnote{jmdn. lynchen},
había llegado a instalarse cerca del pueblo. José era el producto de los
amores ilícitos\footnote{unerlaubt, ungültig, unbefugt} entre José
Treviño padre y una guapa negra. Cuando los padres de Mamá Elena habían
descubierto el amor que existía entre su hija y este mulato,
horrorizados la obligaron inmediatamente a casarse con Juan de la Garza,
su padre.

Esta acción no logró impedir que aún estando casada siguiera manteniendo
correspondencia secreta con José, y tal parecía que no se habían
conformado solamente con este tipo de comunicación, pues según estas
cartas, Gertrudis era hija de José y no de su padre.

Mamá Elena había intentado huir con José al enterarse de este embarazo,
pero la noche en que lo esperaba escondida tras lo oscuro del balcón
presenció cómo un hombre desconocido, sin motivo aparente\footnote{anscheinend, ersichtlich},
protegiéndose entre las sombras de la noche atacaba a José eliminándolo
este mundo.
Después de grandes sufrimientos Mamá Elena se resignó entonces a vivir
al lado de su legítimo marido, Juan de la Garza por muchos años y
ignoró toda esta historia, pero se enteró de ella precisamente cuando
Tita nació. Había ido a cantina a festejar con unos amigos el nacimiento
de su nueva hija y ahí alguna lengua venenosa le había soltado la
información. La terrible noticia le provocó un infarto. Esa era todo.

Tita se sentía culpable de haber participado de este secreto. No sabía
qué hacer con estas cartas. Pensó en quemarlas pero ella no era quien
para hacerlo; si su madre~\sidenote{152}no se había atrevido, ella menos. Guardó
todo tal y como lo había encontrado y lo puso en su lugar.

Durante el entierro Tita realmente lloró por su madre. Pero no por la
mujer castrante\footnote{jmdn. entmannen, jmdn. kastrieren}
que la había reprimido
toda la vida, si no por ese ser que había vivido un amor frustrado.
Y juró ante su tumba que ella nunca renunciaría
el amor, pasara lo que pasara. En esos momentos estaba
convencida de que su verdadero amor era John. El hombre que estaban a su
lado apoyándola incondicionalmente. Pero en cuanto vio que se acercaba
un grupo de gentes al panteón y distinguió a lo lejos la silueta de
Pedro acompañado de Rosaura ya no estuvo tan segura de sus sentimientos.

Rosaura, luciendo una gran panza de embarazada, caminaba lentamente. En
cuanto vio a Tita se le acercó y la abrazó a llorando desconsoladamente.
Le seguía en turno Pedro. En cuanto Pedro la abrazó su cuerpo vibró como
una gelatina. Tita bendijo
a su madre por darle el pretexto de poder volver a ver y abrazar a Pedro.
Inmediatamente después, se retiró bruscamente. Pedro no se merecía el que
lo quisiera tanto. Había mostrado debilidad al irse lejos de ella y eso
no se lo perdonaba.

John tomó a Tita de la mano durante el regreso al rancho, y Tita a su
vez lo tomó del brazo enfatizando\footnote{etw. betonen} que
entre ellos había algo más que amistad. Quería provocarle a Pedro los
mismos dolores que ella siempre había sentido al verlo al lado de su
hermana.

Petro los observó con los ojos entrecerrados. No le gustaba nada la
familiaridad con la que John se acercaba y con la que Tita le hablaba al
oído. ¿Qué era lo que estaba pasando? Tita~\sidenote{153}le pertenecía y no iba
a permitir que se la quitaran. Mucho menos ahora que había desaparecido
el mayor impedimento\footnote{das Hindernis, der Hinderungsgrund}
para su unión: Mamá Elena.

\clearpage

\section*{ Capítulo VIII }
\subsection*{ Agosto }
\subsection*{ Champandongo }
\textbf{Ingredientes}
\begin{itemize}
    \item ¼ de carne molida de res
    \item ¼ de carne molida de puerco
    \item 200 gramos de nueces
    \item 200 gramos de almendras
    \item Una cebolla
    \item Un acitrón
    \item 2 jitomates
    \item azúcar
    \item ¼ de crema
    \item ¼ de queso manchego
    \item ¼ de mole
    \item comino\footnote{der Kreuzkümmel}
    \item caldo de pollo
    \item tortillas de maíz
    \item aceite
\end{itemize}

\section*{ Manera de hacerse: }
La cebolla~\sidenote{159}se pica finamente y se pone a freír junto con la carne
en un poco de aceite.
Mientras se fríe, se le agrega el comino molido y una cucharada de
azúcar.

Como de costumbre, Tita lloraba mientras picaba la cebolla. Tenía la
vista tan nublada que sin darse cuenta se cortó un dedo con el cuchillo.
Lanzó un grito de rabia y prosiguió como si nada con la preparación de
champandongo.

En estos momentos no se podía dar ni siquiera un segundo para atenderse
la herida. Hoy por la noche vendría John a pedir su mano y tenía que
prepararle una buena cena en tan solo media hora. Aquí está no le gusta
para cocinar con premura\footnote{die Dringlichkeit}.

Siempre le daba a los alimentos el tiempo adecuado y preciso para su
cocimiento y procuraba organizar sus actividades de tal manera que le
dieran la tranquilidad se necesita en la cocina para poder preparar
platillos suculentos\footnote{saftig} y en su punto exacto.
Ahora estaba tan atrasada\footnote{spät, nächträglich, rückständig}
que sus movimientos eran agitados y apremiantes\footnote{bedrückend, dringend, dringlich}
y por lo tanto propensos\footnote{anfällig} a provocar este
tipo de accidentes.

El principal motivo de su atraso\footnote{der Zeitrückstand, die Verspätung, der Rückstand}
era su adorable sobrina, que había nacido tres meses antes, al igual
que Tita, y una manera~\sidenote{160}prematura. A Rosaura le afectó tanto la
muerte de su madre que anticipó\footnote{etw. vorausschicken etw. vorgreifen, etw. vorwegnehmen}
el alumbramiento\footnote{die Entbindung} de su hija y
quedó imposibilitada para amamantarla. En esta ocasión Tita no pudo o
no quiso adoptar el papel de nodriza como en el caso de su sobrino,
es más, ni siquiera lo intentó, tal vez por la experiencia demoledora\footnote{erschütternd, verheerend}
que tuvo cuando la separaron del niño. Ahora sabía que no había que
establecer relaciones tan intensas con niños que no eran propios.

Prefirió en cambio proporcionarle a Esperanza la misma alimentación que
Nacha había utilizado con ella cuando era una indefensa criatura: atoles
y tés.

La bautizaron con el nombre de Esperanza a petición de Tita. Pedro
había insistido en que la niña llevaba el mismo nombre de Tita,
Josefita. Pero ella se negó terminantemente. No quería que el nombre
influyera en el destino de la niña. Bastante tenía ya con el hecho de
que al haber nacido, su madre tuviera una serie de alteraciones que
obligaron a Josefita a practicarle una operación de urgencia para
salvarle la vida, y quedara imposibilitada para volverse a embarazar.

John le había explicado a Tita que algunas veces, por causas anormales,
la placenta no solo se implanta en el útero, sino que echa raíces\footnote{die Wurzel, die Zahnwurzel}
dentro del mismo, por tanto, al momento en que el niño nace la placenta no
puede desprenderse. Está tan firmemente afianzada\footnote{für etw. bürgen, etw. bekräftigen, etw. festigen}
que si una persona inexperta trata de ayudar a la madre y jala
la placenta utilizando el cordón umbilical, se trae junto con ella al
útero completo. Entonces hay que operar la emergencia, extrayendo el
útero y dejando a esta persona~\sidenote{161}incapacitada para embarazarse por
el resto de su vida.

Rosaura fue intervenida quirúrgicamente, no por falta de experiencia de
John, sino porque no había de otra para poder desprenderle la
placenta. Por tanto Esperanza sería su única hija, las más pequeña y,
para acabarla de amolar, ¡mujer! Lo cual, dentro de la tradición
familiar, significaba que era la indicada para cuidar a su madre hasta
el fin de sus días. Tal vez Esperanza echó raíces en el vientre de su
madre porque sabía de antemano lo que le esperaba en este mundo. Tita
rezaba para que por la mente de Rosaura no se cruzara la idea de perpetuar\footnote{etw. verewigen, etw. aufrechterhalten, fortbestehen, sich verewigen}
la cruel tradición.
Para ayudar a que así fuera, no quiso darle ideas con el nombre y presionó\footnote{auf jmdn. Druck ausüben} día y noche
hasta lograr que la llamaran Esperanza.

Sin embargo había una serie de coincidencias que asociaban a esta niña
con un destino parecido a el de Tita, por ejemplo, por mera necesitad
pasaba la mayor parte del día en la cocina, pues su madre no la podía
atender y su tía solo le podía procurar esmero dentro de la cocina, así
que con tés y atoles crecía de lo más sana entre los al olores y los
sabores de este paradisíaco y cálido lugar.

A la que no le caía muy bien que digamos está costumbre era a Rosaura,
sentía que Tita le quitaba a la niña por demasiado tiempo por su lado y
en cuanto se recuperó por completo de la operación pidió que
inmediatamente después de que Esperanza tomara sus alimentos la
regresaran a su cuarto para dormirla junto a su cama, donde era su
lugar. Está disposición llegó demasiado tarde, pues la niña para~\sidenote{162}ese entonces ya se había acostumbrado a estar en la cocina y no fue tan
fácil sacarle de ella.

Lloraba muchísimo en cuanto sentía que se alejaba del calor de la
estufa, al grado que lo que Tita tenía que hacer era a llevarse a la
recámara el guisado que estuviera cocinando, para así lograr engañar a
la niña, que al oler y sentir de cerca el calor de la olla en la que
Tita cocinaba conciliaba el sueño. Tita regresaba entonces la enorme
olla a la cocina y proseguía con la elaboración de la comida.

Pero el día de hoy la niña se había lucido, es muy probable que
presintiera que su tía pensaba casarse e irse de rancho, Y que entonces
ella iba a quedar a la deriva\footnote{die Kursabweichung, die Abdrift},
pues no dejó de llorar en todo el día.
Tita subía y bajaba las escaleras llevando ollas con comida de un lado
a otro. Hasta que pasó lo que tenía que pasar:
\textit{tanto va el cántaro\footnote{der Krug, der Tonkrug} al agua hasta que no se rompe}.
Cuando bajaba por octava ocasión tropezó y
la olla con el mole para el champandongo rodó
escaleras abajo. Junto con ella se desperdigaron
cuatro horas de intenso trabajo picando y moliendo ingredientes.

Tita se sentó en un escalón con la cabeza entre las manos para tomar
aire. Se había levantado a las cinco de la mañana para que las carreras no
se apoderaran de ella y todo había sido en vano. Ahora tenía que
preparar nuevamente el mole.

Pedro no podía haber elegido peor momento para hablar con Tita, pero
aprovechando que la encontró en las escaleras, aparentemente tomando un
descanso, se le acercó con la intención de convencerla de que no se
casaba con John.

- Tita, quisiera decirle~\sidenote{163}que considero un lamentable error de su %
parte la idea que tiene de casarse con John. Aún está a tiempo y no %
cometer esa equivocación, ¡no acepte ese matrimonio, por favor! %
\\- Pedro, usted no es nadie para decirme lo que tengo que hacer o no. %
Cuando usted se casó y yo no le pedí que no lo hiciera, a pesar de que %
esa boda me destrozó. Usted hizo su vida, ¡ahorra déjeme hacer la mía en %
paz! %
\\- Precisamente por esa decisión que tomé y de la cual estoy %
completamente arrepentido, le pido que recapacite\footnote{etw. überdenken, an etw. zurückdenken}. %
Usted sabe muy bien cuál fue el motivo que me unió a su hermana, pero %
resultó un acto inútil que no funcionó, ahora pienso que lo mejor %
hubiera sido huir con usted. %
\\- Pues lo piensa demasiado tarde. Ahora ya no hay remedio. Y le suplico %
que nunca más en la vida me vuelve a molestar, ni se atreva a repetir lo %
que me acaba de decir, mi hermana lo podría escuchar y no tiene por qué %
haber otra persona infeliz en esta casa. ¡Con permiso! \ndots Ah, y le %
sugiero que para la próxima vez que se enamore, ¡no sea tan cobarde!\\

Tita, tomando la olla con furia, se encaminó hacia la cocina. Terminó
el mole entre masculleos\footnote{etw. murmeln, etw. nuscheln}
y aventones\footnote{die Mitnahme im Auto} de trastes y mientras
esté se cogía siguío con la preparación del champandongo.

Cuando la carne se empieza a dorar, se le agregan el jitomate picado
junto con el acitrón, las nueces y las almendras partidas en trozos
pequeños.

El calor del vapor de la olla sieconfundía con el que se desprendía el
cuerpo de Tita. El enojo que sentía por dentro~\sidenote{164}actuaba como a la levadura\footnote{die Hefe} con la masa de pan.

Lo sentía crecer atropelladamente, inundando hasta el último resquicio\footnote{die Ritze}
que su cuerpo podía contener y, como levadura en un traste diminuto, se desbordaba\footnote{überlaufen, überschwemmen} hacía el exterior,
saliendo en forma de vapor por los oídos, la nariz y todos los poros de
su cuerpo.

Este desmesurado\footnote{maßlos, übermäßig, unangemessen}
enojo era causado en una mínima parte por la discusión
con Pedro, en otra parte por los incidentes y el trabajo de la cocina y
en una gran parte de las palabras que Rosaura había pronunciado unos
días antes. Estaban reunidos en la recámara de su hermana, Tita, John
y Álex. John había llevado a su hijo a la visita médica, pues el niño
extrañaba mucho la presencia de Tita en su casa y la quería ver
nuevamente. El niño se asomó a la cuna para conocer a Esperanza y quedó
muy impresionado con la belleza de la niña. Y como todos los niños de
esa edad que no se andan con tapujos\footnote{die Vermummung, die Verheimlichung, die Verhüllung},
dijo en voz alta:
\\- Oye, papi, yo quiero casarme también, así como tú. Pero yo con este %
niñita.\\

Todos rieron por la graciosa ocurrencia pero cuando Rosaura le explicó a
Álex que eso no podía ser pues esa niñita estaba destinada a cuidarla
hasta el día de su muerte, Tita sintió que los cabellos se le erizaban.
Solo a Rosaura se le podía ocurrir semejante\footnote{solcher, derart}
horror, perpetuar una tradición por demás inhumana.

¡Ojalá que a Rosaura la boca se le hiciera chicharrón\footnote{knusprig gebratene Schweineschwarten, stark sonnengebräunter Mensch}!
Y que nunca hubiera dejado escapar esas repugnantes\footnote{abscheuerregend, Ekel erregend}. Malolientes\footnote{stinkend}, incoherentes,
pestilentes, indecentes\footnote{unanständig, ungebührlich, unziemlich}
y repelentes palabras. Más valía que se las hubiera tragado y guardado
en el fondo de sus extrañas hasta que se le pudrieran\footnote{faulen, vermodern}
y~\sidenote{165}agusanaran\footnote{madig werde, wurmstichig werden}.
Y ojalá que ella viviera lo suficiente como para impedir que su hermana
llevara a cabo tan nefastas\footnote{unglückbringend, unheilvoll}
intenciones.

En fin, no sabía por qué tenía que pensar en esas cosas tan
desagradables en estos momentos que deberían ser para ella los más
felices de su vida, ni sabía por qué estaba tan molesta. Tal vez Pedro
la había contagiado\footnote{jmdn. anstecken, sich übertragen}
de su mal humor. Desde que regresaron al rancho y se
enteró que Tita se pensaba casar con John andaba de un humor de los mil
demonios. Ni siquiera se le podía dirigir la palabra. Procuraba\footnote{versuchen, etwas zu tun} salirse
muy temprano y recorrer el rancho a galope en su caballo. Regresaba por
la noche justo a tiempo para la cena y se encerraba en su recámara
inmediatamente después.

Nadie se explicaba este comportamiento, algunos creían que era porque
le había afectado profundamente la idea de no volver a tener más hijos.
Por lo que fuera, pero tal parecía que la ira dominaba los pensamientos
y las acciones de todos en la casa. Tita literalmente estaba \glqq{}como agua
para chocolate\grqq{}. Se sentía de lo más irritable. Hasta el canturreo tan
querido de las palomas, que ya se habían reinstalado en el techo de la
casa y que el día de su regreso le habían proporcionado tanto placer, en
este momento la molestaba. Sentía que la cabeza le iba a estallar como roseta\footnote{das Röschen, der Puffmais} de maíz. Tratando de
impedirlo se la apretó fuertemente con las dos manos. Un tímido golpe
que sintió en el hombro la hizo reaccionar sobresaltada, con ganas de
golpear a quien fuera el que lo hizo, que de seguro venía a quitarle
más el tiempo.

Pero cuál no sería su sorpresa~\sidenote{166}al ver a Chencha frente a ella. La
misma Chencha de siempre, sonriente y feliz. Nunca en la vida le había
dado tanto gusta verla, ni siquiera cuando la había visitado en casa de
John. Como siempre, Chencha llegaba caído del cielo, en el momento en
que Tita más lo necesitaba.

Era asombroso observar lo respuesta que se encontraba Chencha, después
de haberla visto irse en el estado de angustia y desesperación en que lo
hizo.

Ni rastro quedaba del trauma que había sufrido. El hombre que había
logrado borrarlo estaba a su lado, luciendo una sincera y amplia
sonrisa. A leguas\footnote{die spanische Meile} se veía que se
trataba de un hombre honrado\footnote{ehrbar, grundehrlich, anständig}
y callado, bueno, eso quién sabe, porque en lo que pasaba era que Chencha
no le permitió abrir la boca más que para decirle a Tita: \glqq{}Jesús Martínez
para servirle a usted\grqq{}. Después Chencha, como siempre, acaparó\footnote{etw. aufkaufen, etw. hamstern, etw. beanspruchen}
por completa de plática y, rompiendo récord de velocidad, en solo dos
minutos logró poner a Tita al día en los acontecimientos de su vida.

Jesús había sido su primer novio y nunca la había olvidado. Los papás
de Chencha se habían opuesto terminantemente a esos amores y no de haber
sido porque Chencha regresó a su pueblo y él la volvió a ver, nunca
hubiera sabido dónde buscarla. Por supuesto no le importó que Chencha no
fuera virgen y se casó inmediatamente con ella. Regresaban juntos al
rancho con la idea de empezar una nueva vida ahora que Mamá Elena había
muerto, y pensaban tener muchos hijos y ser muy felices o los siglos de
los siglos \ndots
Chencha~\sidenote{167}se detuvo para tomar aire pues se estaba poniendo morada
y Tita aprovechó la interrupción para decirle, no tan rápido como ella,
pero casi, que estaba encantada de su regreso al rancho, que mañana
hablarían de la contratación\footnote{die Anstellung, die Neueinstellung}
de Jesús, que hoy venían a pedir su mano, que pronto se casaría, que aún
no terminaba la cena y le pidió que ella la hiciera para poderse dar un
calmante baño de agua helada y de esta manera esta presentable cuando
John llegara, que sería de un momento a otro.

Chencha prácticamente la echó de la cocina y de inmediato tomó el mando.
El champandongo lo podía hacer, según ella, con los ojos tapados y las
manos amarradas\footnote{etw. festbinden, etw. an etw. festmachen}.

Cuando la carne ya está cocida y seca, lo que procede es freír las
tortillas en aceite, no mucho para que no se endurezcan\footnote{erstarren, sich erhärten, sich verhärten}.
Después, en el traste\footnote{das Gerümpel, der Kram} que
vamos a meter al horno se pone primero una capa\footnote{der Belag, die Schicht, der Überwurf, der Umhang}
de crema para que no se pegue el platillo, encima una capa de tortilla,
sobre ellas una capa de picadillo y por último el mole, cubriéndolo con
el queso en rebanadas, y la crema. Se repite esta operación cuantas veces
sea necesario hasta rellenar el molde. Se mete al horno y se saca cuando el
queso ya se derritió y las tortillas se ablandaron\footnote{weich werden, etw. durchweichen, etw. erweichen}.
Se sirve acompañado de arroz y frijoles.

Qué tranquilidad le daba a Tita saber que Chencha estaba en la cocina.
Ahora solo se tenía que preocupar por su arreglo personal. Cruzó el
patio como ráfaga\footnote{die Windbö, die Bö} de viento y
se metió a bañar. Contaba con tan solo diez minutos para bañarse, vestirse,
perfumarse y peinarse adecuadamente. Tenía tal apuro\footnote{der Gram, der Kummer, unangenehme Lage}~\sidenote{168}que ni siquiera vio a Pedro, en el otro extremo del patio trasero, pateando\footnote{trampeln, (vor Wut) aufstampfen}
piedras.

Tita se despojó\footnote{jmdn. ausplündern, auf etw. verzichten, sich etw. entledigen, etw. abstreifen}
de sus ropas, se metió a la regadera y dejó que el agua
fría cayera sobre su cabeza. ¡Qué alivio sentía! Con los ojos cerrados
las sensaciones se agudizan, podía percibir cada gota agua fría
recorriéndole el cuerpo. Sentía los pezones
de sus senos ponerse duros como piedras al contacto con el agua.
Otro hilo de agua bajaba por su espalda y después caía como cascada en la
curva de sus redondos y protuberantes\footnote{hervorragend} glúteos\footnote{das Gesäß}, recorriendo sus firmes piernas
hasta los pies.
Poco a poco se le fue pasando el mal humor, y el dolor de cabeza
desapareció. De pronto empezó a sentir que el agua se entibiaba\footnote{sich abkühlen, lau werden, etw. lau machen, etw. abkühlen}
y se ponía cada vez más caliente hasta empezar a quemarle la piel. Esto
pasaba algunas veces en época de calor cuando el agua del tinaco había
sido calentada todo el día por los poderosos rayos de sol, pero no ahora
que en primera no era verano y segunda empezaba anochecer. Alarmada
abrió sus ojos, temerosa de que nuevamente se fuera a incendiar el
cuarto de baño, y lo que descubrió fue la figura de Pedro del otro lado
de los tablones, observándola detenidamente.

Los ojos de Pedro brillaban de una manera que era imposible
descubrirlos en la penumbra, así como dos insignificantes\footnote{geringfügig, gleichgültig, unscheinbar}
gotas de rocío\footnote{der Tau} no podían pasar inadvertidas\footnote{unauffällig, unbeachtet, unbemerkt}
escondidas entre la maleza\footnote{das Unkraut, das Unterholz, das Buschwerk},
al recibir los primeros rayos del sol. ¡Maldita mirada de Pedro!
¡Y maldito carpintero que había reconstruido el cuarto de baño exactamente
igual al anterior~\sidenote{169}, o sea con separaciones entre uno y otro tablón!
Cuando vio que Pedro se acercaba a ella, con libidinosas intenciones
en los ojos, salió corriendo del cuarto vistiéndose atropelladamente.
Con gran apuro llegó a su recámara y se encerró.

Apenas le dio tiempo de terminar con su arreglo cuando Chencha le fue a
anunciar que John acaba de llegar y la esperaba en la sala.

No pudo acudir de inmediato a recibirlos, pues aún le faltaba poner la
mesa. Antes de poner el mantel hay que cubrir la mesa con un tapete,
para evitar el ruido que hacen las copas y vajilla al chocar contra la
misma. Tiene que ser bayeta\footnote{der Aufwischlappen, der Putzlappen}
blanca para así realizar la blancura del mantel. Tita lo deslizaba
suavemente sobre la enorme mesa para veinte personas, que solo usaban en
ocasión como esta. Trataba de no hacer ruido, ni siquiera al respirar
para escuchar el contenido de la plástica que sostenían en la sala Rosaura,
Pedro y John. La sala y el comedor estaban separados por un largo
pasillo, así que solo llegaba a los oídos de Tita el murmullo de las
varoniles voces de Pedro y John, sin embargo
alcanzaba a percibir en ellas cierto tono de discusión. Antes de esperar
que las cosas llegaran a mayores, colocó rápidamente en el orden debido
los platos, los cubiertos de plata, las copas, los saleros y los portacuchillos\footnote{der Messerblock}. Enseguida
puso las bujías\footnote{die Kerze} bajo los calentadores para los
platos principales, entrada e intermedio y los dejó listos sobre el aparador\footnote{die Anrichte, das Sideboard}. Corrió a
la cocina por el vino de Burdeos que había dejado en baño María. Los
vinos de Burdeos se sacan de la bodega con varias horas
de anticipación y se ponen en un lugar caliente~\sidenote{170}para que un suave
calor desarrolle su aroma, pero como a Tita se le había olvidado sacarlo
a tiempo forzó\footnote{etw. erzwingen, etw. forcieren} el procedimiento\footnote{das Verfahren, die Methode}
artificialmente. Lo único que le faltaba era a poner en el centro de
la mesa una canastilla de bronce dorado con las flores, pero como
estas se deben en colocar unos momentos antes de pasar a la mesa para
que conserven su frescura natural, encargó a Chencha este trabajo, y,
apresuradamente, tanto cómo almidonado vestido se lo permitía, se dirigió la sala.

La primera escena que presenció al abrir la puerta fue la colorada
discusión entre Petro y John sobre la situación política del país.
Parecían que los dos habían olvidado las más elementales reglas de urbanidad\footnote{die Höflichkeit}, qué dicen que en una
reunión social no hay que sacar a colación\footnote{etw. anführen, etw. zur Sprache bringen}
cuestiones sobre personalidades, sobre temas tristes o hechos
infortunados, sobre religión o sobre política. La entrada Tita suspendió
la discusión y los forzó a tratar de reiniciar la plástica en un tono
más amigable\footnote{freundlich, gütlich}.

En un ambiente tenso\footnote{straff, spannend, angespannt},
John dio paso a la petición de mano. Pedro, como
el hombre de la casa, dio su aprobación de una manera hosca\footnote{mürrisch}.
Y se empezaron a establecer los detalles de la misma. Cuando trataban de
fijar la fecha de la boda, Tita se enteró de los deseos de John posponerla\footnote{etw. aufschieben, etw. verschieben, etw. zurückstellen}
un poco para así poder viajar al norte de Estados Unidos
atraer la única tía que le quedaba y que quiera estuviera presente en la
ceremonia. Esto representaba un grave problema para Tita: ella deseaba
irse lo más pronto posible del rancho y de la cercanía de Pedro.

El compromiso\footnote{die Verabredung, die Verpflichtung, die Verlobung}
quedó formalizado~\sidenote{171}cuando John le hizo entrega a Tita
de un hermoso anillo de brillantes. Tita observó largamente cómo lucía
en su mano. Los destellos
que se desprendían de él la hicieron recordar el fulgor\footnote{das Blitzen, das Funkeln}
en los ojos de Pedro momentos antes, cuando la miraba desnuda,
y vino a su mente un poema otomí que Nacha le había enseñado de niña:

\begin{quote}
\itshape
En la gota de rocío brilla el sol \\
la gota de rocío se seca \\
en mis ojos, los míos, brillas tú \\
yo, yo vivo \ndots
\end{quote}

Rosaura se enterneció\footnote{weich werden, gerührt werden}
al ver en los ojos de su hermana lágrimas que
ella interpretó como la felicidad y se sintió un poco aliviada de la
culpa que algunas veces la atormentaba por haberse casado con el novio
de Tita. Entonces, muy entusiasmada, les repartió a todos copas con
champaña y los invitó a brindar por la felicidad de los novios. Al
hacerlo, los cuatro reunidos en el centro de la sala, Pedro golpeó su copa
con la de los demás con tal fuerza que la rompió en mil pedazos y el
líquido de las otras los salpicó en el rostro y la ropa.

Entre el desconcierto reinante\footnote{regierend} fue una bendición que en ese momento Chencha
aparecería y pronunciara las mágicas palabras de \glqq{}la cena está servida\grqq{}.
Este anuncio les proporcionó a los presentes la serenidad y el
espíritu que el momento ameritaba y que estuvieron a punto de perder.
Cuando se habla de comer, hecho por demás importante, solo los necios y
los enfermos no le dan el interés que merece. Y como este no era el
caso, mostrando buen humor todos se dirigieron al comedor.

Durante la cena todo fue más fácil, gracias a las graciosas
intervenciones de Chencha~\sidenote{172}mientras servía. La comida no fue tan
deleitosa como en otras ocasiones, tal vez porque el mal humor acompañó a
Tita mientras la preparaba, pero no se podía decir tampoco que estuviera
desagradable. El champandongo es un platillo de un sabor tan refinado
que ningún mal temperamento puede ponerse a su altura y a alterarle el
gusto. Al terminar, Tita acompañó a John a la puerta y ahí se dieron
un largo beso como despedida. Al día siguiente John pensaba salir de
viaje, para estar de vuelta lo más pronto posible.

De regreso en la cocina, Tita mandó a Chencha a limpiar la habitación
y el colchón donde desde ahora viviría con Jesús, su esposo, no sin
antes agradecerle su gran ayuda. Era necesario que antes de meterse en
la cama se cercioraran\footnote{sich Gewissheit über etw. verschaffen, sich von etw. überzeugen}
que no se iban a encontrar con la indeseable
presencia deel chinche::la chinche|die Wanze::der Reißnagel, die Reißzwecke:Subst\footnote{chinches} en el cuarto.
La última sirvienta que durmió ahí la
había dejado infestada\footnote{jmdn./etw. mit etw. infizieren, etw. befallen}
de estos animalejos\footnote{das Tierchen} y Tita no lo había podido
desinfectar por el intenso trabajo que se le había venido encima con el
nacimiento de la niña de Rosaura.

El mejor método para erradicarlas es mezclar un vaso de espíritu de
vino, media onza esencia de trementina\footnote{das Terpentin}
y media de alcanfor\footnote{der Kampfer} en polvo.
Esta preparación se unta\footnote{etw. einreiben} en los sitios
donde hay chines y las hace desaparecer por completo.

Tita, después de recoger la cocina, empezó a guardar trastes y ollas en
su lugar. Aún no tenía sueño y mejor aprovechaba el tiempo~\sidenote{173}en eso
que dando vueltas en la cama. Experimentaba una serie de sentimientos
encontrados y la mejor manera de ordenarlos dentro de su cabeza era
poniendo primero en orden la cocina. Tomó un gran cazuela de barro y la
llevó a guardar al ahora cuarto de los triques\footnote{der Krimskrams},
antes cuarto obscuro. A la muerte de Mamá Elena vieron que ya nadie
lo pensaba utilizar como lugar para bañarse, pues todos preferían hacerlo
en la regadera, y tratando de darle alguna utilidad\footnote{der Nutzen, die Verwendungsmöglichkeit}
lo convirtieron en el cuarto de los trebejos\footnote{das Gerät, das Handwerkzeug}.

En una mano llevaba la cazuela y en la otra un quinqué. Entró al cuarto
cuidando de no tropezar con la gran cantidad de objetos que estaban en
el camino del sitio donde se guardaban las ollas de cocina que no se
utilizaban frecuentemente. La luz del quinqué le ayudaba bastante, pero
no lo suficiente como para alumbrarle las espaldas por donde
silenciosamente se deslizó una sombra y cerró la puerta del cuarto.

Al sentir una presencia extraña, Tita giró sobre sí misma y la luz delineó\footnote{etw. zeichnen, etw. skizzieren} claramente
la figura de Pedro poniendo una tranca\footnote{der Knüppel, der Türriegel}
en la puerta.
\\- ¡Pedro! ¿Qué hace aquí? \\
\rule{1em}{0pt}Pedro, sin responderle, se acercó a ella, apagó la luz del quinqué, la
jaló hacia donde estaba la cama de latón\footnote{das Messing} que
alguna vez perteneció a Gertrudis, su hermana, y tirándola sobre la cama,
la hizo perder su virginidad y conocer el verdadero amor.

Rosaura, en su recámara, trataba de dormir a su hija que lloraba
desenfrenadamente. La paseaba por todo el cuarto, sin ningún resultado.
Al cruzar por la ventana~\sidenote{174}vio salir del cuarto oscuro un resplandor
extraño. Volutas\footnote{das Schneckengehäuse, die Spirale}
fosforescentes se elevaban hacia el cielo como
delicadas luces de bengala\footnote{die Leuchtrakete, die Wunderkerze}.
Por más gritos de alarma que dio llamando a Tita y a Pedro para que lo observaran,
no tuvo respuesta más que de Chencha, que había ido a buscar un juego\footnote{das Spiel, die Garnitur, das Set}
de sábanas. Al presenciar el singular fenómeno, Chencha por primera vez en
su vida enmudeció\footnote{verstummen, verschweigen} de
sorpresa, ni un solo sonido escapaba de sus labios. Hasta Esperanza, que
no perdía detalle, dejó de llorar. Chencha se arrodilló y persignándose\footnote{jmdn./etw. segnen, sich bekreuzigen}
se puso a orar.

- ¡Virgen Santísima que’stás en los cielos, recoge el alma de mi siñora %
Elene pá ‘que deje de vagar en las tiñeblas del purgatorio! %
\\- ¿Qué dices Chencha, de qué hablas? %
\\- ¡Pos de que’a de ser, no ve que se trata del fantasma de la dijunta! %
¡La probe algo de andar pagando! !Yo por si las dudas ni de chiste me %
güelgo a’cercar por a’i! %
\\- Ni yo. \\
\rule{1em}{0pt}¡Si la pobre Mamá Elena supiera que aún después de muerte su presencia
seguía causando temor y que ese miedo a encontrarse con ella les
proporcionaba a Tita y a Pedro la oportunidad ideal para profanar\footnote{etw. entweihen, etw. schänden} impunemente\footnote{ungeahndet, straflos} su lugar preferido,
al revolcarse voluptuosamente sobre la cama de Gertrudis, se volvería a
morir cien veces!

\clearpage

\section*{ Capítulo 9 }
\subsection*{ Septiembre }
\subsection*{ Chocolate y rosca de reyes }~\sidenote{175}\textbf{Ingredientes}
\begin{itemize}
    \item 2 libras de cacao Soconusco
    \item 2 libras de cacao maracibo
    \item 2 libras de cacao Caracas
    \item Azúcar (entre 4 y 6 libras según el gusto)
\end{itemize}

\section*{ Manera de hacerse: }
La primera operación~\sidenote{179}es tostar el cacao. Para hacerlo es
conveniente utilizar una charola de hojalata\footnote{das Blech}
en vez del comal, pues el aceite que se desprende de los granos se pierde
entre los poros del comal.
Es importantísimo poner cuidado en este tipo de indicaciones\footnote{das Zeichen, der Fingerzeig, der Hinweis},
pues la bondad\footnote{die Güte, die Gutherzigkeit} del
chocolate depende de tres cosas, a saber: de que cacao que se emplee\footnote{etw. benutzen, etw. anwenden, jmdn. beschäftigen}
esté sano y no averiado\footnote{etw. beschädigen, havarieren, fehlerhaft},
de que se mezclen en su fabricación distintas clases de cacao y, por
último, de su grado de tueste\footnote{das Rösten}.
El grado de tueste aconsejable es el del momento en que el cacao
comienza a despedir\footnote{jmdn. verabschieden, jmdn. entlassen, jmdn. kündigen, etw. aussenden, etw. ausstrahlen, etw. ausströmen}
su aceite. Si se retira antes, aparte de presentar
un aspecto descolorido y desagradable, lo hará indigesto\footnote{schwer im Magen liegen, unverdaulich}.
Por el contrario, si se deja más tiempo sobre el fuego, el grana quedará
quemado en gran parte y contaminará de acrimonia\footnote{die Schärfe}
y aspereza\footnote{die Herbe, die Rauheit} al chocolate.

Tita extrajo solo media cucharadita de esta aceite para mezclarlo con
aceite de almendras dulces y preparar una excelente pomada para los
labios. En invierno se le partían\footnote{abreisen, losgehen, losbrechen, zerbrechen} invariablemente\footnote{unveränderlich,, unverändert},
tomara las precauciones que tomara. Cuando era niña esto le causaba malestar, pues
cada vez que se reía~\sidenote{180}, se le abrían sus carnosos labios y le
sangraban produciéndole un intenso dolor. Con el tiempo lo fue tomando
con resignación. Y como ahora no tenía muchas razones que digamos para
reír, no le preocupaba en lo más mínimo. Podía esperar tranquilamente
hasta la llegada de la primavera para que desaparecieran las grietas\footnote{der Spalt, der Riss}.
El único interés que la movía a preparar la pomada era que por la noche
vendrían a la casa algunos invitados a partir la rosca de reyes.

Por vanidad\footnote{die Eitelkeit, die Selbstzufriedenheit},
no porque pensara reírse mucho, quería tener los labios
suaves y brillantes durante la velada\footnote{der Abend, die Nachtwache, die Abendveranstaltung}.
La sospecha de estar embarazada no la hacía sentirse como para tener
la risa a flor de labio. Nunca pensó en esta posibilidad al consumar
su amor con Pedro. Aún no se lo comunicaba a él. Esta noche pensaba hacerlo,
pero no sabía cómo. Qué actitud tomaría Pedro y cuál sería la solución
a este gran problema, lo ignoraba por completo.

Prefería tratar de no atormentarse\footnote{sich quälen} más
y procurar desviar los pensamientos de su mente hacías cosas más triviales
como la preparación de una buena pomada. Para esto no hay como la crema
de cacao. Pero antes de ponerse a elaborarla era necesario que dejara
listo el chocolate.

Cuando el cacao ya está tostado como se indicó, se limpia utilizando un
cedazo para separar la cáscara del grano. Debajo del metate donde se ha
de moler, se pone un cajeta con buena lumbre y cuando ya está caliente
el metate, se procede a moler el grano.Se mezcla entonces con el azúcar, machacándolo\footnote{auf etw. einschlagen, etw. zerstampfen, etw. zerquetschen}
con un mazo\footnote{der Stampfer, der Stößel} y moliendo las dos
cosas juntas. Enseguida se divide la masa en~\sidenote{181}trozos.

Con las manos se moldean\footnote{etw. ausformen} las tablillas,
redondas o alargadas, según el gusto, y se ponen a orear\footnote{etw. auslüften}.
Con la punta de un cuchillo se le pueden señalar las divisiones que se
deseen. Mientras Tita daba forma a las tablillas, añoró con
tristeza los días de reyes de su infancia, en los
que no tenía problemas tan serios. Su mayor preocupación en esa época
era que los Santos Reyes nunca le traían lo que ella pedía, sino lo que
Mamá Elena pensaba que sería lo más adecuado para ella. Hasta hacía
algunos años se había enterado de que la causa por la que en una sola
ocasión se ??? recibió el regalo esperado fue porque Nacha se pasó algún
tiempo ahorrando de su salario para comprarle un \glqq{}cinito\grqq{} que había
visto en el aparador
de una tienda. Le llamaban cinito, por ser un
aparato que proyectaba imágenes en la pared utilizando un quinqué
petróleo como fuente de luz, dando un efecto parecido al del cine, pero
su nombre verdadero era el de \glqq{}zootropo\grqq{}. Qué enorme felicidad le
proporcionó verlo junto a su zapato, al despertarse por la mañana.
Cuántas tardes gozaron ella y sus hermanas viendo las imágenes en
secuencia que venían dibujadas en tiras\footnote{das Band, der Streifen}
de cristal, y que representaban diferentes situaciones de lo más divertidas.
Qué lejos le parecían ahora esos días de felicidad, cuando Nacha estaba
a su lado. ¡Nacha! Extrañaba a su olor o sopa de fideo, a chilaquiles,
champurrado, a salsa de molcajete, a pan con natas, a tiempos pasados.
¡Por siempre serían insuperable\footnote{unübertrefflich, unüberwindlich}
su sazón\footnote{die Reife, die Würze}, sus atoles, sus tés,
su risa, sus chiqueadores\footnote{etw. verwöhnen} en
las sienes, su manera de trenzarle
el pelo, de arroparla\footnote{jmdn./etw. bedecken, sich zudecken}
por las noches, de cuidarla en sus enfermedades, de cocinarle sus
antojos~\sidenote{182}, de batir el chocolate! ¡Si pudiera volver un solo
momento a aquella época para traerse de regreso un poco de alegría
de esos instantes y poder preparar la rosca de reyes con el mismo
entusiasmo que entonces! Si pudiera comerla más tarde con sus hermanas
como en los viejos tiempos, entre chanzas
y bromas, cuando aún no tenían que disputarse\footnote{miteinander um etw. wetteifern}
Rosaura y ella el amor de un hombre, como cuando ella aún ignoraba que
le estaba negado el matrimonio en esta vida, como cuando Gertrudis no
sabía que huiría de la casa y trabajaría en un burdel, como cuando al
sacarse el muñeco de la rosca se tenía la esperanza de que lo que se
deseara se cumpliría milagrosamente al pie de la letra. La vida le había
enseñado que la cosa no era tan fácil, que son pocos los que pasándose
de listos logran realizar sus deseos a costa de lo que sea, y que obtener el
derecho de determinar su propia vida le iba a a costar más trabajo del
que se imaginaba. Esta lucha la tendría que dar sola, y esto le pesaba.
¡Si al menos estuviera a su lado Gertrudis! Pero parecía más probable
que un muerto volviera a la vida a que Gertrudis regresara a la casa.

Nunca habían vuelto a recibir noticias de ella, desde que Nicolás le
había hecho entrega de su ropa, en el burdel donde había ido a caer. En
fin, dejando orear al lado de sus recuerdos las tablillas de chocolate
que acababa de terminar, se dispuso a preparar la rosca de reyes.

\vskip1em

\textbf{Ingredientes}
\begin{itemize}
    \item 30 gramos de levadura~\sidenote{183}fresca
    \item un kilo y un ¼ de harina
    \item 8 huevos
    \item una cucharada de sal
    \item 2 cucharadas de azahar\footnote{die Orangenblüte}
    \item una y un ½ tazas de leche
    \item 300 gramos de azúcar
    \item 800 gramos de mantequilla
    \item 250 gramos de frutas cubiertas
    \item un muñeco de porcelana
\end{itemize}

\subsection*{ Manera de hacerse: }
\FIN

e con las manos o utilizando un tenedor se desbarata la levadura en un
cuarto de kilo de harina, agregándole poco a poco media taza de leche
tibia. Cuando están bien incorporados los ingredientes se amasan un poco
y se dejan reposar en forma de bola, hasta que la masa crezca el doble
de su tamaño.

Justo cuando Tita ponía la masa a reposar, Rosaura hizo su aparición en
la cocina. Venía a pedirle su ayuda para poder llevar a cabo la dieta
que Javi le había recetado. Desde hacía unas semanas tenía graves
problemas digestivos, sufría de flato y~\sidenote{184}mal aliento. Rosaura se
sintió tan apenada por estos trastornos que inclusive tuvo que tomar la
decisión de que Pedro y ella durmieran en recámaras separadas. De esta
manera aminoraba un poco sufrimiento al poder desalojar ventosidades a
su antojo. Javi le había recomendado abstenerse de alimentos tales como
raíces y legumbres y realizar un activo trabajo corporal. Esto último se
le dificultaba por su excesiva gordura. No se explicaba por qué desde
que regresó nuevamente al rancho había empezado a engordar tanto, pues
seguía comiendo lo mismo de siempre. El caso es que le costaba un
trabajo enorme poner en movimiento su voluminoso y gelatinoso cuerpo.
Todos estos males le estaban acarreando infinidad de problemas, pero el
más grave era que Pedro se estaba distanciando de ella cada día más. No
lo culpaba: ni idea misma soportaba su pestífero vaho . Ya no podía más.

Era la primera vez que Rosaura se abría la capa con Tita y trataba estos
temas con ella. Inclusive le confesó que no se le había acercado antes
por los celos que le tenía. Pensaba que entre ella y Pedro había una
relación amorosa, latente, escondida bajo las apariencias. Pero ahora
que veía la enamorada que estaba de Javier, y lo cercano de su
matrimonio con él, Se había dado cuenta de lo absurdo que era seguir
guardando este tipo de recelos. Confiaba en que aún era tiempo para que
entre ellas surgiera una buena comunicación. ¡la verdad, la relación
Rosaura- quita hasta ahora había sido como la del agua en aceite
hirviendo! Con lágrimas en los ojos le rogó que por favor no le guardara
rencor por haberse casado con~\sidenote{185}Pedro. Y le pidió su consejo para
recuperarlo. ¡como si ella estuviera para darle ese tipo de consejos!
Con pena Rosaura le comentó que Pedro tenía muchos meses de no
acercársele con intenciones amorosas. Prácticamente la rehuía. Esto no
la preocupaba mucho pues Pedro nunca había sido muy dado a los excesos
sexuales. Pero últimamente no solo eso, sino que detectaba en sus
actitudes un abierto rechazo a su persona.

Es más, podía precisar exactamente desde cuándo, pues lo recordaba
perfectamente. Fue la noche en que el fantasma de Mamá Elena había
empezado a aparecer. Ella estaba despierta, esperando que Pedro
regresaba de un paseo que había salido a dar. Cuando regresó, casi no le
prestó atención a su historia del fantasma, estaba como ausente. Durante
la noche ella había tratado de abrazarlo, pero él o estaba muy dormido o
fingió estarlo, pues no reaccionó a sus insinuaciones. Más tarde lo
había escuchado llorar quedamente y ella a su vez había fingido no
oírlo.

Sentía que su gordura, su flato y su mal aliento definitivamente
estaban alejando a Pedro de su lado cada día más y no le veía solución.
Le pedía por lo tanto su ayuda. La necesitaba como nunca y no tenía a
nadie más a quien recorrir. Su situación cada día era más grave. No
sabría cómo reaccionar al \glqq{}qué dirán\grqq{} si Pedro la abandonaba, no lo
resistiría. El único consuelo que le quedaba era que al menos tenía a su
hija esperanza, ella tenía la obligación de estar a su lado para
siempre.

Hasta ese momento todo iba muy bien, las primeras palabras de Rosaura
habían causado estragos en la conciencia de Tita, claro~\sidenote{186}en cuanto
escuchó por segunda ocasión cuál sería el destino de esperanza tuvo que
hacer un soberano esfuerzo por no gritarle a su hermana que esta idea
era la más aberrante que había escuchado en toda su vida.

No podía iniciar en estos momentos una discusión entre ellas que diera
al traste con la buena voluntad que sentía de compensar a Rosaura del
daño que le estaba causando. Así que en lugar de externar sus
pensamientos, le prometió a su hermana prepararle una dieta especial
para ayudarla a bajar de peso. Y amablemente le proporcionó una receta
de familia contra el mal aliento: \glqq{}el mal aliento tiene su origen en el
estómago y son varias las causas que contribuyen a ello. Para hacerlo
desaparecer debe principiarse por gargarismos de agua salada,
sorbiéndola al propio tiempo por las narices, mezclasa con algunas gotas
de vinagre de alconfor pulverizado . Paralelamente hay que masticar
continuamente hojas de menta. El plan propuesto, seguido con constancia,
es capaz por sí solo de purificar el aliento más pestilente\grqq{}.

Rosaura le agradeció infinitamente Su ayuda y rápidamente salió a la
huerta a recoger hojas de menta, no sin antes suplicarle absoluta
discreción a este dedicado asunto. El rostro de Rosaura reflejaba un
gran alivio. En cambio Tita estaba destruida. ¡qué era lo que había
hecho! ¿cómo resarcir el daño a Rosaura, a Pedro, a ella misma, a
Javier? ¿con qué cara lo iba a recibir dentro de unos días, cuando
regresara de su viaje? Javi, la persona a quien solo tenía que
agradecer; Javier, el que la había vuelto a la cordura ; Javier, el que
le había mostrado el camino a la libertad.
Javi,~\sidenote{187}la paz, la serenidad, la razón. ¡verdaderamente él no se
merecía esto! ¿Qué decirle, qué hacer? por lo pronto lo mejor era que
continuara preparando la rosca de reyes, pues la masa con levadura que
había dejado reposando mientras plasticaba con Rosaura ya estaba lista
para el paso siguiente.

Con el kilo de harina se forma una fuente sobre la mesa. En el centro
se ponen todos los ingredientes y se van amasando empezando por los del
centro y tomando poco a poco harina de la fuente, hasta que se incorpora
toda. Cuando la masa que contiene la levadura ha subido al doble de su
tamaño, se mezcla con esta otra masa, integrándolas perfectamente, hasta
el punto en que se desprendan de las manos con toda facilidad. Con una
raspa se quita la masa que se va quedando pegada en la mesa, para
integrarla también. Entonces se vacía la masa en un recipiente hondo,
engrasado. Se tapa con una servilleta Y se espera a que suba nuevamente
al doble de su tamaño. Hay que tener en cuenta que la masa tarda
aproximadamente 2 horas en duplicar su tamaño y es necesario que lo haga
tres veces, antes de poder meterla al horno.

Cuando Tita estaba cubriendo con una servilleta la vasija donde puso a
reposar la masa, una fuerte ráfaga de viento azotó la puerta de la
cocina abriéndola de par en par y permitiendo que un frío el aro la
invadiera. La servilleta voló por los aires y un gélido estremecimiento
recorrió la espalda de Tita. Giró su cuerpo y asombrada quedó frente a
frente con Mamá Elena, que la miraba duramente.


- Te dije muchas veces~\sidenote{188}que no te acercaras a Pedro. ¿por qué lo %
hiciste? %
\\- \ndots Yo lo intenté, mami $\ndots$ , pero $\ndots$  %
\\- ¡pero nada! ¡lo que has hecho no tiene nombre! ¡te has olvidado de lo %
que es la moral, respeto, las buenas costumbres! No vales nada, eres una %
cualquiera que no se respeta ni a sí misma. ¡has enlodado el nombre de %
toda mi familia, desde el de mis antepasados hasta de esa maldita %
criatura que guardas en las entrañas! %
\\- ¡no! ¡mi hijo no está malito! %
\\- ¡si lo está! ¡lo maldigo yo! ¡a él y a tí, para siempre! %
\\- No, por favor.\\

La entrada de Chencha en la cocina hizo Mamá Elena diera media vuelta y
saliera por la misma puerta por donde había entrado.
\\- Cierra la puerta, niña, ¿no ves el frío que hace? últimamente te veo %
muy despatoladota. ¿Pos qué tráis pues’n?\\

Nada. No le pasaba nada más que tener un mes de atraso en su
menstruación, la sospecha estar embarazada; tener que decírselo a Javi
en cuanto regresara para casarse con ella, tener que cancelar ese
matrimonio, tener que abandonar ese rancho si es que quería tener a su
hijo sin problemas, tener que renunciar para siempre a Pedro, pues no
podía hacerle más daño a Rosaura.

¡solo eso le pasaba! Pero no se lo podía decir. De hacerlo, con lo
chismosa que era Chencha, al otro día todo el pueblo lo sabría. Prefirió
de darle ninguna respuesta y cambiarle el tema sin más ni más, tal y
como Chencha~\sidenote{189}se lo hacía a ella cuándo es la sorprendía en alguna
falta.
\\- ¡qué barbaridad! La masa ya está desbordándose. %
Déjame termina la rosca, o nos va a caer la noche encima sin haber %
terminado.\\

Aún no se estaba desbordando la masa de la vasija donde la había puesta
a reposar, pero era el pretexto ideal para distraer la atención de
Chencha hacía otro asunto.

Cuando la masa ya doblado su tamaño por segunda vez, se vacía sobre la
mesa y se hace un tira con ella. En medio se le ponen, si se desea,
algunas frutas cubiertas en trozos. Si no, solamente el muñeco de
porcelana, al azar. Se enrolla la tira metiendo una punta en la otra. Se
pone sobre una lámina engrasada y enharinada con la unión hacia abajo.
Se le da la forma de rosca, dejando bastante espacio entre la misma y la
orilla de la lámina, pues todavía va a doblar su tamaño una vez más.
Mientras tanto se enciende el horno para mantener una temperatura
agradable en la cocina, hasta que termine de esponjarse la masa.

Antes de introducir el muñeco de porcelana en la rosca, quita lo
observó detenidamente. Según la tradición, la noche del 6 de enero se
parte la rosca y la persona que se saca el muñeco que viene escondido
dentro de ella queda obligada a celebrar una fiesta el 2 de febrero, Día
de la Candelaria, cuando hay que levantar del nacimiento al Niño Jesús .
Desde que eran muy niñas esta tradición se había convertido en una
especie de competencia entre ella y sus hermanas. Se consideraba muy
afortunada a la que tenía la suerte de quedarse con el muñeco. Por la
noche podría pedir~\sidenote{190}un deseo apretándolo fuertemente con las dos
manos.

Observando detenidamente las delicadas formas del muñeco, pensaba lo
fácil que era desear cosas durante la niñez. Entonces no hay imposibles.
Cuando una crece se da cuenta de todo lo que no se puede desear porque
es algo prohibido, precaminoso. Indecente.

Pero ¿qué es la decencia? ¿Negar todo lo que uno quiere verdaderamente?
ojalá que nunca hubiera crecido, ni conocido a Pedro, ni tuviera que
desear no estar embarazada de él. Ojalá que su madre dejara de
atormentarla, de toparse con ella por todos los rincones y de gritarle
de lo indigno de su proceder . ¡Ojalá que esperanza se casara, sin que
Rosaura lo pudiera impedir, y nunca conocería de estas angustias y
dolores! ¡ojalá que esta niña tuviera la fuerza que había tenido
Gertrudis para huir de la casa, en caso de ser necesario! Ojalá que
Gertrudis regresara a casa, para darle a Tita el apoyo Que tanto
necesitaba en estos momentos! Pidiendo estos deseos introdujo el muñeco
en la rosca y la dejó sobre la mesa, para que siguiera aumentando su
tamaño.

Cuando la masa dobla su tamaño por tercera vez, se decora con las
frutas cubiertas, se barniza con huevo batido y se le pone el azúcar. Se
mete al horno por 20 minutos y después se deja enfriar.

Cuando la rosca estuvo lista, quita le pidió a Pedro que la ayudara a
llevarla a la mesa. Le hubiera podido pedir ayuda a quien fuera, pero
necesitaba hablar con él en privado.
\\- Pedro, necesito hablarle a solas. %
\\- Es muy fácil, ¿por qué~\sidenote{191}no va al cuarto obscuro? ahí no podemos %
hacer sin que nadie nos moleste. Llevo muchos días esperando que vaya. %
\\- Lo que tengo que decirle expresamente sobre esas visitas.\\



La entrada de Chencha interrumpió la conversación, para informar que los
hay Lobo acababan de llegar a la fiesta y solo estaban esperando por
ellos para partir la rosca. Entonces a Tita y a Pedro no les quedó otra
que suspender la conversación y llevar la rosca al comedor, donde era
esperada con ansiedad. Cuando iban atravesando el pasillo, Quita vio a
su madre parada junto a la puerta del comedor, lanzándole una mirada de
furia . Quita se paralizó. El Pulque empezó a ladrarle a Mamá Elena, que
caminaba amenazadoramente hacía Tita. El perro tenía el pelo del lomo
erizado por el miedo y caminaba defensivamente, hacia atrás. Su
aturdimiento hizo que metiera un pata trasera dentro de la escupidera de
latón que se encontraba al final del pasillo, junto al helecho, y que al
tratar de salir corriendo la azotara contra el piso, dejando regado por
todos lados contenido de la misma.
El escándalo que provocó llamó la atención de los 12 invitados, que ya
se encontraban reunidos En la sala. Se asomaron al pasillo muy alarmados
y Pedro les tuvo que explicar que el pulque, tal vez por la vejez,
últimamente hacía este tipo de cosas inexplicables, pero que todo estaba
bajo control. Sin embargo Paquita Lobo se dio cuenta de que Tita estaba
a punto de desmayo. Pidió que alguien más le ayudaba a Pedro a llevar la
rosca hay comedor~\sidenote{192}, pues ella veía a Tita muy indispuesta. La tomó
del brazo Y la llevó a la sala. Le dieron a oler sales y después de un
momento se recuperó por completo. Entonces decidieron pasar al comedor.
Antes de salir, Paquita le tuvo a Tita un segundo y le preguntó:
\\- ¿ya te sientes bien? aún te noto medio mareada, ¡tienes una mirada! %
Que de no ser porque yo sé perfectamente que eres una muchacha decente %
juraría que estás embarazada.\\

Quita, riendo y tratando de no darle importancia, le respondió:
\\- ¿embarazada? ¡solo a usted se le ocurre! ¿y qué tiene que ver la %
mirada con eso? %
\\- Yo puedo ver en los ojos de una mujer inmediatamente cuando está %
embarazada.\\

Quita agradecido que el Pulque la salvara nuevamente una situación
difícil, pues el escándalo de los mil demonios que estaba armando en el
patio le evitó tener que seguir conversando con Paquita. Además de los
ladridos del Pulque se escuchaba el sonido provocado por el galope de
varios caballos. Todos los invitados ya estaban en casa. ¿quién podría
hacer a estas horas? quita se dirigió rápidamente a la puerta, la abrió
y vio cómo el pulque Le hacía fiestas a la persona que venía al frente
de una compañía de revolucionarios. Hasta que se acercaron lo
suficiente, pudo apreciar que quien venía al mando de la tropa era nada
menos que su hermana Gertrudis . A su lado cabalgaba el ahora general
Juan Alejandrez, el mismo que se la había robado tiempo atrás. Gertrudis
se bajó del caballo y como si el tiempo no hubiera pasado,~\sidenote{193}dijo
con desparpajo que sabiendo que era día de partir la rosca de reyes,
había venido por una buena taza de chocolate recién batido. Quita,
abrazándola emocionada, la llevó de inmediato a la mesa para cumplirle
el deseo. En la casa lo hacían como nadie, pues ponían mucho esmero en
todos los pasos a seguir para hacerlo, desde su fabricación hasta la
batida de chocolate, Que es otro capítulo importantísimo. La impericia
al batirlo puede ocasionar que un chocolate excelente calidad se
convierta en detestable, o por su falta de conocimiento o por estar
pasado de punto, o muy espeso o aún quemado.

El método para evitar todos las fallas anteriores es muy sencillo: se
pone en la lumbre una tablilla de chocolate con agua. La cantidad de
agua debe ser un poco mayor que la que se necesite para llenar el
pocillo en que se ha de hervir. Cuando da el primer hervor, se aparta
del fuego y se deshace las tablilla perfectamente, se bate con el
molinillo hasta que esté bien incorporada con el agua. Se vuelve de la
hornilla. Cuando dé otro hervor y quiera subir, se aparta de la lumbre.
Enseguida se vuelve a poner y así hasta que dé el tercer hervor .
Entonces se aparta por última ocasión y se bate. Se sirve la mitad en el
pozuelo y si vuelve a batir el restante. Entonces se sirve todo, dejando
la superficie cubierta de espuma. Se puede hacer también con leche en
lugar de agua, pero en este caso solo se le da un hervor, la segunda
ocasión que se pone al fuego se bate para que no quede muy espeso. El
chocolate hecho con agua es de mejor digestión que el de leche.

Gertrudis cerraba los ojos cada vez~\sidenote{194}que daba sorbo a la taza de
chocolate Que tenía frente a ella . La vida sería mucho más agradable si
uno pudiera llevarse a dondequiera que fuera a los sabores y los olores
de la casa materna. Bueno, esta ya no era su casa materna. Su madre
había muerto sin que ella se enterara.

Sintió mucho pena cuando Tita se lo informó. Ella había regresado con
la intención de mostrarle a Mamá Elena que había triunfado en la vida.
Era generala del ejército revolucionario. Este nombramiento se lo había
ganado a pulso, luchando como nadie en el campo de batalla. En la sangre
traía el don de mando, así que en cuanto ingresó al ejército,
rápidamente empezó a escalar puestos en el poder hasta alcanzar el mejor
puesto, y no solo eso, regresaba felizmente casada con Juan. Se habían
encontrado después de haberse dejado de ver por de un año y entre ellos
habían renacido la misma pasión que la del día en que se conocieron.
¡qué más podía pedir una persona! Cómo le gustaría que su madre la
hubiera visto y como le gustaría volverla a ver, aunque solo fuera para
que le indicara con la mirada que era necesario que utilizara la
servilleta para limpiarse los restos de chocolate en los labios.

Este chocolate estaba preparado como en los viejos tiempos.

Gertrudis lanzó una plegaria en silencio y con los ojos cerrados,
pidiendo que Tita viviera muchos años más cocinando las recetas de la
familia. Ni ella ni Rosaura tenían los conocimientos para hacerlo, por
lo tanto el día quita muriera moriría junto con ella el pasado de su~\sidenote{195}familia. Cuando todos terminaron de cenar pasaron a la sala, donde
dio comienzo el baile. El salón estaba perfectamente iluminado por una
colosal cantidad de velas. Juan impresionó a los invitados tocando de
maravilla la guitarra, la armónica y el acordeón. Gertrudis llevaba el
ritmo de las piezas que Juan interpretaba golpeando el piso con la punta
de su bota.

Lo miraba orgullosamente desde el fondo del salón, donde una corte de
admiradores la tenía rodeada, asediándola con preguntas sobre su
participación en la revolución. Gertrudis, con gran soltura, mientras
fumaba, les narraba fantásticas historias de las batallas en las que
había participado. En ese momento los tenía con la boca abierto
contándoles cómo había sido el primer fusilamiento que ordenó, pero sin
poderse contener, interrumpió el relato y se lanzó al centro del salón
donde empezó a bailar con donaire la polca \textit{ Jesusita en Chihuahua},
que Juan interpretaba magistralmente en el acordeón norteño. Con
liviandad, se levantaba la falda hasta la rodilla, mostrando gran
desenfado.

Esta actitud provocaba comentarios escandalosos de las mujeres ahí
reunidas.

Rosaura le dijo en el oído a Tita.
\\- Yo no sé de dónde sacó ese ritmo Gertrudis. A mamá no le gustaba %
bailar y dicen que papá lo hacía muy mal.\\

Quita levantó los hombros en señal de respuesta, aunque ella sabía
perfectamente de quién había heredado Gertrudis el ritmo y otras cosas.
Este secreto pensaba llevárselo a la tumba, pero no lo pudo hacer. Un
año más tarde~\sidenote{196}Gertrudis dio a la luz a un niño mulato. Juan
enfureció y amenazó con dejarla. No le perdonaba a Gertrudis que hubiera
vuelto a las andadas. Entonces Tita, para salvar ese matrimonio, confesó
todo. Por fortuna no se había atrevido a quemar las cartas, ahora sí que
con el \glqq{}negra pasado\grqq{} de su madre, pues estás le sirvieron perfectamente
de prueba para demostrar la inocencia de Gertrudis.

De cualquier manera fue un golpe difícil de asimilar, al menos no se
separaron, sino que vivieron para siempre juntos y pasando más tiempo
felices que enojados.

Así como sabía la razón del ritmo de Gertrudis, sabía la razón del
fracaso del matrimonio de su hermana y de su propio embarazo.

Ahora le gustaría saber cuál era la mejor solución. Eso era lo
importante. Lo bueno es que ya tenía alguien a quien confiar sus penas.
Esperaba que Gertrudis se quedara en el rancho lo suficiente como para
que las escuchara y la aconsejara. En cambio Chencha deseaba todo lo
contrario. Estaba furiosa con Gertrudis, bueno, no precisamente con ella
sino con el trabajo que representaba al atender a su tropa. En lugar de
gozar de la fiesta, a esas horas de la noche había tenido que poner una
gran mesa en patio y elaborar chocolate para los 50 hombres de su tropa.

Capítulo 10~\sidenote{197}octubre
torrejas de natas
ingredientes~\sidenote{199}una taza de natas
6 huevos
canela
almíbar

