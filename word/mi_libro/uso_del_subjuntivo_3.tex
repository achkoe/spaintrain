% we don't want ae.sty
\expandafter\def\csname ver@ae.sty\endcsname{}

\documentclass[25pt, a0paper]{tikzposter} % See Section 3
\usepackage{multirow}
\usepackage{fontspec}
\usepackage[utf8]{inputenc}
\usepackage[ngerman]{babel}
\usepackage{color}
\newcommand{\hl}[1]{\textbf{\textcolor{magenta}{#1}}}
\newcommand{\tb}[1]{\textbf{#1}}
\title{Subjuntivo 3}

\settitle{ \centering \vbox{\centering \color{titlefgcolor} {\bfseries \Huge \sc \@title \par} }}

\usecolorpalette{BlueGrayOrange}
\colorlet{innerblocktitlebgcolor}{colorOne}%
\colorlet{innerblocktitlefgcolor}{white}%
%\colorlet{innerblockbodybgcolor}{colorThree!30!white}%
%\colorlet{innerblockbodyfgcolor}{black}%


\begin{document}
\maketitle % See Section 4.1


\block{presente, pretérito perfecto, futuro simple, imperativo}{
    \setlength{\tabcolsep}{0em}
    \begin{center}
    \Large
    Wenn im Hauptsatz \\
    \tb{presente}, \tb{pretérito perfecto}, \tb{futuro simple}, \tb{imperativo} \\
    steht, folgt im Nebensatz:
    \end{center}
    \vskip20pt
    \begin{tabular}{cc}
        \innerblock[bodywidthscale=0.45, titlewidthscale=0.45]{
            \Large
            presente de subjuntivo
        }{
            \centering
            \Large
            wenn die Handlung in der \tb{Gegenwart} oder \tb{Zukunft} erfolgt
            \vskip20pt
            \coloredbox[]{
                Me \hl{piden} que les \hl{cuente} sobre Peru. \\
                Me \hl{han pedido} que les \hl{cuente} sobre Peru. \\
                \hl{Haŕe} lo que \hl{pueda}. \\
                \hl{Llámame} cuando \hl{puedas}.
            }
        }
        &
        \innerblock[bodywidthscale=0.45, titlewidthscale=0.45]{
            \Large
            perfecto de subjuntivo
        }{
            \centering
            \Large
            wenn die Handlung \tb{bereits erfolgt} ist
            \vskip20pt
            \coloredbox[]{
                Me \hl{alegro} de que me \hl{hayan invitado}. \\
                Me \hl{he alegrado} de que me \hl{hayan invitado}. \\
                Te \hl{daré} el nombre cuando me \hl{hayan confirmado} la reserva. \\
                \hl{Llámame} cuando \hl{hayas descansado}.
            }
        }
    \end{tabular}
}

\block{pretérito imperfecto, pretérito indefindo, pretérito pluscuamperfecto}{
    \setlength{\tabcolsep}{0em}
    \begin{center}
    \Large
    Wenn im Hauptsatz \\
    \tb{pretérito imperfecto}, \tb{pretérito indefindo}, \tb{pretérito pluscuamperfecto} \\
    steht, folgt im Nebensatz:
    \end{center}
    \vskip20pt
    \begin{tabular}{cc}
        \innerblock[bodywidthscale=0.45, titlewidthscale=0.45]{
            \Large
            imperfecto de subjuntivo
        }{
            \centering
            \Large
            wenn die Handlung \tb{gleichzeitig} oder \tb{nachzeitig} erfolgt.
            \vskip20pt
            \coloredbox[]{
                No \hl{podían} creer  que el clima \hl{fuera} así. \\
                Me \hl{pidieron} que los  \hl{acompañara}. \\
                Me \hl{habían recomendado} que \hl{fuera} personalmente.
            }
        }
        &
        \innerblock[bodywidthscale=0.45, titlewidthscale=0.45]{
            \Large
            pluscuamperfecto de subjuntivo
        }{
            \centering
            \Large
            wenn die Handlung \tb{vor} einer anderen Handlung erfolgt.
            \vskip20pt
            \coloredbox[]{
                No \hl{podían} creer que no \hl{hubiera habido} ciudades. \\
                \hl{Se quejaron} de que no les \hl{hubiera informado} bien. \\
                Me \hl{habían dicho} que \hl{hubiera tenido} que esperar.
            }
        }
    \end{tabular}
}

\block{condicional simple}{
    \setlength{\tabcolsep}{0em}
    \begin{center}
    \Large
    Wenn im Hauptsatz \\
    \tb{condicional simple} \\
    steht, folgt im Nebensatz:
    \vskip20pt
    \innerblock[bodywidthscale=0.45, titlewidthscale=0.45]{
        \Large
        imperfecto de subjuntivo
    }{
        \vskip20pt
        \coloredbox[]{
            Me \hl{encantaría} que \hl{tomáramos} unos mates.
            Me \hl{gustaría} que \hl{conocieran} a mis país.
        }
    }
    \end{center}
}

\block{condicional compuesto}{
    \setlength{\tabcolsep}{0em}
    \begin{center}
    \Large
    Wenn im Hauptsatz \\
    \tb{condicional compuesto} \\
    steht, folgt im Nebensatz:
    \end{center}
    \vskip20pt
    \begin{tabular}{cc}
        \innerblock[bodywidthscale=0.45, titlewidthscale=0.45]{
            \Large
            imperfecto de subjuntivo
        }{
            \centering
            \Large
            bei \tb{Gleichzeitigkeit} oder \tb{Vorzeitigkeit}.
            \vskip20pt
            \coloredbox[]{
                Me \hl{habría encantado} que lo \hl{vieras} todo.
            }
        }
        &
        \innerblock[bodywidthscale=0.45, titlewidthscale=0.45]{
            \Large
            pluscuamperfecto de subjuntivo
        }{
            \centering
            \Large
            bei \tb{Nachzeitigkeit}.
            \vskip20pt
            \coloredbox[]{
                Me \hl{habría gustado} que \hl{hubieran conocido} más.
            }
        }
    \end{tabular}
}

\end{document}








