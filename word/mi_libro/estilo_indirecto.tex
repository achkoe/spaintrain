% we don't want ae.sty
\expandafter\def\csname ver@ae.sty\endcsname{}

\documentclass[25pt, a0paper]{tikzposter} % See Section 3
\usepackage{multirow}
\usepackage{fontspec}
\usepackage[utf8]{inputenc}
\usepackage[ngerman]{babel}
\usepackage{tikz}
\usepackage{color}
\newcommand{\hl}[1]{\textbf{\textcolor{magenta}{#1}}}
\newcommand{\tb}[1]{\textbf{#1}}
\title{Estilo indirecto}

\usetikzlibrary {shapes.geometric}
\usetikzlibrary{automata,positioning}
\usetikzlibrary{shapes.multipart}
\usetikzlibrary{arrows.meta}
\usetikzlibrary{decorations.pathmorphing}
\usetikzlibrary{bending}


\settitle{ \centering \vbox{\centering \color{titlefgcolor} {\bfseries \Huge \sc \@title \par} }}

\usecolorpalette{BlueGrayOrange}
\colorlet{innerblocktitlebgcolor}{colorOne}%
\colorlet{innerblocktitlefgcolor}{white}%
%\colorlet{innerblockbodybgcolor}{colorThree!30!white}%
%\colorlet{innerblockbodyfgcolor}{black}%


\begin{document}
\maketitle % See Section 4.1


\block{Tiempos de estilo indirecto}{
    \Large
   \centering
   \begin{tikzpicture}[align=center]
    \draw (0,75) node[minimum height=1cm] (A0) %
    {Estilo directo};
    \draw (25,75) node[minimum height=1cm] (B0) %
    {Estilo indirecto};
    \draw (55,75) node[minimum height=1cm] (C0) %
    {Ejemplos};

    \bfseries

    \draw (0,70) node[fill=yellow!80, fill=yellow!80, minimum height=6cm, minimum width=18cm, rectangle,draw,rounded corners=1cm=1cm] (A1) %
    {Presente};
    \draw (25,70) node[fill=yellow!80, minimum height=6cm, minimum width=18cm, rectangle,draw,rounded corners=1cm] (B1) %
    {Pretérito Imperfecto};
    \draw (55,70) node[fill=yellow!20, minimum height=6cm, minimum width=28cm, rectangle,draw,rounded corners=1cm] (C1) %
    {\textmd{Estoy estupendamente.}\\[0.2em]$\rightarrow$\textit{\mdseries Juan dijo que estara estupendamente.}};
    \draw[->, blue, -{Stealth[scale=8]}] (A1) -- node[above] {} (B1);

    \draw (0,60) node[fill=yellow!80, minimum height=6cm, minimum width=18cm, rectangle,draw,rounded corners=1cm] (A2) %
    {Pretérito Indefinido\\[0.2em]Pretérito Perfecto\\[0.2em]Pretérito Pluscuamperfecto\\[0.2em]}; %
    \draw (25,60) node[fill=yellow!80, minimum height=6cm, minimum width=18cm, rectangle,draw,rounded corners=1cm] (B2) %
    {Pretérito Pluscuamperfecto}; %
    \draw (55,60) node[fill=yellow!20, minimum height=6cm, minimum width=28cm, rectangle,draw,rounded corners=1cm] (C2) %
    {\textmd{Pagué la cuenta.}\\[0.2em]$\rightarrow$\textit{\mdseries Juan dijo que había pagado la cuenta.}};
    \draw[->, blue, -{Stealth[scale=8]}] (A2) -- node[above] {} (B2);

    \draw (0,50) node[fill=yellow!80, minimum height=6cm, minimum width=18cm, rectangle,draw,rounded corners=1cm] (A3) %
    {Pretérito Imperfecto}; %
    \draw (25,50) node[fill=yellow!80, minimum height=6cm, minimum width=18cm, rectangle,draw,rounded corners=1cm] (B3) %
    {Pretérito Imperfecto}; %
    \draw (55,50) node[fill=yellow!20, minimum height=6cm, minimum width=28cm, rectangle,draw,rounded corners=1cm] (C3) %
    {\textmd{El helado estaba muy rico.}\\[0.2em]$\rightarrow$\textit{\mdseries Juan dijo que el helado estaba muy rico.}};
    \draw[->, blue, -{Stealth[scale=8]}] (A3) -- node[above] {} (B3);

    \draw (0,40) node[fill=yellow!80, minimum height=6cm, minimum width=18cm, rectangle,draw,rounded corners=1cm] (A4) %
    {Futuro Simple\\[0.2em]Condicional Simple}; %
    \draw (25,40) node[fill=yellow!80, minimum height=6cm, minimum width=18cm, rectangle,draw,rounded corners=1cm] (B4) %
    {Condicional Simple}; %
    \draw (55,40) node[fill=yellow!20, minimum height=6cm, minimum width=28cm, rectangle,draw,rounded corners=1cm] (C4) %
    {\textmd{Avisaré a Marina.}\\[0.2em]$\rightarrow$\textit{\mdseries Ella dijo que avisaría a Marina.}\\[0.2em]\textmd{Leería un libro.}\\[0.2em]$\rightarrow$\textit{\mdseries El dijo que leería un libro.}};
    \draw[->, blue, -{Stealth[scale=8]}] (A4) -- node[above] {} (B4);

    \draw (0,30) node[fill=yellow!80, minimum height=6cm, minimum width=18cm, rectangle,draw,rounded corners=1cm] (A5) %
    {Futuro Compuesto\\[0.2em]Condicional Compuesto}; %
    \draw (25,30) node[fill=yellow!80, minimum height=6cm, minimum width=18cm, rectangle,draw,rounded corners=1cm] (B5) %
    {Condicional Compuesto}; %
    \draw (55,30) node[fill=yellow!20, minimum height=6cm, minimum width=28cm, rectangle,draw,rounded corners=1cm] (C5) %
    {\textmd{Habría leído la novela.}\\[0.2em]$\rightarrow$\textit{\mdseries Ella dijo que habría leído la novela.}};
    \draw[->, blue, -{Stealth[scale=8]}] (A5) -- node[above] {} (B5);

    \draw (0,20) node[fill=yellow!80, minimum height=6cm, minimum width=18cm, rectangle,draw,rounded corners=1cm] (A6) %
    {Imperativo\\[0.2em]Presente de Subjuntivo\\[0.2em]Pretérito Imperfecto\\[0.2em]de Subjuntivo}; %
    \draw (25,20) node[fill=yellow!80, minimum height=6cm, minimum width=18cm, rectangle,draw,rounded corners=1cm] (B6) %
    {Pretérito Imperfecto\\[0.2em]de Subjuntivo}; %
    \draw (55,20) node[fill=yellow!20, minimum height=6cm, minimum width=28cm, rectangle,draw,rounded corners=1cm] (C6) %
    {\textmd{Quedemos algún día.}\\[0.2em]$\rightarrow$\textit{\mdseries Juan surgirió que quedaran algún día.}\\[0.2em]%
    \textmd{Quiero que vengas.}\\[0.2em]$\rightarrow$\textit{\mdseries Juan quería que vinieras.}\\[0.2em]%
    \textmd{Quisiera comer algo.}\\[0.2em]$\rightarrow$\textit{\mdseries Juan dice que quisiera comer algo.}};
    \draw[->, blue, -{Stealth[scale=8]}] (A6) -- node[above] {} (B6);

    \draw (0,10) node[fill=yellow!80, minimum height=6cm, minimum width=18cm, rectangle,draw,rounded corners=1cm] (A7) %
    {Pretérito perfecto\\[0.2em]de subjuntivo}; %
    \draw (25,10) node[fill=yellow!80, minimum height=6cm, minimum width=18cm, rectangle,draw,rounded corners=1cm] (B7) %
    {Pretérito pluscuamperfecto\\[0.2em]de subjuntivo}; %
    \draw (55,10) node[fill=yellow!20, minimum height=6cm, minimum width=28cm, rectangle,draw,rounded corners=1cm] (C7) %
    {\textmd{Quizá haya olvidado el monedero en casa.}\\[0.2em]$\rightarrow$\textit{\mdseries Ella pensó que quizá hubiera}\\[0.2em]\textit{\mdseries olvidado el monedero en casa.}};
    \draw[->, blue, -{Stealth[scale=8]}] (A7) -- node[above] {} (B7);

   \end{tikzpicture}
}

\block{Sustituciones}{
   \Large \bfseries
   \centering
   \tabcolsep2em
   \begin{tabular}{rcl}
   Estilo directo   & ~             & Estilo indirecto      \\\cline{1-1}\cline{3-3}
   hoy              & $\rightarrow$ & aquel día             \\
   ahora            & $\rightarrow$ & entonces              \\
   ayer             & $\rightarrow$ & el día anterior       \\
   la semana pasada & $\rightarrow$ & la semana anterior    \\
   el próximo año   & $\rightarrow$ & el año siguiente      \\
   mañana           & $\rightarrow$ & al día siguiente      \\
   aquí             & $\rightarrow$ & allí, ahí             \\
   \end{tabular}

}

\end{document}








