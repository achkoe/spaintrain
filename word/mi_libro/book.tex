\documentclass[twoside, 11pt]{article}
\usepackage[a4paper, textheight=670pt, textwidth=466pt, headheight=17pt, marginparsep=12pt, footskip=47pt, headsep=20pt]{geometry}
\pagestyle{plain}
\usepackage{multicol}
%\documentclass[pagesize, DIV=15, fontsize=11pt, parskip=half, twoside, BCOR=5mm, headings=small]{scrartcl}
\usepackage{layout}
\usepackage[utf8]{inputenc}
\usepackage[german, spanish]{babel}
\usepackage{fontspec}
\usepackage{latexsym}
\usepackage{pifont}
\usepackage{hyperref}
\usepackage{color}
\usepackage{titlesec}
\usepackage{paralist}
\usepackage{booktabs}
\usepackage{xcolor}
\usepackage[framemethod=tikz]{mdframed}
%--
\mdfsetup{}
\newmdenv[roundcorner=5pt, linecolor=green!80, backgroundcolor=green!10]{highlight}
\newmdenv[roundcorner=5pt, linecolor=orange!80, backgroundcolor=brown!10]{hint}

%\setlist{noitemsep}
\setlength{\parindent}{0em}
\parskip2ex plus1ex minus0.5ex
\hypersetup{colorlinks, linkcolor=black, citecolor=blue, urlcolor=black}

\newcommand{\thh}[1]{\footnotesize{#1}}
\newcommand{\danger}{\ding{238}}

%-------------------------------------------------------
% Aligné à gauche, suivi d'un filet horizontal
\makeatletter
\def\section{\@ifstar\unnumberedsection\numberedsection}
\def\numberedsection{\@ifnextchar[%]
  \numberedsectionwithtwoarguments\numberedsectionwithoneargument}
\def\unnumberedsection{\@ifnextchar[%]
  \unnumberedsectionwithtwoarguments\unnumberedsectionwithoneargument}
\def\numberedsectionwithoneargument#1{\numberedsectionwithtwoarguments[#1]{#1}}
\def\unnumberedsectionwithoneargument#1{\unnumberedsectionwithtwoarguments[#1]{#1}}
\def\numberedsectionwithtwoarguments[#1]#2{%
  \ifhmode\par\fi
  \removelastskip
  \vskip 3ex\goodbreak
  \refstepcounter{section}%
  \noindent\hrulefill\nobreak
  \begingroup
  \noindent
  \leavevmode\Large\bfseries\raggedright
  \thesection\ #2\par\nobreak
  \endgroup
  \noindent\hrulefill\nobreak
  \vskip 2ex\nobreak
  \addcontentsline{toc}{section}{%
    \protect\numberline{\thesection}%
    #1}%
  }
\def\unnumberedsectionwithtwoarguments[#1]#2{%
  \ifhmode\par\fi
  \removelastskip
  \vskip 3ex\goodbreak
%  \refstepcounter{section}%
  \noindent\hrulefill\nobreak\\
  \begingroup
  \noindent
  \leavevmode\Large\bfseries\raggedright
%  \thesection\
  #2\\\nobreak
  \endgroup
  \noindent\hrulefill\nobreak
  \vskip 2ex\nobreak
  \addcontentsline{toc}{section}{%
%    \protect\numberline{\thesection}%
    #1}%
  }
\makeatother
%-------------------------------------------------------
\begin{document}

\title{Spanisch}
\date{-- Grammatik und mehr --}
\maketitle
\clearpage

%\layout\clearpage
\section*{El presente}

\begin{highlight}
Das Präsens wird für Handlungen und Zustände in der Gegenwart verwendet.
\end{highlight}

\begin{compactitem}
    \item Mi novia **trabaja** en un restaurante.
    \item Pepe **vive** cerca de Bilbao.
    \item Jesús **come** un  bocadillo
\end{compactitem}

\begin{highlight}
Des Weiteren werden Gewohnheiten und zukünftige Vorgänge mit dem Präsens wiedergegeben.
\end{highlight}

\begin{compactitem}
    \item Los jueves **juego** al tenis.
    \item Mi madre siempre **me llama** por la noche.
    \item El tren **sale** a las diez y media.
\end{compactitem}

Genau wie im Deutschen gibt es auch im Spanischen das historische Präsens.

\begin{compactitem}
    \item Cervantes **publica** el Quijote en 1605.
    \item Lorca **ecscribe** una serie de poesias.
\end{compactitem}

\begin{highlight}
Starke Befehle können ebenfalls im Präsens erteilt werden.
\end{highlight}

\begin{compactitem}
    \item ¡**Te duchas** y **te vas**!
    \item ¡**Limpias** tu cuarto y **haces** tus deberes!
\end{compactitem}
\clearpage
\section*{Pretérito indefindo}

Das Pretérito indefindo bezeichnet eine begrenzte und abgeschlossene Handlung
in der Vergangenheit.
Es wird gebraucht \ldots

\begin{highlight}
In Erzählungen mit einzelnen, abgeschlossenen Ereignissen,
die aufeinander folgen ("`zuerst geschah das, dann das ..."'). \\
Die Dauer spielt dabei keine Rolle, wichtig ist nur, dass Beginn und Ende der
Handlung erfasst werden.
\end{highlight}

\begin{compactitem}
    \item Ayer, a las 9 de la mañana, una mujer **entró** en el museo y **robó** un
          cuadro. Despues **fue** al aeropuerto, **compró** un pasaje y **tomó** el
          avion a Madrid. Alli, **llegó** a la 1 y **cambió** de avion.
    \begin{compactitem}
        \item Gestern um 9 Uhr ging eine Frau ins Museum and stahl ein Bild. Anschliessend
              fuhr sie zum Flughafen, kaufte eine Flugkarte und flog nach Madrid.
              Dort kam sie um 1 Uhr an und wechselte das Flugzeug.
    \end{compactitem}
\end{compactitem}

\begin{highlight}
In Erzählungen mit Ereignissen, die zu einem bestimmten
Zeitpunkt stattgefunden haben ("`um 1 Uhr, letzte Woche, am Dienstag,
gestern, ..."').
\end{highlight}

\begin{compactitem}
    \item Las chicas **llegaron** a las una.
    \begin{compactitem}\itshape
        \item Die Mädchen kamen um 1 Uhr an.
    \end{compactitem}
    \item Ayer **cené** en casa de un amigo.
    \begin{compactitem}\itshape
        \item  Gestern aß ich bei einem Freund zu Abend.
    \end{compactitem}
    \item La semana pasada Carlos se **compró** un Coche.
    \begin{compactitem}\itshape
        \item Letzte Woche kaufte Carlos ein Auto.
    \end{compactitem}
\end{compactitem}


\begin{highlight}
Um etwas auszudrücken, was in einer begrenzten Zeitspanne
war, geschah oder vor sich ging ("`innerhalb von 5 Jahren, 3 Jahrhunderten, ..."').
\end{highlight}

\begin{compactitem}
    \item **Vivieron** alli cinco años.
    \begin{compactitem}\itshape
        \item Sie lebten fünf Jahre dort.
    \end{compactitem}
    \item Columbia **perteneció** a Espanda durante más de dos siglos.
    \begin{compactitem}\itshape
        \item Kolumbien gehörte mehr als zwei Jahrhunderte zu Spanien.
    \end{compactitem}
\end{compactitem}


\begin{highlight}
Für eine plötzlich neu eintretende Handlung.
\end{highlight}

\begin{compactitem}
    \item Estaba en la calle cuando de repente **llegó** un joveny me **quitó** el bolso.
    \begin{compactitem}\itshape
        \item Ich war auf der Straße, als plötzlich ein junger Mann kam und mir die
              Tasche entriss.
    \end{compactitem}
\end{compactitem}


\begin{highlight}
Für eine Wiederholung, die abgeschlossen ist.
\end{highlight}

\begin{compactitem}
    \item Pepe me **preguntó** cuatro veces si le podia prestar dinero, y cuatro veces
          le **contesté** que no me era posible.
    \begin{compactitem}\itshape
        \item Pepe fragte mich viermal, ob ich ihm Geld leihen könne, und viermal
              antwortete ich ihm, dass es mir nicht möglich sei.
    \end{compactitem}
\end{compactitem}


\paragraph{Signalwörter} (adverbiale Bestimmungen) für das Pretérito indefindo sind:

\begin{compactitem}
    \item **anoche** -- //gestern Abend//, //gestern Nacht//
    \item **aquel día** -- //an jenem Tag//
    \item **ayer** -- //gestern//
    \item **en** 1992 -- 1992
    \item **hace ... años** -- //vor ... Jahren//
    \item **entonces** -- //dann//, //danach//
    \item **la semana pasada** -- //letzte Woche//
    \item **el año pasado** -- //letztes Jahr//
    \item **un día** -- //eines Tages//
\end{compactitem}

\subsection*{Verben}

\begin{tabular}{ccccccc}
\hline
\thh{Infinitiv} & \thh{yo}       & \thh{tú}        & \thh{él/ella/usted} & \thh{nosotros/-as} & \thh{vosotros/-as} & \thh{ellos/ellas/ustedes} \\
\hline
\multicolumn{7}{c}{//regelmäßig//} \\
\hline
habl**ar**      & habl**é**   & habl**aste**   & habl**ó**        & habl**amos**    & habl**asteis**     & hablar**aron**             \\
beb**er**       & beb**í**    & beb**iste**    & beb**ió**         & beb**imos**     & beb**isteis**      & beb**ieron**              \\
escrib**ir**    & escrib**í** & escrib**iste** & escrib**ió**      & escrib**imos**  & escrib**isteis**   & escrib**ieron**           \\
\hline
\multicolumn{7}{c}{//Verben auf **-ir**: **e** --> **i**//} \\
\hline
sentir     & sentí      & sentiste   & sintió     & sentimos   & sentisteis & sintieron  \\
pedir     & pedí      & pediste   & pidió     & pedimos   & pedisteis & pidieron  \\
seguir     & seguí      & seguiste   & siguió     & seguimos   & seguisteis & siguieron  \\
reir     & reí      & reiste   & rió      & reimos   & reisteis & rieron   \\
repetir     & repetí      & repetiste   & repitió     & repetimos   & repetisteis & repitieron  \\
\hline
\multicolumn{7}{c}{//Verben auf **-ir**: **o** --> **u**//} \\
\hline
morir     & morí      & moriste   & murió     & morimos   & moristeis & murieron  \\
dormir     & dormí      & dormiste   & durmió     & dormimos   & dormisteis & durmieron  \\
\hline
\multicolumn{7}{c}{//Verben auf **-uir**, einige Verben auf **-er** und **-ir**: **i** --> **y**.//} \\
\hline
destruir & destruí & destruiste & destruyó & destruimos & destruisteis & destruyeron \\
leer & leí & leiste & leyó & leimos & leisteis & leyeron \\
creer & creí & creiste & creyó & creimos & creisteis & creyeron \\
caer & caí & caiste & cayó & caimos & caisteis & cayeron \\
oir & oí & oiste & oyó & oimos & oisteis & oyeron \\
poseer & poseí & poseiste & poseyó & poseimos & poseisteis & poseyeron \\
\hline
\multicolumn{7}{c}{//unregelmäßig//} \\
\hline
hacer & hice & hiciste & hizo & hicimos & hicisteis & hicieron \\
tener & tuve & tuviste & tovo & tuvimos & tuvisteis & tuvieron \\
estar & estuvé & estuviste & estuvo & estuvimos & estuvisteis & estuvieron \\
poder & pude & pudiste & pudo & pudimos & pudisteis & pudieron \\
dar & di & diste & dio & dimos & disteis & dieron \\
ir / ser & fui & fuiste & fue & fuimos & fuisteis & fueron \\
\hline
\end{tabular}


\clearpage
\section*{Pretérito perfecto}

\begin{highlight}
Das Pretérito Perfecto (Perfekt) wird verwendet für abgeschlossene Handlungen, die noch einen
Bezug zur Gegenwart haben. Dies kann eine abgeschlossene Handlung in einem Zeitraum sein, der
noch zur Gegenwart zählt (esto/esta \ldots, hoy, ya oder todavia), oder eine abgeschlossene Handlung,
die Einfluss auf die Gegenwart oder Zukunft hat.
\end{highlight}

\begin{highlight}
Gebildet wird das Pretérito perfecto mit der Präsensform des Hilfsverbs //haber// und dem
unveränderlichen Partizip Perfekt:\\
**\{he, has, ha, hemos, habéis, han\} + Partizip Perfekt**
\end{highlight}

\begin{hint}
\danger{}
Im Spanischen gibt es nur ein Hilfsverb
für die Bildung des Perfekts.
\end{hint}

\begin{multicols}{2}
\begin{compactitem}
    \item **Hemos viajado** mucho.
    \begin{compactitem}\itshape
        \item Wir **sind** viel gereist.
    \end{compactitem}
    \item ¿Qué **has visto**?
    \begin{compactitem}\itshape
        \item Was **hast** du gesehen?
    \end{compactitem}
\end{compactitem}
\end{multicols}

\begin{hint}
Hilfsverb und Partizip werden nie getrennt, Dies gilt auch für die Verneinung,
die vor dem konjugierten Verb und / oder Reflexivpronomen steht
(**no** + Pronomen + **haber**).
\end{hint}

\begin{compactitem}
    \item Montse **ha leído** un libro en 2 días.
    \item **No he dicho** nada.
    \item Jimena aún **no se ha levantado**.
\end{compactitem}

Ein Fragesatz, der nach der Realisierung einer Handlung fragt, wird im Spanischen meist mit Pretérito perfecto
formuliert. Erwidert man im Pretérito perfecto, so misst man der Handlung und dem Gegenwartsbezug Wichtigkeit bei.
Antwortet man degegen im indefindo, so kommt dem Zeitpunkt und der Abgeschlossenheit der Handlung Bedeutung zu.

\begin{compactitem}
    \item ¿**Has visto** la nueva película de Tarantino?
    \begin{compactitem}
        \item Si, ya la **he visto**.
        \item Si, la **vi** la semana pasada.
    \end{compactitem}
\end{compactitem}

\begin{compactitem}
    \item Ya **había aprendido** tres meses antes del examen.
    \begin{compactitem}\itshape
      \item Er hatte schon drei Monate vor der Prüfung gelernt.
    \end{compactitem}
\end{compactitem}

\paragraph{Signalwörter} für das Pretérito perfecto sind:

\begin{multicols}{2}
\begin{compactitem}
    \item **ya** -- //schon//
    \item **todavía no** / **aún no** -- //noch nicht//
    \item **nunca** -- //noch nie//
    \item **siempre** -- //schon immer//
    \item **estos días**  -- //in diesen Tagen//
    \item **desde entonces** -- //seitdem//
    \item **hoy** -- //heute//
    \item **alguna vez** -- //schon einmal//
    \item **hasta ahora** -- //bis jetzt//
    \item **esta mañana** -- //heute morgen//
    \item **esta tarde** -- //heute Nachmittag//
    \item **esta semana**-- //diese Woche//
    \item **este mes** -- //diesen Monat//
    \item **este año** -- //dieses Jahr//
\end{compactitem}
\end{multicols}





\clearpage
\section*{Plusquamperfecto}

\begin{highlight}
Mit dem Pretérito Pluscuamperfecto (Plusquamperfekt) werden  Handlungen ausgedrückt, die zeitlich vor einem
bestimmten Zeitpunkt in der Vergangenheit stattgefunden haben.
\end{highlight}

Das Plusquamperfekt wird immer dann verwendet, wenn bei einer Erzählung über die Vergangenheit
(im Präteritum) auf etwas zurückgeblickt wird, das zuvor passierte.

Man bildet den Plusquamperfekt mit dem Imperfekt von **haber** und dem Partizip Perfekt des Hauptverbs.

\begin{tabular}{cc}
\hline
yo **había**                   & ~             \\
tú **habías**                  & compr**ado**  \\
él/ella/usted **había**        & entend**ido** \\
nosotros/-as **habíamos**      & recib**ido**  \\
vosotros/-as **habíais**        & ~             \\
ellos/ellas/Ustedes **habían** & ~             \\
\hline
\end{tabular}



\begin{compactitem}
    \item Ya **había aprendido** tres meses antes del examen.
    \begin{compactitem}\itshape
      \item Er hatte schon drei Monate vor der Prüfung gelernt.
    \end{compactitem}
\end{compactitem}

\section*{Partizip Perfekt}

Das Partizip Perfect wird unter anderem for den Pretérito Perfecto und den Pretérito Pluscuamperfecto
verwendet.

\subsection*{Verben}

\begin{tabular}{ccc}
\toprule
\thh{Infinitivform} & \thh{Partizip Perfekt } & \thh{Beispiel} \\
\midrule
\multicolumn{3}{c}{//regelmäßig//} \\
\midrule
-**ar**             & -**ado**                & habl**ado**    \\
-**er**             & -**ido**                & com**ido**     \\
-**ir**             & -**ido**                & viv**ido**     \\
\midrule
\multicolumn{3}{c}{//unregelmäßig//} \\
\midrule
hacer               & hecho                   & ~              \\
poner               & puesto                  & ~              \\
decir               & dicho                   & ~              \\
volver              & vuelto                  & ~              \\
ver                 & visto                   & ~              \\
romper              & roto                    & ~              \\
abrir               & abierto                 & ~              \\
descubrir           & descubierto             & ~              \\
escribir            & escrito                 & ~              \\
morir               & muerto                  & ~              \\
\bottomrule
\end{tabular}


\clearpage
\section*{Conditional}

Das Condicional ist ein Modus, in dem ausgedrückt wird, das eine Handlung passieren könnte.

\begin{highlight}
Das Condicional wird verwendet um

\begin{compactitem}
    \item einen höflichen Wunsch auszudrücken
    \begin{compactitem}\itshape
        \item Me **gustaría** hablar con el Sr. Perez.
        \item **Prefería** no salir este sábado.
    \end{compactitem}
    \item einen Gefallen zu bitten
    \begin{compactitem}\itshape
        \item ¿Me **podrías** ayurdar?
    \end{compactitem}
    \item Ratschläge zu erteilen
    \begin{compactitem}\itshape
        \item Os **recomendaríamos** no fumar tanto.
        \item En tu lugar, no **conduciría** sin gafas.
    \end{compactitem}
    \item die persönliche Meinung etwas abgeschwächt wiederzugeben
    \begin{compactitem}\itshape
        \item Yo **diría** que este argumente no vale.
        \item Yo **añadiría** que no funciona así.
    \end{compactitem}
    \item irreale Sachverhalte auszudrücken
    \begin{compactitem}\itshape
        \item Con un millón de euros **dejaría** de trabajar.
        \item En tu lagar no **compría** ese coche.
    \end{compactitem}
\end{compactitem}
\end{highlight}

Das Condicional wird auch für die indirekte Rede verwendet. Wenn in der direkten Rede das Condicional oder das Futur auftaucht, dann wird das entsprechende Verb in der indirekten Rede mit dem Condicional ausgedrückt.

\begin{compactitem}
    \item Rosa: "`Me gustaría volver este viernes."' --> Dijo que le **gustaría** volver ese viernes.
    \item Paula: "`Iré a México el próximo año."' --> Paula me dijo que **iría** a México el año que viene.
\end{compactitem}

\begin{hint}
\danger{}
Anstelle von **querría** (ich möche gerne) findet man häufig die Form **quisiera**.
Dabei handelt es sich um den Pretérito imperfecto de subjuntivo von **querer**, das
verwendet wird, um das Condicional auszudrücken. Dies gilt nur für **querer** in der 1. Person
Singular.
\end{hint}

\begin{compactitem}
    \item **Quisiera** pasar las vacaciones con mi novio.
    \begin{compactitem}\itshape
        \item Statt: **Querría** pasar las vacaciones con mi novio.
    \end{compactitem}
    \item **Quisiera** hacer un viaje alrededor del mundo.
    \begin{compactitem}\itshape
        \item Statt: **Querría** hacer un viaje alrededor del mundo.
    \end{compactitem}
\end{compactitem}


\subsection*{Verben}

\begin{tabular}{ccccccc}
\hline
\thh{Infinitiv} & \thh{yo}       & \thh{tú}        & \thh{él/ella/usted} & \thh{nosotros/-as} & \thh{vosotros/-as} & \thh{ellos/ellas/ustedes} \\
\hline
\multicolumn{7}{c}{//regelmäßig//} \\
\hline
habl**ar**      & hablar**ía**   & hablar**ías**   & hablar**ía**        & hablar**íamos**    & hablar**íais**     & hablar**ían**             \\
beb**er**       & beber**ía**    & beber**ías**    & beber**ía**         & beber**íamos**     & beber**íais**      & beber**ían**              \\
escrib**ir**    & escribir**ía** & escribir**ías** & escribir**ía**      & escribir**íamos**  & escribir**íais**   & escribir**ían**           \\
\hline
\multicolumn{7}{c}{//unregelmäßig//} \\
\hline
poner           & pondr**ía**    & pondr**ías**    & pondr**ía**         & pondr**íamos**     & pondr**íais**      & pondr**ían**              \\
salir           & saldr**ía**    & saldr**ías**    & saldr**ía**         & saldr**íamos**     & saldr**íais**      & saldr**ían**              \\
poder           & podr**ía**     & podr**ías**     & podr**ía**          & podr**íamos**      & podr**íais**       & podr**ían**               \\
tener           & tendr**ía**    & tendr**ías**    & tendr**ía**         & tendr**íamos**     & tendr**íais**      & tendr**ían**              \\
caber           & capr**ía**     & capr**ías**     & capr**ía**          & capr**íamos**      & capr**íais**       & capr**ían**               \\
decir           & dir**ía**      & dir**ías**      & dir**ía**           & dir**íamos**       & dir**íais**        & dir**ían**                \\
venir           & vend**ía**     & vend**ías**     & vend**ía**          & vend**íamos**      & vend**íais**       & vend**ían**               \\
querrer         & querr**ía**    & querr**ías**    & querr**ía**         & querr**íamos**     & querr**íais**      & querr**ían**              \\
valer           & valdr**ía**    & valdr**ías**    & valdr**ía**         & valdr**íamos**     & valdr**íais**      & valdr**ían**              \\
saber           & sabr**ía**     & sabr**ías**     & sabr**ía**          & sabr**íamos**      & sabr**íais**       & sabr**ían**               \\
hacer           & har**ía**      & har**ías**      & har**ía**           & har**íamos**       & har**íais**        & har**ían**                \\
haber           & habr**ía**     & habr**ías**     & habr**ía**          & habr**íamos**      & habr**íais**       & habr**ían**               \\
\hline
\end{tabular}
\clearpage
\section*{Imperativo afirmativa - Bejahter Imperativ}
\subsection*{Verben}

\begin{tabular}{cccccc}
\hline
\thh{Infinitiv} & \thh{tú}    & \thh{él/ella/usted} & \thh{nosotros/-as} & \thh{vosotros/-as} & \thh{ellos/ellas/ustedes} \\
\hline
habl**ar**      & habl**a**   & habl**e**           & habl**emos**       & habl**ad**         & habl**en**                \\
beb**er**       & beb**e**    & beb**a**            & beb**amos**        & beb**ed**          & beb**an**                 \\
escrib**ir**    & escrib**e** & escrib**a**         & escrib**amos**     & escrib**id**       & escrib**an**              \\
\hline
ir              & ve          & vay**a**            & vay**amos**        & i**d**             & vay**an**                 \\
ser             & sé          & se**a**             & se**amos**         & s**ed**            & se**an**                  \\
decir           & di          & di**ga**            & di**gamos**        & dec**id**          & dig**an**                 \\
poner           & pon         & pong**a**           & pong**amos**       & pon**ed**          & pong**an**                \\
salir           & sal         & salg**a**           & salg**amos**       & sal**id**          & salg**an**                \\
venir           & ven         & veng**a**           & veng**amos**       & ven**id**          & veng**an**                \\
tener           & tener       & teng**a**           & teng**amos**       & ten**id**          & teng**an**                \\
hacer           & haz         & hag**a**            & hag**amos**        & hac**ed**          & hag**an**                 \\
haber           & he          & hay**a**            & hay**amos**        & hab**ed**          & hay**an**                 \\
\hline
\end{tabular}

\subsection*{Stellung des Pronomens}
\begin{itemize}
    \item Pronomen stehen nach dem Verb
    \item Ein Akzent wird gesetzt zur Erhaltung der Betonung:
    \begin{itemize}
        \item ¡Compra la ropa! $\rightarrow$ ¡Cómprala!
        \item ¡Pregunten a nosotros! $\rightarrow$ ¡Pregúntenos! (???)
    \end{itemize}
\end{itemize}


\subsection*{Besonderheit beim Reflexixpronomen}
\begin{itemize}
    \item In der 1. Person Plural werden die Endungen **emos**, **amos** zu **émo**, **ámo**.
    \item In der 2. Person Plural entfällt bei den Endungen **ad**, **ed**, **id** das **d**.
    \item Die anderen Personen erhalten einen Akzent auf dem letzten Vokal des Stammes.
\end{itemize}

\begin{tabular}{cccccc}
\hline
~        & \thh{tú} & \thh{él/ella/usted} & \thh{nosotros/-as} & \thh{vosotros/-as} & \thh{ellos/ellas/ustedes} \\
\hline
hablarse & háblate  & háblese             & hablémonos         & hablaos            & háblense                  \\
comerse  & cómete   & cómase              & comámonos          & comeos             & cómanse                   \\
vivirse  & vívete   & vívase              & vivámonos          & vivíos             & vívanse                   \\
\hline
\end{tabular}
\clearpage
\section*{El futuro simple}

\begin{highlight}
Das Futuro simple wird verwendet, um Vorgänge und Handlungen auszudrücken, die in der Zukunft liegen.

Es wird ebenfalls verwendet, um schriftliche formelle Ankündigungen oder Einladungen auszudrücken.

Das Futuro simple kann auch Zweifel, Unsicherheit und Vermutungen in Bezug auf die Zukunft wiedergeben.

Mit dem Futuro simple können auch Gebote, Forderungen oder Verbote zum Ausdruck gebracht werden.
\end{highlight}

\begin{hint}
Umgangssprachlich wird der Futuro simple häufig durch **ir a** + Infinitiv ersetzt, um über Vorgänge und Handlungen
in naher Zukunft zu sprechen.
\end{hint}

\begin{hint}
Betont wird im Futuro simple immer die Endung. Bis auf die erste Person Plural tragen alle Endungen einen Akzent.
\end{hint}

Beispiele regelmäßig:

\begin{compactitem}
    \item **Prepararé** un plato italiano.
    \item  ¿**Te mudarás** este fin de semana?
    \item  La nueva tienda **abrirá** el mes que viene.
    \item  Esto lo **preguntaremos** a nuestro profe.
    \item  Mañana **aprenderéis** más palabras-
    \item  Los niños **compartirán** una habitación.
\end{compactitem}

Beispiele unregelmäßig:

\begin{compactitem}
    \item El mueble no **cabrá** en el maletero.
    \item ¿Quién les **dirá**?
    \item Lo **haremos** mejor en el futuro.
    \item Los estudiantes **podrán** usar el internet.
    \item Siempre te **querré**.
    \item Mañana **sabrás** más.
    \item ¿Qué nombre lo **pondrás** a tu hijo?
    \begin{compactitem}
        \item Wie würden Sie Ihren Sohn nennen?
    \end{compactitem}
    \item ¿Qué **tendremos** que hacer?
    \begin{compactitem}
        \item Was müssen wir tun?
    \end{compactitem}
    \item Las entradas **saldrán** a la venta.
    \begin{compactitem}
        \item Die Tickets werden in den Verkauf gehen.
    \end{compactitem}
    \item ¿Cuánto **valdrà** el coche?
    \begin{compactitem}
        \item Wie viel wird das Auto wert sein?
    \end{compactitem}
    \item Mi padre **vendrà** por Navidad.
    \begin{compactitem}
        \item Mein Vater wird zu Weihnachten kommen.
    \end{compactitem}
\end{compactitem}

\clearpage

\subsection*{Verben}

\begin{tabular}{ccccccc}
\toprule
\thh{Infinitiv} & \thh{yo}       & \thh{tú}        & \thh{él/ella/usted} & \thh{nosotros/-as} & \thh{vosotros/-as} & \thh{ellos/ellas/ustedes} \\
\midrule
\multicolumn{7}{c}{//regelmäßig//}                                                                                                             \\
\midrule
hablar          & hablar**é**    & hablar**ás**    & hablar**á**         & hablar**emos**     & hablar**éis**      & hablar**án**             \\
comer           & comer**é**     & comer**ás**     & comer**á**          & comer**emos**      & comer**éis**       & comer**án**              \\
escribir        & escribir**é**  & escribir**ás**  & escribir**á**       & escribir**emos**   & escribir**éis**    & escribir**án**           \\
\midrule
\multicolumn{7}{c}{//unregelmäßig: das **e** bzw. **i** der Infinitivendung entfällt//}                                                        \\
\midrule
caber       & cab**ré**      & cab**rás**      & cab**rá**           & cab**remos**       & cab**rèis**        & cab**rán**                \\
decir       & dir**é**       & dir**ás**       & dir**á**            & dir**emos**        & dir**èis**         & dir**án**                 \\
hacer       & har**é**       & har**ás**       & har**á**            & har**emos**        & har**éis**         & har**án**                 \\
poder       & podr**é**      & podr**ás**      & podr**á**           & podr**emos**       & podr**éis**        & podr**án**                \\
querer      & querr**é**     & querr**ás**     & querr**á**          & querr**emos**      & querr**éis**       & querr**án**               \\
saber       & sabr**é**      & sabr**ás**      & sabr**á**           & sabr**emos**       & sabr**éis**        & sabr**án**                \\
\midrule
\multicolumn{7}{c}{//unregelmäßig: das **e** bzw. das **i** der Infinitivendung --> **d**//}                                                   \\
\midrule
poner       & pondr**é**     & pondr**ás**     & pondr**á**          & pondr**emos**      & pondr**éis**       & pondr**án**               \\
tener       & tendr**é**     & tendr**ás**     & tendr**á**          & tendr**emos**      & tendr**éis**       & tendr**án**               \\
salir       & saldr**é**     & saldr**ás**     & saldr**á**          & saldr**emos**      & saldr**éis**       & saldr**án**               \\
valer       & valdr**é**     & valdr**ás**     & valdr**á**          & valdr**emos**      & valdr**éis**       & valdr**án**               \\
venir       & vendr**é**     & vendr**ás**     & vendr**á**          & vendr**emos**      & vendr**éis**       & vendr**án**               \\
\bottomrule
\end{tabular}




\end{document}
