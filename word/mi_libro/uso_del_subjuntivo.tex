% we don't want ae.sty
\expandafter\def\csname ver@ae.sty\endcsname{}

\documentclass[25pt, a0paper]{tikzposter} % See Section 3
\usepackage{multirow}
\usepackage{fontspec}
\usepackage[utf8]{inputenc}
\usepackage[ngerman]{babel}
\usepackage{color}
\newcommand{\hl}[1]{\textbf{\textcolor{magenta}{#1}}}
\title{Subjuntivo 1}
% \institute{Inst} % See Section 4.1
%\author{Auth} \titlegraphic{Logo}
%\usetheme{Basic} % See Section 5
\usecolorpalette{BlueGrayOrange}
\begin{document}
\maketitle % See Section 4.1


\begin{columns} % See Section 4.4
\column{0.5} % See Section 4.4
\block{quizá/s, tal vez, a lo mejor}{%
    \Large
    Nach Adverbien \textbf{quizá/s} und \textbf{tal vez}: \textbf{Indikativ} oder \textbf{subjuntivo}. \\

    \begin{tabular}{lcll}
        \textbf{Indikativ} & $\Rightarrow$ & \textbf{höhere} & \multirow{2}{10mm}{Wahrscheinlichkeit} \\
        \textbf{subjuntivo} & $\Rightarrow$ & \textbf{geringere} \\
    \end{tabular}
    \vskip20pt
    Anstelle von \textbf{quizá/s} und \textbf{tal vez} kann \textbf{a lo mejor} verwendet werden,
    aber immer mit dem Indikativ.
    \vskip20pt
    \coloredbox[]{
        \textbf{Quizás} es mejor así. \\
        \textbf{Quizás} Marta tenga razón. \\
        \textbf{Tal vez} va a llover hoy. \\
        \textbf{Tal vez} vaya al cine esta noche. \\
        \textbf{A lo mejor} hago  un pastel. \\
        \textbf{A lo mejor} visito a mi padre. \\
    }
}
\column{0.5}
\block{aunque}{
    \Large
    Die Konjunktion \textbf{aunque} hat zwei Bedeutungen. \\

    \begin{tabular}{llcl}
    \multirow{2}{65mm}{\textbf{aunque} +} & \textbf{Indikativ} & $\Rightarrow$ & \textbf{obwohl} \\
    & \textbf{subjuntivo} & $\Rightarrow$ & \textbf{selbst wenn} \\
    \end{tabular}
    \vskip20pt
    \coloredbox[]{
        \textbf{Aunque} gano poco,  me gusta mi trabajo. \\
        \textbf{Aunque} no gane nada,  quiero trabajar aquí.
    }

}
\end{columns}

\begin{columns}
\column{0.5}
\block{Meinungsäußerungen}{
    \Large
    Nach Meinungsäußerungen benutzt man den Indikativ. \\

    \begin{tabular}{lll}
    \textbf{creer que} &
    \multirow{5}{15mm}{$\left.\rule{0pt}{110pt} \right\}$} &
    \multirow{5}{25mm}{\textbf{Indikativ}} \\
    \textbf{pensar que}\\
    \textbf{opinar que}\\
    \textbf{estar seguro que}\\
    \textbf{parecer que}\\
    \end{tabular}

    \vskip20pt

    \coloredbox[]{
        Creo que los animales se \textbf{aburren} en el zoo. \\
        Pienso que los zoológicos \textbf{hacen} un buen trabajo.\\
        Me parece que los animales se \textbf{aburren}.
    }
}
\column{0.5}
\block{Verneinte Meinungsäußerungen}{
    \Large
    Nach verneinten Meinungsäußerungen benutzt man den subjuntivo. \\

    \begin{tabular}{lll}
    \textbf{no creer que} &
    \multirow{5}{15mm}{$\left.\rule{0pt}{110pt} \right\}$} &
    \multirow{5}{25mm}{\textbf{subjuntivo}} \\
    \textbf{no pensar que}\\
    \textbf{no opinar que} \\
    \textbf{no estar seguro que}\\
    \textbf{no parecer que}\\
    \end{tabular}

    \vskip20pt

    \coloredbox[]{
        No creo que eso \textbf{esté} bien.\\
        No estoy seguro de que  \textbf{pueda} viajar.\\
        No me parece que Nerón \textbf{sea} un perro feliz.
    }
}
\end{columns}
%-------------------------------------------------------------------
\begin{columns}
\column{0.35}
\block{Ausdrücke nur mit subjuntivo}{
    \Large
    \begin{tabular}{lcl}
    \textbf{a menos que} & $\Rightarrow$ & es sei denn \\
    \textbf{en (el) caso de que} & $\Rightarrow$ & falls \\
    \textbf{antes de que} & $\Rightarrow$ & bevor
    \end{tabular}
}
\block{importar, dar, igual, ser indiferente}{
    \Large
    Nebensätze, die durch \hl{importar}, \hl{dar igual}, \hl{ser indiferente}
    erfordern den \textbf{subjuntivo}.

    \vskip20pt
    \coloredbox{
        \textbf{No me importa} que Pillar no me \textbf{invite}.\\
        \textbf{¿Te da igual} el vendedor \textbf{sea} antipático?\\
        \textbf{Me es indiferente que} mi novia \hl{gane} más dinero que yo.
    }
}
\column{0.65}
\block{Temporale Konjunktionen}{
    \Large
    \begin{tabular}{llll}
    \textbf{cuando} & \multirow{6}{105mm}{$\left.\rule{0pt}{130pt}\right\}$~\parbox{6cm}{Nebensatz bezieht sich auf}~$\left\{\rule{0pt}{130pt}\right.$} & \multirow{3}{170mm}{\textbf{Vergangenheit, Gegenwart}} & \multirow{3}{90mm}{$\Rightarrow$ \textbf{Indikativ}} \\
    \textbf{hasta que} \\
    \textbf{tan pronto como} \\
    \textbf{después de que} & & \multirow{3}{170mm}{\textbf{Zukunft}} & \multirow{3}{90mm}{$\Rightarrow$ \textbf{subjuntivo}} \\
    \textbf{mientras} \\
    \textbf{siempre que} \\
    \end{tabular}
    \vskip40pt
    Unterschiedliche Übersetzungen je nach Modus
    \vskip20pt
    \begin{tabular}{lll}
    \multirow{2}{90mm}{\textbf{mientras} +}    & \textbf{Indikativ}  & $\Rightarrow$ ~ während \\
                                      & \textbf{subjuntivo} & $\Rightarrow$ ~ solange \\
    \\
    \multirow{2}{90mm}{\textbf{siempre que} +} & \textbf{Indikativ}  & $\Rightarrow$ ~ immer wenn \\
                                      & \textbf{subjuntivo} & $\Rightarrow$ ~ sofern, vorausgesetzt dass
    \end{tabular}
}
\end{columns}
%-----------------------------------------------------
\begin{columns}
\column{0.33333}
\block{Subjuntivo nach unpersönlichen Ausdrücken}{
    \Large
    \hl{es / está} + \textbf{Adjektiv} + \hl{que} \\
    \hl{está} + \textbf{Adverb } + \hl{que}

    \vskip20pt

    \coloredbox[]{
    \textbf{Es importante que} \textbf{aprendas} idiomas.\\
    \textbf{Está mal que} Luis no \textbf{hable} con sus padres.
    }

}
\column{0.33333}
\block{Indikativ wenn etwas sicher oder wahr ist}{
    \Large
    \hl{es verdad}, \hl{es cierto}, \hl{es seguro}, \hl{es evidente} \\
    \hl{está claro}, \hl{está demostrado que}

    \vskip20pt

    \coloredbox[]{
    \textbf{Es verdad que} Javier \textbf{trabaja} mucho. \\
    \textbf{Es seguro que} la empresa no \textbf{va} a cerrar.
    }.
}.
\column{0.33333}
\block{Subjuntivo bei Verneinung}{
    \Large
    \hl{no es verdad}, \hl{no es cierto}, \hl{no es seguro}, \hl{no es evidente} \\
    \hl{no está claro}, \hl{no está demostrado que}

    \vskip20pt

    \coloredbox[]{
    \textbf{No es verdad que} Javier \textbf{trabaje} mucho. \\
    \textbf{No es sueuro que} la empresa \textbf{vaya} a cerrar.
    }
}

\end{columns}

\end{document}








