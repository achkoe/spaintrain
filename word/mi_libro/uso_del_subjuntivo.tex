% we don't want ae.sty
\expandafter\def\csname ver@ae.sty\endcsname{}

\documentclass[25pt]{tikzposter} % See Section 3

\usepackage{fontspec}
\usepackage[utf8]{inputenc}
\usepackage[ngerman]{babel}

\title{Subjuntivo}
% \institute{Inst} % See Section 4.1
%\author{Auth} \titlegraphic{Logo}
%\usetheme{Basic} % See Section 5
\usecolorpalette{BlueGrayOrange}
\begin{document}
\maketitle % See Section 4.1


\begin{columns} % See Section 4.4
\column{0.5} % See Section 4.4
\block{quizá/s, tal vez, a lo mejor}{%
    Nach den Adverbien \textbf{quizá/s} und \textbf{tal vez} kann der Indikativ oder der subjuntivo stehen. \\
    Der \textbf{subjuntivo} drückt dabei die \textbf{geringere Wahrscheinlichkeit} aus.\\
    \textbf{A lo mejor} kann anstelle von \textbf{quizá/s} und \textbf{tal vez} verwendet werden,
    steht aber immer mit dem Indikativ. \\

    \coloredbox[]{
        \textbf{Quizás} es mejor así. \\
        \textbf{Quizás} Marta tenga razón. \\
        \textbf{Tal vez} va a llover hoy. \\
        \textbf{Tal vez} vaya al cine esta noche. \\
        \textbf{A lo mejor} hago  un pastel. \\
        \textbf{A lo mejor} visito a mi padre. \\
    }
}
\column{0.5}
\block{aunque}{
    Die Konjunktion \textbf{aunque} hat zwei Bedeutungen.

    \textbf{aunque} + Indikativ $\Rightarrow$ obwohl \\
    \textbf{aunque} + subjuntivo $\Rightarrow$ selbst wenn \\

    \coloredbox[]{
        \textbf{Aunque} gano poco,  me gusta mi trabajo. \\
        \textbf{Aunque} no gane nada,  quiero trabajar aquí.
    }

}
\end{columns}

\begin{columns}
\column{0.5}
\block{Meinungsäußerungen}{
    Nach Meinungsäußerungen benutzt man den Indikativ. \\

    \textbf{creer que} \\
    \textbf{pensar que}\\
    \textbf{opinar que} \rule{80pt}{0pt} + Indikativ  \\
    \textbf{estar seguro que}\\
    \textbf{parecer que}\\

    \coloredbox[]{
        Creo que los animales se \textbf{aburren} en el zoo. \\
        Pienso que los zoológicos \textbf{hacen} un buen trabajo.\\
        Me parece que los animales se \textbf{aburren}.
    }
}
\column{0.5}
\block{Verneinte Meinungsäußerungen}{
    Nach verneinten Meinungsäußerungen benutzt man den subjuntivo. \\

    \textbf{nu creer que} \\
    \textbf{no pensar que}\\
    \textbf{no opinar que} \rule{80pt}{0pt} + subjuntivo  \\
    \textbf{no estar seguro que}\\
    \textbf{no parecer que}\\

    \coloredbox[]{
        No creo que eso \textbf{esté} bien.\\
        No estoy seguro de que  \textbf{pueda} viajar.\\
        No me parece que Nerón \textbf{sea} un perro feliz.
    }
}
\end{columns}

\end{document}








