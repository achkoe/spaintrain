\documentclass[fontsize=11pt]{scrartcl}
\usepackage[textwidth=10cm, lmargin=2cm, rmargin=5cm ]{geometry}
\newcommand{\uproman}[1]{\uppercase\expandafter{\romannumeral#1}}
\begin{document}
\begin{tabular}{ll}
\uproman{1} & 1 \\
\uproman{2} & 2 \\
\uproman{3} & 3 \\
\uproman{4} & 4 = 5 - 1  \\
\uproman{5} & 5 \\
\uproman{6} & 6 = 5 + 1  \\
\uproman{7} & 7 = 5 + 2  \\
\uproman{8} & 8 = 5 + 3  \\
\uproman{9} & 9 = 10 - 1  \\
\uproman{10} & 10 \\
\uproman{11} & 11 = 10 + 1  \\
\uproman{12} & 12 = 10 + 2  \\
\uproman{13} & 13 = 10 + 3  \\
\uproman{14} & 14 = 10 + 5 - 1  \\
\uproman{15} & 15 = 10 + 5  \\
\uproman{16} & 16 = 10 + 5 + 1  \\
\uproman{17} & 17 = 10 + 5 + 2  \\
\uproman{18} & 18 = 10 + 5 + 3 \\
\uproman{19} & 19 = 10 + 10 - 1  \\
\uproman{20} & 20 = 10 + 10  \\
\uproman{21} & 21 = 10 + 10 + 1  \\
\end{tabular}

\end{document}
