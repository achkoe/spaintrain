\documentclass[a4paper, pagesize, DIV=15, fontsize=11pt, parskip=half, oneside, BCOR=5mm]{scrartcl}
\usepackage{hyperref}
\hypersetup{colorlinks, linkcolor=black, citecolor=blue, urlcolor=black}
\usepackage[utf8]{inputenc}
\usepackage[german]{babel}
\usepackage{graphicx}
\graphicspath{{images/}{tiff/}}
\usepackage{booktabs}
\setcounter{secnumdepth}{-1}
\usepackage{paralist} % needed for compact lists
\setdefaultleftmargin{2em}{1em}{1em}{1em}{1em}{1em}

\usepackage{xcolor}
\usepackage[normalem]{ulem} % needed by strike

%%% Custom LaTeX preamble
\usepackage{enumitem}
\setlist{nosep}
\renewcommand{\baselinestretch}{1}
\itemsep0pt
\renewcommand{\labelitemi}{$\triangleright$}

\usepackage{amsmath}

\newcommand{\ul}[1]{\underline{#1}}
\newcommand{\hl}[1]{\textbf{#1}}
\newcommand{\il}[1]{\textit{#1}}
\newcommand{\tb}[1]{\mbox{\scriptsize{#1}}}

\pagestyle{empty}


\begin{document}
\fbox{\parbox{0.96\textwidth}{
Objektpronomen ersetzen Objekte.
}}

\section*{Direkte Objektpronomen (Akkusativ)}

\fbox{\parbox{0.96\textwidth}{
Direktes Objekt = Akkusativobjekt auf die Frage ``wen?'' oder ``was?''
}}

\begin{tabular}{llll}
Compro \ul{el libro}.       & Ich kaufe \ul{das Buch}.   & Wen oder was kaufe ich? & das Buch   \\
Juan come \ul{espaguettis}. & Juan isst \ul{Spaghetti}.  & Wen oder was isst er?   & Spaghetti  \\
¿Conoces a \ul{Shakira}?    & Kennst du \ul{Shakira}?    & Wen oder was kennt er?  & Shakira    \\
\end{tabular}

\fbox{\parbox{0.96\textwidth}{
Direkte Objektpronomen sind \il{me}, \il{te}, \il{lo/la}, \il{nos}, \il{os}, \il{los/las} \\[0.5em]
Das direkte Objektpronomen steht \hl{vor} dem konjugierten Verb. \\
An einen Infinitiv oder ein Gerundium \hl{kann} es angehängt werden. \\
An den bejahten Imperativ wird es \hl{immer} angehängt.
}}


\begin{tabular}{ll}
Compro \ul{el libro}.                        & \ul{Lo} compro.                                                    \\
Queremos comprar \ul{el coche azul}.         & Queremos comprar\ul{lo}./\ul{Lo} queremos comprar.        \\
Alquilaron \ul{el piso}.                     & \ul{Lo} alquilaron.                                        \\
Vamos a pasar \ul{las vacaciones} en Italia. & \ul{Las} vamos a pasar en Italia. \\
~                                            & Vamos a pasar\ul{las} en Italia. \\
He comprado \ul{la revista} para mi madre.   & \ul{La} he comprado para mi madre.                               \\
No podemos comprar \ul{la case en el campo}. & No \ul{la} podemos comprar./No podemos comprar\ul{la}.       \\
¿Ya has devuelto \ul{los libros}?            & ¿Ya \ul{los} has devuelto?                                        \\
¿Por qué no tomáis \ul{la paella}?           & ¿Por qué no \ul{la} tomáis ?                                       \\ 
¡Abre \ul{la puerta}!                        & ¡Ábrela!                                                            \\
\end{tabular}

\vfill*
%------------------------------------------------------

\section*{Indirekte Objektpronomen (Dativ)}

\fbox{\parbox{0.96\textwidth}{
Indirektes Objekt = Dativobjekt auf die Frage ``wem?''
}}

\begin{tabular}{llll}
\ul{Te} escribo una carta.             & Ich schreibe \ul{dir} einen Brief.    & Wem schreibt man? & dir \\
¡Da\ul{me} tu diccionario!             & Gib \ul{mir} dein Wörterbuch!         & Wem wird gegeben? & mir \\
\ul{Le} mando una postal \ul{al jefe}. & Ich schicke \ul{dem Chef} eine Karte. & Wem schickt man?  & dem Chef \\
\end{tabular}

\fbox{\parbox{0.96\textwidth}{
Indirekte Objektpronomen sind \\ 
\hl{betont}: \il{a mí}, \il{a ti}, \il{a él/ella/usted}, \il{a nosotros/-as}, \il{a vosotros/-as}, \il{a ellos/ellas/ustedes} \\
\hl{unbetont}: \il{me}, \il{te}, \il{le}, \il{nos}, \il{os}, \il{les} 
\\[0.5em]
Das \hl{unbetonte} indirekte Objektpronomen steht \hl{vor} dem konjugierten Verb. \\
An einen Infinitiv oder ein Gerundium \hl{kann} es angehängt werden.\\
An den bejahten Imperativ wird es \hl{immer} angehängt.
}}

\fbox{\parbox{0.96\textwidth}{
\hl{Betonte} indirekte Objektpronomen zur Hervorhebung, stets mit der Präposition \hl{a}.
Steht es am Satzanfang, so muss das entsprechende unbetonte Objektpronomen folgen.
}}

\begin{tabular}{ll}
\ul{A mí} \ul{me} gustan mucho los perros.     & Ich mag Hunde sehr gern. \\
\ul{A Ana} \ul{le} gusta nadar.                & Ana schwimmt gerne.      \\
¿\ul{A quién} \ul{le} toca?                    & Wer ist der Nächste?     \\
\ul{A nosotros} \ul{nos} interesan las lenguas & Wir interessieren uns für  Sprachen.
\end{tabular}

\clearpage
%------------------------------------------------------

\section*{Direkte und indirekte unbetonte Objektpronomen}

\fbox{\parbox{0.96\textwidth}{
In einem Satz mit \hl{zwei unbetonten} Objektpronomen steht das das \hl{indirekte} (Dativ, ``wem?'')
vor dem \hl{direkten} (Akkusativ, ``wen?'', ``was?'') Objektpronomen.
}}

\begin{tabular}{ll}
Te presto 1000 euros. $\rightarrow$ \ul{Te} \ul{los} presto. & Ich leihe dir 1000 Euro. $\rightarrow$ Ich leihe sie dir.
\end{tabular}

\fbox{\parbox{0.96\textwidth}{
Das \hl{indirekte} (Dativ, ``wem?'') Objektpronomen der 3. Person -- \il{le} und \il{les} -- wird vor den 
\hl{direkten} (Akkusativ, ``wen?'', ``was?'')
Objektpronomen der 3. Person -- \il{lo}, \il{la}, \il{los},\il{las} -- zu \il{se}.
}}

\begin{align*}
& \underbrace{\mbox{Le}}_{\tb{indirekt}} 
\mbox{ doy } 
\underbrace{\mbox{las flores}}_{\tb{was? $\rightarrow$ direkt}}
\mbox{ } 
\underbrace{\mbox{a mi novia}}_{\tb{wem? $\rightarrow$ indirekt}}
\mbox{.} \\[1em]
& \underbrace{\mbox{Le}}_{\tb{falsch}} \mbox{ las doy.} \\[1em]
& \underbrace{\mbox{Se}}_{\tb{richtig}} \mbox{ las doy.}
\end{align*}

\fbox{\parbox{0.96\textwidth}{
An einen Infinitiv, ein Gerundium und an den bejahten Imperativ wird in der gleichen Reihenfolge 
angehängt.
}}

\begin{compactitem}
    \item No quiere prestar(les) el coche a sus amigos. $\rightarrow$ No \hl{se lo} quiere prestar. / No quiere prestár\hl{selo}.
    \begin{compactitem}\itshape
        \item Er will seinen Freunden das Auto nicht leihen. $\rightarrow$ Er will es ihnen nicht leihen.
    \end{compactitem}
    \item Te estoy escribiendo una carta. $\rightarrow$ \hl{Te la} estoy escribiendo. / Estoy escribiéndo\hl{tela}.
    \begin{compactitem}\itshape
        \item Ich schreibe dir gerade einen Brief. $\rightarrow$ Ich schreibe ihn dir gerade.
    \end{compactitem}
    \item ¡Dá\hl{melo}!
    \begin{compactitem}\itshape
        \item Gib es mir!
    \end{compactitem}
    \item ¡Díga\hl{selo}!
    \begin{compactitem}\itshape
        \item Sagen Sie es ihm/ihr/ihnen.
    \end{compactitem}
\end{compactitem}


\end{document}
